\NeedsTeXFormat{LaTeX2e}
\documentclass[10pt,a4paper,twoside]{scrartcl}

\usepackage{amsmath}
\usepackage{amssymb}
\usepackage{amsfonts}
\usepackage{amscd}
\usepackage[amsthm,thmmarks]{ntheorem}
%%\usepackage{TheoremCollection}
\usepackage{accents}
\usepackage{bbm} % BlackBoeard letters
%\usepackage[bbgreekl]{MyBbol} % Other Doublestroke Charakters
\usepackage{bbold}
\usepackage{fancyhdr} %
\usepackage{a4wide} %
\usepackage[small]{caption} %
\usepackage{makeidx} %
\usepackage{fleqn} 
\usepackage{indentfirst} 
\usepackage{cancel} % \cancel{stuff} provides slashed stuff
\usepackage{graphicx} % \including PostScript
\usepackage{color}
\usepackage{braket} %\bra{stuff} -> <stuff|
\usepackage{pstricks}
\usepackage{pst-node,pst-plot}
\usepackage{nomencl} %Nomenklatur
\usepackage{mathrsfs}   % Schoenes Lagrange -L mit \mathscr{L}
\usepackage{pifont} % dinbgbats
\usepackage[small]{subfigure}
\usepackage{longtable}
\usepackage[plain]{fancyref}
\usepackage{cite}
%\usepackage[bbgreekl]{MyBbol}  % bb-Symbole auch griechisch
\usepackage[sort&compress,numbers,colon]{natbib}
\bibliographystyle{apsrev}

%-- page parameters -------------------------------------------------

\pagestyle{fancyplain}

\advance \headheight by 3.0truept       % for 12pt mandatory...
\lhead[\fancyplain{}{\thepage}]{\fancyplain{}{\rightmark}}
\rhead[\fancyplain{}{\leftmark}]{\fancyplain{\thepage}{\thepage}}
\cfoot{}

%\addtolength{\oddsidemargin}{1.0truecm}
%\addtolength{\evensidemargin}{-0.3truecm}
\setlength{\parindent}{0cm}

%-- end of page parameters ------------------------------------------


%\makeindex
%\makeglossary

% \newcommand{\WeightConnect}[4]{\ncline{->}{#1}{#2}\mput*{\ovalnode{#3}{#4}}
% \ncline{-}{#1}{#3}\ncline{->}{#3}{#2}}
% 
% \def\Nf{i}
% \def\Ng{j}
% \newcommand{\GroupIndex}[1]{\ifcase#1\or \or a\or b\or c\or d\or e\or f\or g\or h\or i\or j\or
% k\or l\or m\or n\or o\or p\or q\or r\or s\or t\or u\or v\or w\or x\or
% y\or z\else\@ctrerr\fi}
% \newcommand{\FamilyIndex}[1]{\ifcase#1\or \or f\or g\or h\or i\or j\or
% k\or l\or m\or n\or o\or p\or q\or r\or s\or t\or u\or v\or w\or x\or
% y\or z\else\@ctrerr\fi}
\def\chargec{\mathrm{C}}        
\def\ChargeC{\mathrm{C}}        
\def\NuMSSM{{$\nu$MSSM}\ }
\newcommand{\SimpleRoot}[1]{\alpha^{(#1)}}
\newcommand{\FundamentalWeight}[1]{\mu^{(#1)}}
\newcommand{\ChargeConjugate}[1]{#1^\chargec}
\newcommand{\SFConjugate}[1]{#1^\chargec}
\newcommand{\CenterFmg}[1]{\ensuremath{\vcenter{\hbox{\input{#1.fmg}}}}}
\newcommand{\CenterObject}[1]{\ensuremath{\vcenter{\hbox{#1}}}}
\newcommand{\CenterEps}[2][1]{\ensuremath{\vcenter{\hbox{\includegraphics[scale=#1]{#2.eps}}}}} % Input eps files - Usage: \CenterEps[ScaleFactor]{FileName}
\newcommand{\Commutator}[2]{{\left[ #1,#2\right]}_-}
\newcommand{\AntiCommutator}[2]{{\left\{ #1,#2\right\}}}
\newcommand{\RightHandedNeutrino}{\nu}
\newcommand{\SuperCommutator}[2]{\left\Lbracket #1,#2\right\Rbracket}
\newcommand{\SuperField}[1]{\bbsymbol{#1}}
\newcommand{\package}[1]{\texttt{#1}}
\newcommand{\function}[1]{\texttt{#1}}
\newcommand{\param}[1]{\texttt{#1}}
\newcommand{\option}[1]{\texttt{#1}}
\newcommand{\variable}[1]{\texttt{#1}}
\newcommand{\optparam}[1]{\texttt{\textit{#1}}}
\newcommand{\warning}[1]{\hspace{2ex} \textit{warning: #1}}
\newcommand{\internal}[1]{#1 \warning{This function is for internal use only.}}
\newcommand{\eV}{\ensuremath{\,\mathrm{eV}}}
\newcommand{\keV}{\ensuremath{\,\mathrm{keV}}}
\newcommand{\MeV}{\ensuremath{\,\mathrm{MeV}}}
\newcommand{\GeV}{\ensuremath{\,\mathrm{GeV}}}
\newcommand{\TeV}{\ensuremath{\,\mathrm{TeV}}}

\newenvironment{implementation}{\em}{\normalfont}

\DeclareMathOperator{\re}{Re}
\DeclareMathOperator{\im}{Im}
\DeclareMathOperator{\tr}{tr}
\DeclareMathOperator{\Tr}{Tr}
\DeclareMathOperator{\diag}{diag}
\DeclareMathOperator{\quabla}{\boldsymbol{\square}}
\DeclareMathOperator{\STr}{STr}
\DeclareMathOperator{\Li}{Li}
\DeclareMathOperator{\ad}{ad}
\DeclareMathOperator{\Ad}{Ad}
\DeclareMathOperator{\AD}{AD}
\DeclareMathOperator{\ind}{ind}
\DeclareMathOperator{\ch}{ch}
\DeclareMathOperator{\Pf}{Pf}
\DeclareMathOperator{\arcosh}{arcosh}

\def\D{\mathrm{d}}
\def\I{\mathrm{i}}
\def\Nf{f}
\def\Ng{g}
\def\PlusIEpsilon{}
\newcommand{\ChargeConjMatrix}{\mathsf{C}}      % Charge conjugation matrix
\newcommand{\ChargeConjOp}{\boldsymbol{\ChargeConjMatrix}}% cc. operator
\def\NuMSSM{{$\nu$MSSM}\ }
\def\secname{section}
\newif\ifappendix
\newcommand{\secref}[1]{%
        \ifappendix\appendixname~\ref{#1}\else\secname~\ref{#1}\fi
        }
\appendixfalse
\renewcommand{\thesection}{\arabic{section}}
\renewcommand{\thesubsection}{\arabic{section}.\arabic{subsection}}
%\renewcommand{\thesubsubsection}{\arabic{section}.\arabic{subsection}.\arabic{subsubsection}}
% 
% \renewcommand{\thesection}{\arabic{section}}
% \renewcommand{\thesection}{\arabic{section}.\arabic{section}}
% \renewcommand{\thesubsection}{(\roman{subsection})}

\newcommand*{\fancyrefsubseclabelprefix}{subsec}
\newcommand*{\subsecname}{subsection}
\fancyrefaddcaptions{english}{%
\newcommand*{\frefsubsecname}{subsection}
\newcommand*{\Frefsubsecname}{subsection}
}
\numberwithin{equation}{section}
\numberwithin{table}{section}
\renewcommand{\thetable}{\arabic{section}-\arabic{table}}
\renewcommand{\theequation}{\arabic{section}.\arabic{equation}}
\renewcommand{\labelenumi}{(\arabic{enumi})}


\definecolor{MyBlue}{rgb}{0.8,0.85,1}
\def\labelitemi{$\bullet$}
\def\labelitemii{--}
\unitlength=1mm

\allowdisplaybreaks[1]


\begin{document}
\font\TitleFont=cmbx10 at 40pt \font\SubTitleFont=cmbx10 at 25pt
\pagenumbering{arabic} \title{\TitleFont{REAP 1.11.5
}\\[1cm]\SubTitleFont{Users guide} } \author{S.~Antusch, J.~Kersten,
M.~Lindner, M.~Ratz and M.A.~Schmidt} \maketitle
\begin{abstract}
  This is an users guide of the \package{REAP} add-on
  for Mathematica. We describe the functions which allow to calculate the
  evolution of the neutrino mass matrix in different models (SM, MSSM, 2HDM).
  There is a reference of the most important functions. Besides this function
  reference there is a description of the installation procedure, a short HowTo and a section about frequently asked questions.
\end{abstract}
\thispagestyle{empty}
\tableofcontents
\clearpage

%%%%%%%%%%%%%%%%%%%%%%%%%%%%%%%%%%%%%%%%%%%%%%%%%%%%%%%%%%%%%%%%%%%%%%%%%%%%%%%%%%%%%%%%%%%%%%%%%%%%%%%%%%%%%%%%%%%%%%%%%%%%%%
\vspace{3ex}

\begin{center}
  \noindent\fbox{
  \begin{minipage}{0.9\textwidth}
The package \package{REAP} is written for Mathematica 5 and is distributed under the terms of GNU Public License http://www.gnu.org/copyleft/gpl.html
\end{minipage}
}
\end{center}

\vspace{3ex}

\endinput

\section{Introduction}

The \package{REAP} (\textbf{R}enormalization group \textbf{E}volution of \textbf{A}ngles
and \textbf{P}hases) package is a Mathematica package to solve the
renormalization group equations (RGE) of the quantities relevant for neutrino
masses, for example the dimension 5 neutrino mass operator, the Yukawa matrices
and the gauge couplings.  So far, the $\beta$-functions for the standard model
(SM), the minimal supersymmetric standard model (MSSM) and the two higgs doublet
model with $\mathbb{Z}_2$ symmetry (2HDM) with and without right-handed
neutrinos are implemented.  Heavy degrees of freedom such as singlet neutrinos
can be integrated out automatically at the correct mass thresholds which are
determined by a fixed-point iteration. Thus the evolution is described by
several effective theories.
In addition all models are implemented with Dirac neutrinos.  By means of the
\package{MixingParameterTools} package, the calculated running of the neutrino
mass matrix can be translated into the running of the mixing parameters and the
mass eigenvalues.


If you would like to refer to REAP in a publication or talk, please cite
the accompanying paper hep-ph/0501272.


\endinput


%\begin{center}
% \fbox{\textbf{Note}: The documentation for the mixing parameter tools is in a
% separate directory!}
%\end{center}
%\input{MixingParameterTools.tex}

\section{Installation}

\subsection{Automatic Installation under UNIX/Linux}

Execute REAP.installer and you are done.
\begin{verbatim}
sh REAP.installer
\end{verbatim}
After the execution the Mathematica packages are copied to
\verb+~/.Mathematica/Applications/REAP+ and the documentation and notebooks are in a
subdirectory of the working directory, which is called REAP.

In addition, you have to install the package MixingParameterTools. There is also
a script which installs both REAP and MPT at the same time:
REAP\_MPT.installer.


\subsection{Semi-Automatic Installation under UNIX/Linux}

Unpack the archive REAP.tar.gz.
\begin{verbatim}
tar -xvzf REAP.tar.gz
\end{verbatim}
Then go to the directory REAPInstall and execute the script install.sh.
\begin{verbatim}
cd REAPInstall
sh ./install.sh
\end{verbatim}
The script copies the Mathematica packages to
\verb+~/.Mathematica/Applications/REAP+.  
The documentation and some sample notebooks are placed in a new
subdirectory of the working directory called REAP.  Hence, the folder
REAPInstall can be deleted now.

In addition, you have to install the package MixingParameterTools.
REAP and MPT can be installed simultaneously by using the archive
REAP\_MPT.tar.gz.  The procedure is completely analogous to the one
described above.


\subsection{Installation by Hand}

In order to install the package(s) manually, unpack the archive
REAP.tar.gz first.  Under UNIX/Linux, type
\begin{verbatim}
tar -xvzf REAP.tar.gz
\end{verbatim}
On Windows systems, a program like WinZip can be used.
Then move the directory REAP from the folder
REAPInstall to the directory
where the Mathematica add-ons are located, e.g.\
\begin{verbatim}
mv REAPInstall/REAP ~/.Mathematica/Applications/
\end{verbatim}
under UNIX/Linux. 
Under Windows XP, the path to the add-on directory should be something
like \verb+Application Data\Mathematica\Applications+.
The documentation and some sample notebooks can be found in
REAPInstall/Doc/REAP/.

In addition, you have to install the package MixingParameterTools.
To install both REAP and MPT at the same time, you can use the archive
REAP\_MPT.tar.gz.  The procedure is the same as above (except that the 
installation directory is called REAP\_MPTInstall now), supplemented by
an analogous step for moving the MPT directory, e.g.
\begin{verbatim}
mv REAP_MPTInstall/MixingParameterTools ~/.Mathematica/Applications/
\end{verbatim}


\endinput


%\section{Solving the RG equation for the Neutrino mass matrix}
\section{First Steps\label{sec:FirstSteps}}

The following simple example demonstrates how to calculate the RG
evolution of the neutrino mass matrix in the MSSM extended by three
heavy singlet neutrinos.

\begin{enumerate}

\item The package corresponding to the model at the highest energy has
to be loaded.  All other packages needed in the course of the
calculation are loaded automatically.
\begin{verbatim}
  Needs["REAP`RGEMSSM`"]
\end{verbatim}
Note that ` is the backquote, which is used in opening quotation marks,
for example.

\item Next, we specify that we would like to use the MSSM with singlet
neutrinos:
\begin{verbatim}
  RGEAdd["MSSM"]
\end{verbatim}

%the model has to be defined. \function{RGEAddEFT[\param{model name},\param{cutoff},\optparam{options}]} adds a new part to the model. The parameters are the cutoff, the maximum valid energy, the name of the used model (e.g. ``SM'') and several options of the model (e.g. RGEIntegratedOut, which is a list of righthanded Neutrinos which are integrated out at that specific scale).
%The SM and the MSSM have a function implemented to search transitions, which are induced by integrating out heavy neutrinos.
%So you don't have to define the transitions by hand.
%This option is controlled by the option $\mathrm{SearchTransition}\rightarrow\mathrm{True/False}$.
%\begin{verbatim}
%  RGEAddEFT["SM",10^15];
%\end{verbatim}
%adds the SM as an effective model with cutoff $10^{15}$ GeV. The default options of the model are used.
%\begin{verbatim}
%  RGEAddEFT["SM",10^4,RGEIntegratedOut->1];
%\end{verbatim}
%adds the SM with cutoff $10^4$ GeV, but one right-handed neutrino is already integrated out.

\item Now we have to provide the initial values.  Here we use the
default values of the package (see Sec.~\ref{sec:Models} for details) and a
simple diagonal pattern for the neutrino Yukawa matrix.
%The function \function{RGESetInitial[\param{scale},\param{list of initial values}]} takes as parameters the scale where the initial values are given and a list of the initial values.
% The specific format of the list of initial values depends on the model used at that scale. (see \ref{models}).
%
%A model can give suggestions of initial values by the function \function{RGESuggestInitialValues[\param{model name},\optparam{name of suggestion}]}
\begin{verbatim}
  RGESetInitial[2*10^16,RGEY\[Nu]->{{1,0,0},{0,0.5,0},{0,0,0.1}}]
\end{verbatim}

\item \function{RGESolve[\param{low},\param{high}]} solves the RGEs
between the energy scales low and high.  The heavy singlets are
integrated out automatically at their mass thresholds.
%
%\function{RGESolve[\param{down},\param{up},\optparam{options}]}
%RGESolve accepts the same options as \function{NDSolve}.
\begin{verbatim}
  RGESolve[100,2*10^16]
\end{verbatim}

\item Using \function{RGEGetSolution[\param{scale},\optparam{quantity}]}
we can query the value of the quantity given in the second argument at
the energy given in the first one.  For example, this returns the mass
matrix of the light neutrinos at $100\GeV$:
 %returns the solution at a specific scale. The second parameter of RGEGetSolution specifies the output format of RGEGetSolution (e.g. RGEM$\nu$ returns the Neutrino mass, RGEMu returns the mass matrix of the up-type quarks,..., the standard behavior is to return the whole solution)
\begin{verbatim}
  MatrixForm[RGEGetSolution[100,RGEM\[Nu]]]
\end{verbatim}
%returns the light neutrino mass matrix $M\nu$ and 
%\begin{verbatim}
%  RGEGetSolution[100];
%\end{verbatim}
%returns all parameters.

\item To find the leptonic mass parameters, we use the function
\function{MNSParameters[\param{$m_\nu$},\param{$Y_e$}]} (which also
needs the Yukawa matrix of the charged leptons).  The results are given
in the order
$\{\{\theta_{12},\theta_{13},\theta_{23},\delta,\delta_e,\delta_\mu,
\delta_\tau,\varphi_1,\varphi_2\},
\{m_1,m_2,m_3\},\{y_e,y_\mu,y_\tau\}\}$.
\begin{verbatim}
  MNSParameters[RGEGetSolution[100,RGEM\[Nu]],RGEGetSolution[100,RGEYe]]
\end{verbatim}

\item Finally, we can plot the running of the mixing angles:
\begin{verbatim}
  Needs["Graphics`Graphics`"]
  mNu[x_]:=RGEGetSolution[x,RGEM\[Nu]]
  Ye[x_]:=RGEGetSolution[x,RGEYe]
  \[Theta]12[x_]:=MNSParameters[mNu[x],Ye[x]][[1,1]]
  \[Theta]13[x_]:=MNSParameters[mNu[x],Ye[x]][[1,2]]
  \[Theta]23[x_]:=MNSParameters[mNu[x],Ye[x]][[1,3]]
  LogLinearPlot[{\[Theta]12[x],\[Theta]13[x],\[Theta]23[x]},{x,100,2*10^16}]
\end{verbatim}
To produce nicer plots, the notebook RGEPlots.nb, which is included in
the package, can be used.

\end{enumerate}


In a second run, let us try some more modifications of the defaults.
For example, model parameters can be changed by including a command
after step (2):
\begin{verbatim}
  RGESetOptions["MSSM",RGEtan\[Beta]->20]
\end{verbatim}

Furthermore, we set the SUSY breaking scale to 200 GeV and use the SM
as an effective theory below this scale.
\begin{verbatim}
  RGEAdd["SM",RGECutoff->200]
\end{verbatim}

The initial values of the neutrino mass parameters can be changed by
adding replacement rules in step (3).  For instance, to set the
GUT-scale value of $\theta_{13}$ to $6^\circ$ and the Majorana phases to
$50^\circ$ and $120^\circ$:
\begin{verbatim}
  RGESetInitial[2*10^16,
    RGEY\[Nu]->{{1,0,0},{0,0.5,0},{0,0,0.1}},RGE\[Theta]13->6 Degree,
    RGE\[CurlyPhi]1->50 Degree,RGE\[CurlyPhi]2->120 Degree]
\end{verbatim}
The results of the RG evolution with these parameters are now obtained
by repeating the above steps (4)--(7).

%\begin{verbatim}
%{g1,g2,g3,Yu,Yd,Ye,Y\[Nu],\[Kappa],M,\[Lambda]}=
%       RGEGetParameters["SM"]/.RGESuggestInitialValues["SM","GUT"]
%       /.{RGEMassHierarchy -> "r", RGE\[Theta]12 -> 45, RGE\[Theta]13 -> 0,
%         RGE\[Theta]23 -> 45, RGEDeltaCP -> 0, RGE\[Phi]1 -> 0, RGE\[Phi]2 -> 0, 
%    RGEMlightest -> 0.05, RGE\[Delta]atm -> 2.5*10^-3, 
%    RGE\[Delta]sol -> 7.3*10^-5, RGE\[Tau]\[Nu] -> 0.25, RGE\[Nu]Ratio -> 2};
%RGESetInitial[RGEInitialScale/.RGESuggestInitialValues["SM","GUT"],
%    {g1,g2,g3,Yu,Yd,Ye,Y\[Nu],\[Kappa],M,\[Lambda]}];
%\end{verbatim}

\endinput



\section{Reference}

\subsection{Implementation details}

\package{REAP} is divided in three parts. The main part
is \package{RGESolver} which
provides a standard interface between the different models and the user. Thus
the user does not have to know anything about the implementation details of the
different models besides the parameters of the models. 
The second part are the different models, like \package{RGESM},
\package{RGEMSSM}, \dots which contain the model
specific parts of the package. 
The third part is formed by some utility packages
(\package{RGEUtilities}, \package{RGEParameters}, 
\package{RGEInitial}, \package{RGEFusaokaYukawa}, \package{RGESymbol}, \package{RGETakagi}) which provide several useful functions
to the different models. In principle, a user only needs a limited set of
functions of \package{RGESolver}.



%% The package \package{REAP} is divided in three parts. The main part is the Mathematica File \package{REAP`RGESolver`} which provides a standard interface to all model. The second part are the different models, like \package{REAP`RGESM`}, \dots ,which contain the model specific parts of the package. The third part is formed by some utility packages (\package{REAP`RGEUtilities`}, \package{REAP`RGEParameters`}, \package{REAP`RGESymbol`}, \package{REAP`RGEInitial`}, \package{REAP`RGEFusaoka`}, \package{REAP`RGETakagi`}) which provide several useful functions to the different models.

%% The \package{RGESolver} provides the interface between the user and the different models. Every model has to register with \function{RegisterModel}.
%% First the model name is saved in the list named Model. The other parameters are stored in similar lists.
%% The user has to tell to \package{RGESolver} how his model does look like. He adds several EFT's to his model by \function{RGEAddEFT}. These EFT's are stored in the list \variable{RGEModel}.


\subsubsection{RGEAdd}

\function{RGEAdd[\param{model},\optparam{options}]} 
specifies that \param{model} should be used as an effective theory (EFT)
up to a cutoff energy given in the \optparam{options}.  If no cutoff is
given, $\infty$ is used.  \optparam{options} can also be used to specify
various parameters such as $\tan\beta$.  See Sec.~\ref{sec:Models} for a
complete list of the models and options available.

\begin{verbatim}
  RGEAdd["MSSM",RGEtan\[Beta]->50]
  RGEAdd["SM",RGECutoff->10^3]
\end{verbatim}
In this case, the MSSM with $\tan\beta=50$ is used at high energies.
Below $10^3\GeV$ (the SUSY breaking scale in this example), the SM is
used as an EFT.



\subsubsection{RGEGetOptions}

\function{RGEGetOptions[\param{model}]} returns the options set by
\function{RGEAdd} or \function{RGESetOptions} for the EFT \param{model}.
Wildcards can be used in \param{model}.
 
  \begin{verbatim}
   RGEGetOptions["SM*"]
 \end{verbatim}
 This returns the options which are currently set for all EFTs whose
 names start with ``SM''.
 

\subsubsection{RGEGetParameters}

\function{RGEGetParameters[\param{model}]} returns the quantities that run
in the \param{model}.
% 
%  \begin{verbatim}
%   RGEGetParameters["SM"];
% \end{verbatim}
% This returns the parameters of the ``SM''.
% 

\subsubsection{RGEGetSolution}

\label{sec:RGEGetSolution}
\function{RGEGetSolution[\param{scale},\optparam{parameter}]} 
returns the solution of the RGEs at the energy \param{scale}.
The \optparam{parameter} (optional) specifies the quantity
of interest (cf.\ Sec.~\ref{sec:Models} for the lists for each model).
If no \optparam{parameter} is given, the values of all running
quantities are returned.


\begin{verbatim}
  RGEGetSolution[100,RGEM\[Nu]]
\end{verbatim}
returns the neutrino mass matrix at $100\GeV$.
\begin{verbatim}
  RGEGetSolution[100]
\end{verbatim}
returns all parameters at $100\GeV$.



\subsubsection{RGEGetTransitions}

\function{RGEGetTransitions[]} returns the transitions (thresholds)
between the various EFTs in a list
containing the energy scale, the model name and its options.
%
%\begin{verbatim}
%  Print[RGEGetTransitions[]];
%  {{10000000,MSSM,{}},{10000,MSSM,{RGEIntegratedOut->1}}}
%\end{verbatim}
%


\subsubsection{RGEReset}

\function{RGEReset[]} removes all EFTs and resets all options which have
been changed by \function{RGEAdd} or \function{RGESetOptions}
%functions named RGE*EFT* 
to their default values. 

\subsubsection{RGESetInitial}

\function{RGESetInitial[\param{scale},\param{initial conditions}]} sets
the initial values at the energy \param{scale}.  They are entered as
replacement rules and can also contain options (e.g.\ to select the
neutrino mass hierarchy).  See Sec.~\ref{sec:Models} for the names of
the variables and options in the different models.

\begin{verbatim}
  RGESetInitial[10^16,RGE\[Theta]13->4 Degree,RGEMlightest->0.1]
\end{verbatim}
This sets the initial values at $10^{16}\GeV$.  The mixing angle
$\theta_{13}$ is set to $4^\circ$, and the mass of the lightest neutrino
to $0.1\eV$.  For the other parameters, the default values are used.



\subsubsection{RGESetOptions}

\function{RGESetOptions[\param{model},\param{options}]} changes the options
of the EFTs defined by \function{RGEAdd} with name matching \param{model} to \param{options}. Metacharacters, like * and @, are
allowed in the name.

\begin{verbatim}
  RGESetOptions["MSSM",RGEtan\[Beta]->40]
\end{verbatim}
This sets $\tan\beta$ of the ``MSSM'' to 40.  The EFT must have been
added earlier by \function{RGEAdd["MSSM"]}.  The other options are
unchanged.



\subsubsection{RGESolve}

\function{RGESolve[\param{low},\param{high},\optparam{options}]} solves
the RGEs between the energies \param{low} and \param{high}. 

\begin{verbatim}
  RGESolve[100,10^15]
\end{verbatim}
This solves the RGEs between $100\GeV$ and $10^{15}\GeV$.
%using the method \variable{StiffnessSwitching} for the calculation.









  











%%%%%%%%%%%%%%%%%%%%%%%%%%%%%%%%%%%%%%%%%%%%%%%%%%%%%%%%%%%%%%%%%%%%%%%%%%%%%%%%%%%
%% models
%%%%%%%%%%%%%%%%%%%%%%%%%%%%%%%%%%%%%%%%%%%%%%%%%%%%%%%%%%%%%%%%%%%%%%%%%%%%%%%%%%%

\section{Models\label{sec:Models}}

%\subsection{\package{RGEToyModel}}
%This package contains a toy model for testing purposes.

\subsection{Standard Model (SM)}

\subsubsection[\package{RGESM}]{\package{REAP`RGESM`}}
This package contains the Standard Model extended by an arbitrary number of right-handed neutrinos (SM)
to 1 loop order.  It is possible to automatically find transitions where heavy
neutrinos are integrated out.  However, quarks are not integrated out.

\vspace{2ex}

Options used by \function{RGESetInitial}:

If the default values of all parameters are used, the resulting parameters will
be compatible to the experimental data at the Z boson mass. The number of right-handed neutrinos is given by the initial conditions. There
is no need to specify the number of neutrinos somewhere else.
\begin{itemize}
\item RGEM$\nu$r\ is the mass matrix of the right-handed neutrinos.
  If this parameter is specified, it also determines the light neutrino
  mass matrix via the see-saw formula (together with RGEY$\nu$).  Thus,
  RGEMassHierarchy, RGEMlightest, RGE$\Delta$m2atm, RGE$\Delta$m2sol,
  RGE$\varphi$1, RGE$\varphi$2, RGE$\delta$, RGE$\delta$e,
  RGE$\delta\mu$, RGE$\delta\tau$, RGE$\theta$12, RGE$\theta$13, and
  RGE$\theta$23 do not have any effect in this case.
  
\item RGEMassHierarchy\ is the hierarchy of the neutrino masses; "r" or "n"
  means normal hierarchy, "i" means inverted hierarchy (default: "r").
  
\item RGEMlightest \ is the mass of the lightest neutrino in eV (default: $\mathcal{O}(0.01)
  \eV$).
  
\item RGEY$\nu$\ is the neutrino Yukawa matrix in ``RL convention''. This option overrides the
  built-in Yukawa matrix, i.e.\ RGEY$\nu33$ and RGEY$\nu$Ratio do not have any
  effect. 
  
\item RGEY$\nu$33\ is the (3,3) entry in the neutrino Yukawa matrix at the GUT
  scale.
\item RGEY$\nu$Ratio\ determines the relative value of the neutrino Yukawa couplings.
\item RGEYd\ is the Yukawa matrix of the down-type quarks.
  If this parameter is given, RGEyd, RGEys, RGEyb, RGEq$\varphi$1,
  RGEq$\varphi$2, RGEq$\delta$, RGEq$\delta$e, RGEq$\delta\mu$,
  RGEq$\delta\tau$, RGEq$\theta$12, RGEq$\theta$13, and RGEq$\theta$23
  are ignored.
  
\item RGEYe\ is the charged lepton Yukawa matrix.
  If this parameter is given, RGEye, RGEy$\mu$ and RGEy$\tau$ are
  ignored.
  
\item RGEYu\ is the Yukawa matrix of the up-type quarks.
  If this parameter is given, RGEyu, RGEyc and RGEyt are ignored;
  it is recommended not to use RGEq$\varphi$1, RGEq$\varphi$2,
  RGEq$\delta$, RGEq$\delta$e, RGEq$\delta\mu$, RGEq$\delta\tau$,
  RGEq$\theta$12, RGEq$\theta$13, and RGEq$\theta$23 in this case, since
  they are not necessarily equal to the CKM mixing parameters.
\item RGE$\Delta$m2atm\ is the atmospheric mass squared difference (default: $ \mathcal{O}(10^{-3}) \eV^2$).
\item RGE$\Delta$m2sol\ is the solar mass squared difference (default:
  $\mathcal{O}(10^{-4}) \eV^2$).
\item RGE$\varphi$1\ and RGE$\varphi2$ are the Majorana CP phases $\varphi_1$ and $\varphi_2$ in radians (default: $0$).
  
\item RGE$\delta$\ is the Dirac CP phase $\delta$ in radians (default: $0$).
\item RGE$\delta$e, RGE$\delta\mu$ and RGE$\delta\tau$ are the unphysical phases $\delta_e$,
  $\delta_\mu$ and $\delta_\tau$ (default: $0$). 
\item RGE$\kappa$\ is the coupling of the dimension 5 neutrino mass operator.
  
\item RGE$\lambda$\ is the quartic Higgs self-coupling (default: 0.5).  We use the
  convention that the corresponding term in the Lagrangian is
  $-\frac{\lambda}{4} (\phi^\dagger \phi)^2$.
  
\item RGE$\theta$12, RGE$\theta13$ and RGE$\theta23$ are the angles $\theta_{12}$, $\theta_{13}$
and $\theta_{23}$ of the MNS matrix in radians. (default: $\theta_{13}=0$ and
$\theta_{23}=\frac{\pi}{4}$). The default of $\theta_{12}$ depends on the
model. It is chosen in such a way, that the parameters are compatible with the
experimental data. 
\item RGEg RGEg is the coupling constants of SU(5)
  
\item RGEg1, RGEg2 and RGEg3 are the coupling constants of U$(1)_\mathrm{Y}$,
  SU$(2)_\mathrm{L}$ and SU$(3)_\mathrm{C}$, respectively.  GUT charge
  normalization is used for $g_1$.
  
\item RGEm RGEm is the Higgs mass
  
\item RGEq$\varphi$1\ and RGEq$\varphi2$ are the unphysical phases $\varphi_1$ and $\varphi_2$ of the
 CKM matrix which correspond to the Majorana phases in the MNS matrix (default: $0$).
\item RGEq$\delta$\ is the Dirac CP phase $\delta$ of the CKM matrix.
\item RGEq$\delta$e, RGEq$\delta\mu$ and RGE$\delta\tau$ are the unphysical phases $\delta_e$,
$\delta_\mu$ and $\delta_\tau$ of the CKM matrix (default: $0$).
\item RGEq$\theta$12, RGEq$\theta13$ and RGEq$\theta23$ are the angles of the CKM matrix. 
\item RGEyd, RGEys and RGEyb are the Yukawa coupling of the down-type quarks $d$,
  $s$ and $b$.
\item RGEye, RGEy$\mu$ and RGEy$\tau$ are the Yukawa couplings of the charged
  leptons $e$, $\mu$ and $\tau$.
\item RGEyu, RGEyc and RGEyt are the Yukawa couplings of the up-type quarks $u$,
  $c$ and $t$.

\end{itemize}

Parameters accepted by \function{RGEGetSolution}:
\begin{itemize}
\item 
RGECoupling is used to get the coupling constants.
\item 
RGEGWCondition returns the Gildener Weinberg condition.
\item 
RGEGWConditions returns all Gildener Weinberg conditions.
\item 
RGEM$\nu$ is used to get the mass matrix of the left-handed neutrinos.
\item 
RGEM$\nu$r is the mass matrix of the right-handed neutrinos.
\item 
RGEMd is used to get the mass matrix of the down-type quarks.
\item 
RGEMe is used to get the mass matrix of the charged leptons.
\item 
RGEMu is used to get the mass matrix of the up-type quarks.
\item 
RGERawY$\Delta$ is used to get the Yukawa coupling matrix of the coupling to the Higgs triplet.
\item 
RGEAll returns all parameters of the model.
\item 
RGEVEVratio returns the squared ratio of $v_R$ over the EW symmetry breaking scale.
\item 
RGEVEVratios returns the squared ratio of $v_R$ over the EW symmetry breaking scale.
\item 
RGEY$\nu$ is used to get the Yukawa coupling matrix of the neutrinos.
\item 
RGEYd is used to get the Yukawa coupling matrix of the down-type quarks.
\item 
 RGEYe is used to get the Yukawa coupling matrix of the charged leptons.
\item 
RGEYu is used to get the Yukawa coupling matrix of the up-type quarks.
\item 
RGE$\alpha$ is used to get the fine structure constants.
\item 
RGE$\lambda$ is used to get the quartic Higgs self coupling.

\end{itemize}

\subsubsection[\package{RGESM0N}]{\package{REAP`RGESM0N`}}
This package contains the Standard Model without any right-handed neutrinos
(SM0N) to 1 loop order.

\vspace{2ex} It has the same parameters and options as \package{RGESM},
with the following exceptions:  The only
missing options are RGEIntegratedOut, RGESearchTransition, RGEThresholdFactor,
RGEPrecision and RGEMaxNumberIterations, which are used to control the process
of integrating out.  Besides, RGEM$\nu$r and RGEY$\nu$ are no parameters of
\function{RGESetInitial}, and RGE$\epsilon$Max, RGE$\epsilon$, RGEM1Tilde, RGERawM$\nu$r and RGERawY$\nu$ are not accepted as
parameters by \function{RGEGetSolution}.  \function{RGESetInitial} has
an additional option: RGESuggestion can be used to choose between
different sets of default values, ``GUT'' (default) and ``MZ''.  They
refer to typical parameter values at the GUT scale or at the $Z$ mass,
respectively.

\subsubsection[\package{RGESMDirac}]{\package{REAP`RGESMDirac`}}
This package contains the Standard Model with Dirac Neutrinos to 1 loop order.

\vspace{2ex} It has the same parameters and options as \package{RGESM},
with the following exceptions:  The only
missing options are RGEIntegratedOut, RGESearchTransition, RGEThresholdFactor,
RGEPrecision and RGEMaxNumberIterations, which are used to control the process
of integrating out.  In addition RGE$\kappa$ and RGEM$\nu$r are no parameters of
\function{RGESetInitial} and RGEMixingParameters, RGE$\epsilon$Max, RGE$\epsilon$, RGEM1Tilde, RGERawM$\nu$r, RGERawY$\nu$ and RGE$\kappa$ are not
accepted as parameters by \function{RGEGetSolution}.
\function{RGESetInitial} has an additional option: RGESuggestion can be
used to choose between different sets of default values, ``GUT''
(default) and ``MZ''.  They refer to typical parameter values at the GUT
scale or at the $Z$ mass, respectively.

%%%%%%%%%%%%%%%%%%%%%%%%%%%%%%%%%%%%%%%%%%%%%%%%%%%%%%%%%%%%%%%%%%%%%%%%%%%%%%%%%%%%%%%%%%%%%%%%%%%%%%%%%%%%%%%%%%%%%%%%%%%%%

\subsection{Minimal Supersymmetric Standard Model (MSSM)}

\subsubsection[\package{RGEMSSM}]{\package{REAP`RGEMSSM`}}
This package contains the Minimal Supersymmetric Standard Model extended by an arbitrary number of 
right-handed neutrinos (MSSM) to 1 and 2 loop order.

It is possible to automatically find transitions where heavy neutrinos are
integrated out.  But neither quarks are integrated out nor MSSM thresholds are
considered. 

Options:
\begin{itemize}
\item RGE$\Gamma$d\ parameterizes the finite supersymmetric threshold corrections
\begin{equation}
Y_d^\mathrm{SM} = Y_d^\mathrm{MSSM} (1 + \mathrm{RGE}\Gamma\mathrm{d}) * \cos(\beta)
\end{equation}
in the basis, in which $Y_u$ is diagonal and the left-handed mixing is entirely contained in $Y_d$. It is related to the notation in \cite{Blazek:1995nv}
\begin{equation}
\mathrm{RGE}\Gamma\mathrm{d} \equiv \epsilon (V_{CKM} \Gamma_D^\dagger V_{CKM}^\dagger +\Gamma_U^\dagger)
\end{equation}
with $\epsilon = \tan\beta/(16 \pi^2)$ and $\Gamma_{U,D}$ defines as in Eq.~(1) of Ref.~\cite{Blazek:1995nv}. 
\item RGE$\Gamma$e\ parameterizes the finite supersymmetric threshold corrections
\begin{equation}
Y_e^\mathrm{SM} = Y_e^\mathrm{MSSM} (1 + \mathrm{RGE}\Gamma\mathrm{e}) * \cos\beta
\end{equation}
in the basis, in which the Weinberg operator $\kappa$ is diagonal and the left-handed mixing is entirely contained in $Y_e$. It is defined in a similar way to RGE$\Gamma$d.
\item RGEtan$\beta$\ is the value of $\tan\beta=\frac{v_u}{v_d}$, the ratio of the 2
  Higgs vevs (default: 50).

\end{itemize}

Options used by \function{RGESetInitial}:

If the default values of all parameters are used, the resulting parameters will
be compatible to the experimental data at the Z boson mass. The number of right-handed neutrinos is given by the initial conditions. There
is no need to specify the number of neutrinos somewhere else.
\begin{itemize}
\item RGEM$\nu$r\ is the mass matrix of the right-handed neutrinos.
  If this parameter is specified, it also determines the light neutrino
  mass matrix via the see-saw formula (together with RGEY$\nu$).  Thus,
  RGEMassHierarchy, RGEMlightest, RGE$\Delta$m2atm, RGE$\Delta$m2sol,
  RGE$\varphi$1, RGE$\varphi$2, RGE$\delta$, RGE$\delta$e,
  RGE$\delta\mu$, RGE$\delta\tau$, RGE$\theta$12, RGE$\theta$13, and
  RGE$\theta$23 do not have any effect in this case.
  
\item RGEMassHierarchy\ is the hierarchy of the neutrino masses; "r" or "n"
  means normal hierarchy, "i" means inverted hierarchy (default: "r").
  
\item RGEMlightest \ is the mass of the lightest neutrino in eV (default: $\mathcal{O}(0.01)
  \eV$).
  
\item RGEY$\nu$\ is the neutrino Yukawa matrix in ``RL convention''. This option overrides the
  built-in Yukawa matrix, i.e.\ RGEY$\nu33$ and RGEY$\nu$Ratio do not have any
  effect. 
  
\item RGEY$\nu$33\ is the (3,3) entry in the neutrino Yukawa matrix at the GUT
  scale.
\item RGEY$\nu$Ratio\ determines the relative value of the neutrino Yukawa couplings.
\item RGEYd\ is the Yukawa matrix of the down-type quarks.
  If this parameter is given, RGEyd, RGEys, RGEyb, RGEq$\varphi$1,
  RGEq$\varphi$2, RGEq$\delta$, RGEq$\delta$e, RGEq$\delta\mu$,
  RGEq$\delta\tau$, RGEq$\theta$12, RGEq$\theta$13, and RGEq$\theta$23
  are ignored.
  
\item RGEYe\ is the charged lepton Yukawa matrix.
  If this parameter is given, RGEye, RGEy$\mu$ and RGEy$\tau$ are
  ignored.
  
\item RGEYu\ is the Yukawa matrix of the up-type quarks.
  If this parameter is given, RGEyu, RGEyc and RGEyt are ignored;
  it is recommended not to use RGEq$\varphi$1, RGEq$\varphi$2,
  RGEq$\delta$, RGEq$\delta$e, RGEq$\delta\mu$, RGEq$\delta\tau$,
  RGEq$\theta$12, RGEq$\theta$13, and RGEq$\theta$23 in this case, since
  they are not necessarily equal to the CKM mixing parameters.
\item RGE$\Delta$m2atm\ is the atmospheric mass squared difference (default: $ \mathcal{O}(10^{-3}) \eV^2$).
\item RGE$\Delta$m2sol\ is the solar mass squared difference (default:
  $\mathcal{O}(10^{-4}) \eV^2$).
\item RGE$\varphi$1\ and RGE$\varphi2$ are the Majorana CP phases $\varphi_1$ and $\varphi_2$ in radians (default: $0$).
  
\item RGE$\delta$\ is the Dirac CP phase $\delta$ in radians (default: $0$).
\item RGE$\delta$e, RGE$\delta\mu$ and RGE$\delta\tau$ are the unphysical phases $\delta_e$,
  $\delta_\mu$ and $\delta_\tau$ (default: $0$). 
\item RGE$\kappa$\ is the coupling of the dimension 5 neutrino mass operator.
  
\item RGE$\theta$12, RGE$\theta13$ and RGE$\theta23$ are the angles $\theta_{12}$, $\theta_{13}$
and $\theta_{23}$ of the MNS matrix in radians. (default: $\theta_{13}=0$ and
$\theta_{23}=\frac{\pi}{4}$). The default of $\theta_{12}$ depends on the
model. It is chosen in such a way, that the parameters are compatible with the
experimental data. 
\item RGEg RGEg is the coupling constants of SU(5)
  
\item RGEg1, RGEg2 and RGEg3 are the coupling constants of U$(1)_\mathrm{Y}$,
  SU$(2)_\mathrm{L}$ and SU$(3)_\mathrm{C}$, respectively.  GUT charge
  normalization is used for $g_1$.
  
\item RGEm RGEm is the Higgs mass
  
\item RGEq$\varphi$1\ and RGEq$\varphi2$ are the unphysical phases $\varphi_1$ and $\varphi_2$ of the
 CKM matrix which correspond to the Majorana phases in the MNS matrix (default: $0$).
\item RGEq$\delta$\ is the Dirac CP phase $\delta$ of the CKM matrix.
\item RGEq$\delta$e, RGEq$\delta\mu$ and RGE$\delta\tau$ are the unphysical phases $\delta_e$,
$\delta_\mu$ and $\delta_\tau$ of the CKM matrix (default: $0$).
\item RGEq$\theta$12, RGEq$\theta13$ and RGEq$\theta23$ are the angles of the CKM matrix. 
\item RGEyd, RGEys and RGEyb are the Yukawa coupling of the down-type quarks $d$,
  $s$ and $b$.
\item RGEye, RGEy$\mu$ and RGEy$\tau$ are the Yukawa couplings of the charged
  leptons $e$, $\mu$ and $\tau$.
\item RGEyu, RGEyc and RGEyt are the Yukawa couplings of the up-type quarks $u$,
  $c$ and $t$.

\end{itemize}

Parameters accepted by \function{RGEGetSolution}:
\begin{itemize}
\item 
RGECoupling is used to get the coupling constants.
\item 
RGEGWCondition returns the Gildener Weinberg condition.
\item 
RGEGWConditions returns all Gildener Weinberg conditions.
\item 
RGEM$\nu$ is used to get the mass matrix of the left-handed neutrinos.
\item 
RGEM$\nu$r is the mass matrix of the right-handed neutrinos.
\item 
RGEMd is used to get the mass matrix of the down-type quarks.
\item 
RGEMe is used to get the mass matrix of the charged leptons.
\item 
RGEMu is used to get the mass matrix of the up-type quarks.
\item 
RGERawY$\Delta$ is used to get the Yukawa coupling matrix of the coupling to the Higgs triplet.
\item 
RGEAll returns all parameters of the model.
\item 
RGEVEVratio returns the squared ratio of $v_R$ over the EW symmetry breaking scale.
\item 
RGEVEVratios returns the squared ratio of $v_R$ over the EW symmetry breaking scale.
\item 
RGEY$\nu$ is used to get the Yukawa coupling matrix of the neutrinos.
\item 
RGEYd is used to get the Yukawa coupling matrix of the down-type quarks.
\item 
 RGEYe is used to get the Yukawa coupling matrix of the charged leptons.
\item 
RGEYu is used to get the Yukawa coupling matrix of the up-type quarks.
\item 
RGE$\alpha$ is used to get the fine structure constants.
\item 
RGE$\kappa$ is used to get $\kappa$.

\end{itemize}

\subsubsection[\package{RGEMSSM0N}]{\package{REAP`RGEMSSM0N`}}
This package contains the Minimal Supersymmetric Standard Model (MSSM) without
any right-handed neutrinos to 1 and 2 loop order.

\vspace{2ex} It has the same parameter and options as \package{RGEMSSM}.  The
only missing options are RGEIntegratedOut, RGESearchTransition,
RGEThresholdFactor, RGEPrecision and RGEMaxNumberIterations, which are used to
control the process of integrating out.  In addition RGEM$\nu$r and RGEY$\nu$
are no parameters of \function{RGESetInitial} and RGE$\epsilon$Max,
RGE$\epsilon$, RGEM1Tilde, RGERawM$\nu$r and RGERawY$\nu$
are not accepted as parameters by \function{RGEGetSolution}.


\subsubsection[\package{RGEMSSMDirac}]{\package{REAP`RGEMSSMDirac`}}
This package contains the MSSM with Dirac Neutrinos to 1 loop order and 2 loop
order.

\vspace{2ex} It has the same parameter and options as \package{RGEMSSM}.  The
only missing options are RGEIntegratedOut, RGESearchTransition,
RGEThresholdFactor, RGEPrecision and RGEMaxNumberIterations, which are used to
control the process of integrating out.  In addition RGEM$\nu$r and RGE$\kappa$
are no parameter of \function{RGESetInitial} and RGEMixingParameters, RGE$\epsilon$Max,
RGE$\epsilon$, RGEM1Tilde, RGERawM$\nu$r, RGERawY$\nu$ and
RGE$\kappa$ are not accepted as parameters by \function{RGEGetSolution}.

\subsection{Two Higgs Doublet Model (2HDM)}

\subsubsection[\package{RGE2HDM}]{\package{REAP`RGE2HDM`}}
This package contains the Two Higgs Doublet Model (2HDM) with a $\mathbb{Z}_2$
symmetry extended by an arbitrary number of right-handed neutrinos. The charged leptons always couple to the first Higgs. In addition
there are right-handed neutrinos. The $\beta$-functions are to 1 loop order. The
vevs of the Higgs fields are $v_1=\braket{\phi_1}$ and
$v_2=\braket{\phi_2}$. They obey $v^2=v_1^2+v_2^2$, $v_1=v\cos\beta$ and
$v_2=v\sin\beta$, where v is the v.e.v. of the SM Higgs and $\beta$
($\tan\beta=\frac{v_2}{v_1}$, $\beta\in\left(0,\frac{\pi}{2}\right)$) is used to
parametrize the Higgs vevs.

Thus there are 2 dimension 5 operators which give mass to the light neutrinos.
\begin{equation*}
  \mathcal{L}_\kappa^{(ii)}=\frac{1}{4}\kappa_{gf}^{(ii)}\overline{l_{Lc}^C}^g\epsilon^{cd}\phi_d^{(i)}l_{Lb}^f\epsilon^{ba}\phi_a^{(i)}+\mathrm{h.c.}\quad\quad(i=1\;
  or\; 2)
\end{equation*}
The Higgs potential is
\begin{eqnarray*}
\mathcal{L}_{2Higgs}&=& -\frac{\lambda_1}{4}
\left(\phi^{(1)\dagger}\phi^{(1)}\right)^2 -\frac{\lambda_2}{4}
\left(\phi^{(2)\dagger}\phi^{(2)}\right)^2\\
&&-\lambda_3\left(\phi^{(1)\dagger}\phi^{(1)}\right)\left(\phi^{(2)\dagger}\phi^{(2)}\right)
-\lambda_4\left(\phi^{(1)\dagger}\phi^{(2)}\right)\left(\phi^{(2)\dagger}\phi^{(1)}\right)
\\ &&
-\left[\frac{\lambda_5}{4}\left(\phi^{(1)\dagger}\phi^{(2)}\right)^2+\mathrm{h.c.}\right]\\
\end{eqnarray*}

The charged leptons always couple to the first Higgs field and the coupling of
the other fields to the Higgs fields is determined by RGEModelOptions.

It is possible to automatically find transitions where heavy neutrinos are
integrated out.  But no other particles are integrated out.

Options:
\begin{itemize}
\item RGEtan$\beta$ is the value of $\tan\beta=\frac{v_2}{v_1}$, the ratio of the 2
  Higgs vevs (default: 50).
\item RGEz$\nu$\ is a list defining the Higgs the neutrinos are coupling to. If the $n^{th}$ component is one,
    the Higgs couples to the neutrinos. If it is 0, it won't
    couple (default: $\{0,1\}$). The charged leptons always couple to the
  first Higgs.
\item RGEzd\ is a list defining the Higgs the down-type quarks are coupling to. If the $n^{th}$ component is one,
    the Higgs couples to the down-type quarks. If it is 0, it won't
    couple (default: $\{1,0\}$).
\item RGEzu\ is a list defining the Higgs the up-type quarks are coupling to. If the $n^{th}$ component is one,
    the Higgs couples to the up-type quarks. If it is 0, it won't
    couple (default: $\{0,1\}$).

\end{itemize}

Options used by \function{RGESetInitial}:

If the default values of all parameters are used, the resulting parameters will
be compatible to the experimental data at the Z boson mass. The number of right-handed neutrinos is given by the initial conditions. There
is no need to specify the number of neutrinos somewhere else.
\begin{itemize}
\item RGEM$\nu$r\ is the mass matrix of the right-handed neutrinos.
  If this parameter is specified, it also determines the light neutrino
  mass matrix via the see-saw formula (together with RGEY$\nu$).  Thus,
  RGEMassHierarchy, RGEMlightest, RGE$\Delta$m2atm, RGE$\Delta$m2sol,
  RGE$\varphi$1, RGE$\varphi$2, RGE$\delta$, RGE$\delta$e,
  RGE$\delta\mu$, RGE$\delta\tau$, RGE$\theta$12, RGE$\theta$13, and
  RGE$\theta$23 do not have any effect in this case.
  
\item RGEMassHierarchy\ is the hierarchy of the neutrino masses; "r" or "n"
  means normal hierarchy, "i" means inverted hierarchy (default: "r").
  
\item RGEMlightest \ is the mass of the lightest neutrino in eV (default: $\mathcal{O}(0.01)
  \eV$).
  
\item RGEY$\nu$\ is the neutrino Yukawa matrix in ``RL convention''. This option overrides the
  built-in Yukawa matrix, i.e.\ RGEY$\nu33$ and RGEY$\nu$Ratio do not have any
  effect. 
  
\item RGEY$\nu$33\ is the (3,3) entry in the neutrino Yukawa matrix at the GUT
  scale.
\item RGEY$\nu$Ratio\ determines the relative value of the neutrino Yukawa couplings.
\item RGEYd\ is the Yukawa matrix of the down-type quarks.
  If this parameter is given, RGEyd, RGEys, RGEyb, RGEq$\varphi$1,
  RGEq$\varphi$2, RGEq$\delta$, RGEq$\delta$e, RGEq$\delta\mu$,
  RGEq$\delta\tau$, RGEq$\theta$12, RGEq$\theta$13, and RGEq$\theta$23
  are ignored.
  
\item RGEYe\ is the charged lepton Yukawa matrix.
  If this parameter is given, RGEye, RGEy$\mu$ and RGEy$\tau$ are
  ignored.
  
\item RGEYu\ is the Yukawa matrix of the up-type quarks.
  If this parameter is given, RGEyu, RGEyc and RGEyt are ignored;
  it is recommended not to use RGEq$\varphi$1, RGEq$\varphi$2,
  RGEq$\delta$, RGEq$\delta$e, RGEq$\delta\mu$, RGEq$\delta\tau$,
  RGEq$\theta$12, RGEq$\theta$13, and RGEq$\theta$23 in this case, since
  they are not necessarily equal to the CKM mixing parameters.
\item RGE$\Delta$m2atm\ is the atmospheric mass squared difference (default: $ \mathcal{O}(10^{-3}) \eV^2$).
\item RGE$\Delta$m2sol\ is the solar mass squared difference (default:
  $\mathcal{O}(10^{-4}) \eV^2$).
\item RGE$\varphi$1\ and RGE$\varphi2$ are the Majorana CP phases $\varphi_1$ and $\varphi_2$ in radians (default: $0$).
  
\item RGE$\delta$\ is the Dirac CP phase $\delta$ in radians (default: $0$).
\item RGE$\delta$e, RGE$\delta\mu$ and RGE$\delta\tau$ are the unphysical phases $\delta_e$,
  $\delta_\mu$ and $\delta_\tau$ (default: $0$). 
\item RGE$\kappa$1\ is the coupling of the dimension 5 operator associated
  with the first Higgs in the 2HDM.
  
\item RGE$\kappa$2\ is the coupling of the dimension 5 operator associated
  with the second Higgs in the 2HDM.
  
\item RGE$\lambda$1, RGE$\lambda2$, RGE$\lambda3$, RGE$\lambda4$ and RGE$\lambda5$ are the parameters
$\lambda_1$, $\lambda_2$, $\lambda_3$, $\lambda_4$ and $\lambda_5$ in the Higgs potential
(default: $\lambda_1=\lambda_2=0.75$, $\lambda_3=\lambda_4=0.2$, $\lambda_5=0.25$).
\item RGE$\theta$12, RGE$\theta13$ and RGE$\theta23$ are the angles $\theta_{12}$, $\theta_{13}$
and $\theta_{23}$ of the MNS matrix in radians. (default: $\theta_{13}=0$ and
$\theta_{23}=\frac{\pi}{4}$). The default of $\theta_{12}$ depends on the
model. It is chosen in such a way, that the parameters are compatible with the
experimental data. 
\item RGEg RGEg is the coupling constants of SU(5)
  
\item RGEg1, RGEg2 and RGEg3 are the coupling constants of U$(1)_\mathrm{Y}$,
  SU$(2)_\mathrm{L}$ and SU$(3)_\mathrm{C}$, respectively.  GUT charge
  normalization is used for $g_1$.
  
\item RGEm RGEm is the Higgs mass
  
\item RGEq$\varphi$1\ and RGEq$\varphi2$ are the unphysical phases $\varphi_1$ and $\varphi_2$ of the
 CKM matrix which correspond to the Majorana phases in the MNS matrix (default: $0$).
\item RGEq$\delta$\ is the Dirac CP phase $\delta$ of the CKM matrix.
\item RGEq$\delta$e, RGEq$\delta\mu$ and RGE$\delta\tau$ are the unphysical phases $\delta_e$,
$\delta_\mu$ and $\delta_\tau$ of the CKM matrix (default: $0$).
\item RGEq$\theta$12, RGEq$\theta13$ and RGEq$\theta23$ are the angles of the CKM matrix. 
\item RGEyd, RGEys and RGEyb are the Yukawa coupling of the down-type quarks $d$,
  $s$ and $b$.
\item RGEye, RGEy$\mu$ and RGEy$\tau$ are the Yukawa couplings of the charged
  leptons $e$, $\mu$ and $\tau$.
\item RGEyu, RGEyc and RGEyt are the Yukawa couplings of the up-type quarks $u$,
  $c$ and $t$.

\end{itemize}

Parameters accepted by \function{RGEGetSolution}:
\begin{itemize}
\item 
RGECoupling is used to get the coupling constants.
\item 
RGEGWCondition returns the Gildener Weinberg condition.
\item 
RGEGWConditions returns all Gildener Weinberg conditions.
\item 
RGEM$\nu$ is used to get the mass matrix of the left-handed neutrinos.
\item 
RGEM$\nu$r is the mass matrix of the right-handed neutrinos.
\item 
RGEMd is used to get the mass matrix of the down-type quarks.
\item 
RGEMe is used to get the mass matrix of the charged leptons.
\item 
RGEMu is used to get the mass matrix of the up-type quarks.
\item 
RGERawY$\Delta$ is used to get the Yukawa coupling matrix of the coupling to the Higgs triplet.
\item 
RGEAll returns all parameters of the model.
\item 
RGEVEVratio returns the squared ratio of $v_R$ over the EW symmetry breaking scale.
\item 
RGEVEVratios returns the squared ratio of $v_R$ over the EW symmetry breaking scale.
\item 
RGEY$\nu$ is used to get the Yukawa coupling matrix of the neutrinos.
\item 
RGEYd is used to get the Yukawa coupling matrix of the down-type quarks.
\item 
 RGEYe is used to get the Yukawa coupling matrix of the charged leptons.
\item 
RGEYu is used to get the Yukawa coupling matrix of the up-type quarks.
\item 
RGE$\alpha$ is used to get the fine structure constants.
\item 
RGE$\kappa$1 is the parameter of the dimension 5 operator associated
  with the first Higgs in the 2HDM.
\item 
RGE$\kappa$2 is the parameter of the dimension 5 operator associated
  with the second Higgs in the 2HDM.
  
\item 
RGE$\lambda$ is used to get the Higgs couplings.

\end{itemize}

  

\subsubsection[\package{RGE2HDM0N}]{\package{REAP`RGE2HDM0N`}}
This package contains the Two Higgs Doublet Model (2HDM) with a $\mathbb{Z}_2$
symmetry without right-handed neutrinos.

\vspace{2ex} It has the same parameters and options as
\package{RGE2HDM}, with the following exceptions:  The
only missing options are RGEIntegratedOut, RGESearchTransition,
RGEThresholdFactor, RGEPrecision and RGEMaxNumberIterations, which are used to
control the process of integrating out.  In addition RGEM$\nu$r and RGEY$\nu$
are no parameters of \function{RGESetInitial} and RGEM1Tilde, RGERawM$\nu$r and RGERawY$\nu$
are not accepted as parameters by \function{RGEGetSolution}.
\function{RGESetInitial} has an additional option: RGESuggestion can be
used to choose between different sets of default values, ``GUT''
(default) and ``MZ''.  They refer to typical parameter values at the GUT
scale or at the $Z$ mass, respectively.


\subsubsection[\package{RGE2HDMDirac}]{\package{REAP`RGE2HDMDirac`}}
This package contains the 2HDM with Dirac neutrinos to 1 loop order.

\vspace{2ex} It has the same parameters and options as
\package{RGE2HDM}, with the following exceptions: The
only missing options are RGEIntegratedOut, RGESearchTransition,
RGEThresholdFactor, RGEPrecision and RGEMaxNumberIterations, which are used to
control the process of integrating out.  In addition RGEM$\nu$r, RGE$\kappa$1
and RGE$\kappa$2 are no parameter of \function{RGESetInitial} and
RGEMixingParameters, RGEM1Tilde, RGERawM$\nu$r,
RGERawY$\nu$, RGE$\kappa$1 and RGE$\kappa$2 are not accepted as parameters by
\function{RGEGetSolution}.
\function{RGESetInitial} has an additional option: RGESuggestion can be
used to choose between different sets of default values, ``GUT''
(default) and ``MZ''.  They refer to typical parameter values at the GUT
scale or at the $Z$ mass, respectively.


%\input{RGEShortReference}
%\input{RGEModels}

\section{Frequently Asked Questions and their Answers (FAQ)}

\subsection{Physics Questions}

\subsubsection{How can I have more or less than 3 right-handed neutrinos?}

The default initial values have 3 right-handed neutrinos but you can define a
model with an arbitrary number of right-handed neutrinos by changing the initial
values for the right-handed neutrino mass matrix and the Yukawa couplings of the
neutrinos.

\subsubsection{How are neutrino mixing parameters defined above the
see-saw scale?}
In order to define mass and mixing parameters as functions of the
renormalization scale $\mu$ above the highest see-saw scale,
we consider the effective light neutrino mass matrix
\begin{equation} \label{eq:mnuFullTheory}
  m_\nu(\mu)\, =\, -\frac{v^2}{2} \, Y_\nu^T(\mu)\, M^{-1}(\mu)
\,Y_\nu(\mu) \;,
\end{equation}
where $Y_\nu$ and $M$ are $\mu$-dependent. (We do not take into account
the running of the Higgs vev.)
$m_\nu$ is the mass matrix of the three light neutrinos as obtained from
block-diagonalizing the complete $6\times6$ (for 3 singlet neutrinos) neutrino mass matrix,
following the standard see-saw calculation.
The energy-dependent mixing parameters are obtained from $m_\nu(\mu)$
and the running charged lepton Yukawa matrix $Y_e(\mu)$.
Between the see-saw scales or in a type II see-saw, there is an
additional contribution to $m_\nu$ from the dimension 5 neutrino mass
operator.

\subsubsection{How can I obtain the CP asymmetry for leptogenesis?}
The CP asymmetry in the case of thermal leptogenesis (in the limit $M_1 \ll
M_2,M_3$) is implemented as output
function in \package{REAP}. It can be obtained in the SM and MSSM by
\begin{verbatim}
RGEGetSolution[M_1, RGE\[Epsilon]1,1] .
\end{verbatim}
The CP asymmetry is not implemented for other leptogenesis scenarios. However,
the relevant quantities can be obtained via \function{RGEGetSolution}. See the
notebook \texttt{RGETemplate.nb} for an example.

\subsection{Implementation Details}

\subsubsection[``SM'' added, but \function{RGESetOptions} does not have any effect.]{I added the ``SM'' with
  \function{RGEAdd[``SM'',RGECutoff->1000]}, but
  \function{RGESetOptions[``SM'',RGE$\lambda$->0.3]} does not have any
  effect. Is this an error?}

This is no error, but sometimes the EFT's are changed in such a way that the whole model is
consistent. In this case the ``SM'' was changed to ``SM0N'', because all
right-handed neutrinos are integrated out above 1000 GeV.
You can use \function{RGESetOptions[``SM*'',RGE$\lambda$->0.3]} to change
\variable{RGE$\lambda$} in ``SM'' and ``SM0N'' at the same time.
Then you do not have to care whether all neutrinos have been integrated out.

\subsubsection{I want to change the Standard Model Higgs vev in all EFT's.}
You can use wildcards with \function{RGESetOptions},
\function{RGESetEFTOptions}, \function{RGESetModelOptions}, \\
\function{RGEGetOptions},
\function{RGEGetEFTOptions} and  \function{RGEGetModelOptions}, because the name
you enter is matched with \function{StringMatchQ}. See the documentation of
Mathematica for the possible wildcards.

\subsubsection{RGEAll does not work in ``*0N''}
\function{RGEGetSolution[Scale,RGEAll]} returns all parameters used in a see-saw
model. Thus, $Y_\nu$ and $M_{\nu_R}$ are returned in addition to the parameters
valid in a model without right-handed neutrinos. However, an error message will
be produced, unless there are right-handed neutrinos at a higher scale, because
\function{RGEGetSolution} obtains the values of $Y_\nu$ and $M_{\nu_R}$
recursively by determining the values at the cutoff. RGERaw which returns the
parameters valid in ``*0N'' can be used instead of RGEAll, if $Y_\nu$ and $M_{\nu_R}$ are not defined.

\subsubsection{Sometimes there are errors when RGESolve is executed twice.}

\function{RGESolve} adds new EFT's to the model. This is in conflict with the automatic detection of transitions.

The simplest workaround is either to set
\variable{RGERemoveAutoGeneratedEntries}, an option of \function{RGESolve}, to
`True' (This is the default value of this option.) or if this does not help, make sure that \function{RGEReset} is executed before \function{RGESolve} is executed again.


\subsubsection{\function{RGEGetSolution} does not return the leptogenesis
  parameters at the lightest right-handed neutrino mass.}

In order to get e.g.\ the CP asymmetry $\epsilon_1$ at the mass of the
lightest right-handed neutrino MR1, use 

\begin{verbatim}
RGEGetSolution[MR1,RGE\[Epsilon]1,1]
\end{verbatim}

The additional \verb|,1| tells \function{RGEGetSolution} to use the EFT
valid immediately above the energy MR1 for returning the value of
\param{RGE$\epsilon$1}.  This is necessary because the leptogenesis
parameters are not defined in the EFT without right-handed neutrinos
that is valid below MR1 and that would be used by
\function{RGEGetSolution} by default.


\subsubsection{What will happen, if \param{RGEye}, \param{RGEy$\mu$} and
  \param{RGEy$\tau$} are passed to \function{RGESetInitial} in addition to the
  matrix \param{RGEYe}?}
Generally, the matrices will be taken first and only if there is no matrix specified, the
  matrix is built from the specified eigenvalues and angles. In particular,
  \param{RGEYe} defines the Yukawa coupling matrix of the charged
  leptons if specified, and the eigenvalues \param{RGEye} etc.\ are not
  used then. The same applies for all other Yukawa coupling matrices.
  
\subsubsection{Changing the value of \param{RGE$\theta$12} doesn't have
  any effect.}
If the parameter \param{RGEM$\nu$r} is specified in
\function{RGESetInitial}, it determines the effective mass matrix
$m_\nu$ of the light neutrinos (together with \param{RGEY$\nu$}) via the
see-saw formula.  Therefore, all options affecting $m_\nu$ such as
\param{RGE$\theta$12}, \param{RGEMlightest} etc.\ do not have any effect
in this case.  If you would like to use these options, you have to
remove the replacement rule for \param{RGEM$\nu$r}.


%%%%%%%%%%%%%%%%%%%%%%%%%%%%%%%%%%%%%%%%%%%%%%%%%%%%%%%%%%%%%%%%%%%%%%%%%%%%%%%%%%%%%%%%%%%%%%%%%%%%


\bibliography{reap}

\end{document}

