\section{Conventions}

\subsection{Definition of the parameters}

\subsubsection{GUT Charge Normalization}
We use the GUT charge normalization in all models which is related to the charge normalization in the SM by
\begin{equation}
  q_Y^\text{GUT}=\sqrt{\frac{3}{5}}q^\text{SM}_Y\; .
\end{equation}
Therefore the coupling constant satisfies
\begin{equation}
  \left(g^\text{SM}_1\right)^2=\frac{3}{5}\left(g_1^\text{GUT}\right)^2\; .
\end{equation}

\subsubsection{Convention for Yukawa Matrices}
We use the RL convention for Yukawa matrices, i.e.\
\begin{equation}
  \mathscr{L}_\text{Yukawa}=-Y_{ij} \overline{\psi_\mathrm{R}^i}
  \psi_\mathrm{L}^j \cdot \phi + \text{h.c.} \;.
\end{equation}

\subsubsection{Vacuum Expectation Value of the SM Higgs}
The vacuum expectation value of the SM Higgs is $v=246\GeV$.
It is assumed to be constant.

\subsubsection{Standard Parameterization of the Lepton Mixing Matrix}

A unitary matrix can be described by 3 angles and 6 phases. Thus it can be
written in the following way:
\begin{equation}
 U 
 \,=\,
 \diag(e^{\I\delta_{e}},e^{\I\delta_{\mu}},e^{\I\delta_{\tau}}) \cdot V({ \theta_{12}},{ \theta_{13}},{\theta_{23}}) \cdot 
 \diag(e^{-\I\varphi_1/2},e^{-\I\varphi_2/2},1)
\end{equation}
V is a special unitary matrix and is parameterized in standard parameterization like the CKM
matrix in the quark sector with 3 angles ($\theta_{12},\,\theta_{13},\,\theta_{23}$) and 1 CP phase ($\delta$).
\begin{equation}
 V({ \theta_{12}},{ \theta_{13}},{\theta_{23}})=\left(
 \begin{array}{ccc}
 { c_{12}}{ c_{13}} & { s_{12}}{ c_{13}} & { s_{13}}
 \,e^{-\I{ \delta}}\\
 -{  c_{23}}{ s_{12}}-{  s_{23}}{ s_{13}}{ c_{12}}
 \,e^{\I{ \delta}} & {  c_{23}}{ c_{12}}-{  s_{23}}{ s_{13}}
 { s_{12}}\,e^{\I{ \delta}} & {  s_{23}}{ c_{13}}\\
 {  s_{23}}{ s_{12}}-{  c_{23}}{ s_{13}}{ c_{12}}
 \,e^{\I{ \delta}} & -{  s_{23}}{ c_{12}}-{  c_{23}}{ s_{13}}
 { s_{12}}\,e^{\I{ \delta}} & {  c_{23}}{ c_{13}}
 \end{array}
 \right)
\label{eq:DefV}
\end{equation}
where $s_{ij}$ and $c_{ij}$ are defined as$s_{ij}=\sin\theta_{ij}$ and
$c_{ij}=\cos\theta_{ij}$, respectively.
In addition there are phase matrices multiplied from both sides. The matrix on
the left-hand side is characterized by the unphysical phases
$\delta_e$, $\delta_\mu$ and $\delta_\tau$ which can be rotated away by a change of
the phases in the charged left-handed leptons in the extended (MS)SM.
\begin{equation}
\ket{\ell_L}\rightarrow
\diag(e^{-\I\delta_{e}},e^{-\I\delta_{\mu}},e^{-\I\delta_{\tau}}) \ket{\ell_L}
\end{equation}
The matrix on the right-hand side is described by the Majorana phases
$\varphi_1$ and $\varphi_2$. These can not be rotated away by a redefinition of
fields, because the effective neutrino mass term is a Majorana mass term which is
diagonalized by an unitary transformation and not by a biunitary transformation
like the Yukawa matrix of the charged leptons.

%%%%%%%%%%%%%%%%%%%%%%%%%%%%%%%%%%%%%%%%%%%%%%%%%%%%%%%%%%%%%%%%%%%%%%%%%%%%%%%%%%%
%% naming conventions
%%%%%%%%%%%%%%%%%%%%%%%%%%%%%%%%%%%%%%%%%%%%%%%%%%%%%%%%%%%%%%%%%%%%%%%%%%%%%%%%%%%

\subsection{Naming conventions}

\subsubsection{Variable names}
The first letter of each word and abbreviations like (``RGE'') in a variable
name are capitalized. The remaining letters are uncapitalized.  All local
variables have ``l'' as prefix and all parameters in functions have ``p'' as
prefix.

\subsubsection{Function names}
The first letter of each word and abbreviations like (``RGE'') in a function
name are capitalized. The remaining letters are uncapitalized.  All public
functions in \package{REAP.m} begin with ``RGE''.

\subsubsection{Exceptions}
The first letter of each word and abbreviations like (``RGE'') in the name of a
exception are capitalized. The remaining letters are uncapitalized.

The defined exceptions are:\\
\begin{itemize}
\item RGEModelAlreadyRegistered will be thrown if the model already is
registered.
\item RGELessThanZero will be thrown if a parameter is less than zero, thus out
of range.
\item RGEScaleTooBig will be thrown if the scale parameter is too big, thus out
of range.
\item RGENotImplementedYet will be thrown if the transition function isn't
implemented yet.
\item RGEOutOfRange will be thrown if the parameter is out of range.
\item RGEModelDoesNotExist will be thrown if the model name given does not exist
in the list Model. Thus the model hasn't been registered yet.
\item RGEWrongModel will be thrown if the model name does not corresponding to
the model valid at the given scale.
\item RGE$\nu$MassAboveCutoff will be thrown if an eigenvalue of M$\nu$ is above
the cutoff.
\item RGENotAValidMassHierarchy will be thrown if the type of the given mass
hierarchy does not exist.
\end{itemize}
