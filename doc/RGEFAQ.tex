\section{Frequently Asked Questions and their Answers (FAQ)}

\subsection{Physics Questions}

\subsubsection{How can I have more or less than 3 right-handed neutrinos?}

The default initial values have 3 right-handed neutrinos but you can define a
model with an arbitrary number of right-handed neutrinos by changing the initial
values for the right-handed neutrino mass matrix and the Yukawa couplings of the
neutrinos.

\subsubsection{How are neutrino mixing parameters defined above the
see-saw scale?}
In order to define mass and mixing parameters as functions of the
renormalization scale $\mu$ above the highest see-saw scale,
we consider the effective light neutrino mass matrix
\begin{equation} \label{eq:mnuFullTheory}
  m_\nu(\mu)\, =\, -\frac{v^2}{2} \, Y_\nu^T(\mu)\, M^{-1}(\mu)
\,Y_\nu(\mu) \;,
\end{equation}
where $Y_\nu$ and $M$ are $\mu$-dependent. (We do not take into account
the running of the Higgs vev.)
$m_\nu$ is the mass matrix of the three light neutrinos as obtained from
block-diagonalizing the complete $6\times6$ (for 3 singlet neutrinos) neutrino mass matrix,
following the standard see-saw calculation.
The energy-dependent mixing parameters are obtained from $m_\nu(\mu)$
and the running charged lepton Yukawa matrix $Y_e(\mu)$.
Between the see-saw scales or in a type II see-saw, there is an
additional contribution to $m_\nu$ from the dimension 5 neutrino mass
operator.

\subsubsection{How can I obtain the CP asymmetry for leptogenesis?}
The CP asymmetry in the case of thermal leptogenesis (in the limit $M_1 \ll
M_2,M_3$) is implemented as output
function in \package{REAP}. It can be obtained in the SM and MSSM by
\begin{verbatim}
RGEGetSolution[M_1, RGE\[Epsilon]1,1] .
\end{verbatim}
The CP asymmetry is not implemented for other leptogenesis scenarios. However,
the relevant quantities can be obtained via \function{RGEGetSolution}. See the
notebook \texttt{RGETemplate.nb} for an example.

\subsection{Implementation Details}

\subsubsection[``SM'' added, but \function{RGESetOptions} does not have any effect.]{I added the ``SM'' with
  \function{RGEAdd[``SM'',RGECutoff->1000]}, but
  \function{RGESetOptions[``SM'',RGE$\lambda$->0.3]} does not have any
  effect. Is this an error?}

This is no error, but sometimes the EFT's are changed in such a way that the whole model is
consistent. In this case the ``SM'' was changed to ``SM0N'', because all
right-handed neutrinos are integrated out above 1000 GeV.
You can use \function{RGESetOptions[``SM*'',RGE$\lambda$->0.3]} to change
\variable{RGE$\lambda$} in ``SM'' and ``SM0N'' at the same time.
Then you do not have to care whether all neutrinos have been integrated out.

\subsubsection{I want to change the Standard Model Higgs vev in all EFT's.}
You can use wildcards with \function{RGESetOptions},
\function{RGESetEFTOptions}, \function{RGESetModelOptions}, \\
\function{RGEGetOptions},
\function{RGEGetEFTOptions} and  \function{RGEGetModelOptions}, because the name
you enter is matched with \function{StringMatchQ}. See the documentation of
Mathematica for the possible wildcards.

\subsubsection{RGEAll does not work in ``*0N''}
\function{RGEGetSolution[Scale,RGEAll]} returns all parameters used in a see-saw
model. Thus, $Y_\nu$ and $M_{\nu_R}$ are returned in addition to the parameters
valid in a model without right-handed neutrinos. However, an error message will
be produced, unless there are right-handed neutrinos at a higher scale, because
\function{RGEGetSolution} obtains the values of $Y_\nu$ and $M_{\nu_R}$
recursively by determining the values at the cutoff. RGERaw which returns the
parameters valid in ``*0N'' can be used instead of RGEAll, if $Y_\nu$ and $M_{\nu_R}$ are not defined.

\subsubsection{Sometimes there are errors when RGESolve is executed twice.}

\function{RGESolve} adds new EFT's to the model. This is in conflict with the automatic detection of transitions.

The simplest workaround is either to set
\variable{RGERemoveAutoGeneratedEntries}, an option of \function{RGESolve}, to
`True' (This is the default value of this option.) or if this does not help, make sure that \function{RGEReset} is executed before \function{RGESolve} is executed again.


\subsubsection{\function{RGEGetSolution} does not return the leptogenesis
  parameters at the lightest right-handed neutrino mass.}

In order to get e.g.\ the CP asymmetry $\epsilon_1$ at the mass of the
lightest right-handed neutrino MR1, use 

\begin{verbatim}
RGEGetSolution[MR1,RGE\[Epsilon]1,1]
\end{verbatim}

The additional \verb|,1| tells \function{RGEGetSolution} to use the EFT
valid immediately above the energy MR1 for returning the value of
\param{RGE$\epsilon$1}.  This is necessary because the leptogenesis
parameters are not defined in the EFT without right-handed neutrinos
that is valid below MR1 and that would be used by
\function{RGEGetSolution} by default.


\subsubsection{What will happen, if \param{RGEye}, \param{RGEy$\mu$} and
  \param{RGEy$\tau$} are passed to \function{RGESetInitial} in addition to the
  matrix \param{RGEYe}?}
Generally, the matrices will be taken first and only if there is no matrix specified, the
  matrix is built from the specified eigenvalues and angles. In particular,
  \param{RGEYe} defines the Yukawa coupling matrix of the charged
  leptons if specified, and the eigenvalues \param{RGEye} etc.\ are not
  used then. The same applies for all other Yukawa coupling matrices.
  
\subsubsection{Changing the value of \param{RGE$\theta$12} doesn't have
  any effect.}
If the parameter \param{RGEM$\nu$r} is specified in
\function{RGESetInitial}, it determines the effective mass matrix
$m_\nu$ of the light neutrinos (together with \param{RGEY$\nu$}) via the
see-saw formula.  Therefore, all options affecting $m_\nu$ such as
\param{RGE$\theta$12}, \param{RGEMlightest} etc.\ do not have any effect
in this case.  If you would like to use these options, you have to
remove the replacement rule for \param{RGEM$\nu$r}.
