%\section{Solving the RG equation for the Neutrino mass matrix}
\section{First Steps\label{sec:FirstSteps}}

The following simple example demonstrates how to calculate the RG
evolution of the neutrino mass matrix in the MSSM extended by three
heavy singlet neutrinos.

\begin{enumerate}

\item The package corresponding to the model at the highest energy has
to be loaded.  All other packages needed in the course of the
calculation are loaded automatically.
\begin{verbatim}
  Needs["REAP`RGEMSSM`"]
\end{verbatim}
Note that ` is the backquote, which is used in opening quotation marks,
for example.

\item Next, we specify that we would like to use the MSSM with singlet
neutrinos:
\begin{verbatim}
  RGEAdd["MSSM"]
\end{verbatim}

%the model has to be defined. \function{RGEAddEFT[\param{model name},\param{cutoff},\optparam{options}]} adds a new part to the model. The parameters are the cutoff, the maximum valid energy, the name of the used model (e.g. ``SM'') and several options of the model (e.g. RGEIntegratedOut, which is a list of righthanded Neutrinos which are integrated out at that specific scale).
%The SM and the MSSM have a function implemented to search transitions, which are induced by integrating out heavy neutrinos.
%So you don't have to define the transitions by hand.
%This option is controlled by the option $\mathrm{SearchTransition}\rightarrow\mathrm{True/False}$.
%\begin{verbatim}
%  RGEAddEFT["SM",10^15];
%\end{verbatim}
%adds the SM as an effective model with cutoff $10^{15}$ GeV. The default options of the model are used.
%\begin{verbatim}
%  RGEAddEFT["SM",10^4,RGEIntegratedOut->1];
%\end{verbatim}
%adds the SM with cutoff $10^4$ GeV, but one right-handed neutrino is already integrated out.

\item Now we have to provide the initial values.  Here we use the
default values of the package (see Sec.~\ref{sec:Models} for details) and a
simple diagonal pattern for the neutrino Yukawa matrix.
%The function \function{RGESetInitial[\param{scale},\param{list of initial values}]} takes as parameters the scale where the initial values are given and a list of the initial values.
% The specific format of the list of initial values depends on the model used at that scale. (see \ref{models}).
%
%A model can give suggestions of initial values by the function \function{RGESuggestInitialValues[\param{model name},\optparam{name of suggestion}]}
\begin{verbatim}
  RGESetInitial[2*10^16,RGEY\[Nu]->{{1,0,0},{0,0.5,0},{0,0,0.1}}]
\end{verbatim}

\item \function{RGESolve[\param{low},\param{high}]} solves the RGEs
between the energy scales low and high.  The heavy singlets are
integrated out automatically at their mass thresholds.
%
%\function{RGESolve[\param{down},\param{up},\optparam{options}]}
%RGESolve accepts the same options as \function{NDSolve}.
\begin{verbatim}
  RGESolve[100,2*10^16]
\end{verbatim}

\item Using \function{RGEGetSolution[\param{scale},\optparam{quantity}]}
we can query the value of the quantity given in the second argument at
the energy given in the first one.  For example, this returns the mass
matrix of the light neutrinos at $100\GeV$:
 %returns the solution at a specific scale. The second parameter of RGEGetSolution specifies the output format of RGEGetSolution (e.g. RGEM$\nu$ returns the Neutrino mass, RGEMu returns the mass matrix of the up-type quarks,..., the standard behavior is to return the whole solution)
\begin{verbatim}
  MatrixForm[RGEGetSolution[100,RGEM\[Nu]]]
\end{verbatim}
%returns the light neutrino mass matrix $M\nu$ and 
%\begin{verbatim}
%  RGEGetSolution[100];
%\end{verbatim}
%returns all parameters.

\item To find the leptonic mass parameters, we use the function
\function{MNSParameters[\param{$m_\nu$},\param{$Y_e$}]} (which also
needs the Yukawa matrix of the charged leptons).  The results are given
in the order
$\{\{\theta_{12},\theta_{13},\theta_{23},\delta,\delta_e,\delta_\mu,
\delta_\tau,\varphi_1,\varphi_2\},
\{m_1,m_2,m_3\},\{y_e,y_\mu,y_\tau\}\}$.
\begin{verbatim}
  MNSParameters[RGEGetSolution[100,RGEM\[Nu]],RGEGetSolution[100,RGEYe]]
\end{verbatim}

\item Finally, we can plot the running of the mixing angles:
\begin{verbatim}
  Needs["Graphics`Graphics`"]
  mNu[x_]:=RGEGetSolution[x,RGEM\[Nu]]
  Ye[x_]:=RGEGetSolution[x,RGEYe]
  \[Theta]12[x_]:=MNSParameters[mNu[x],Ye[x]][[1,1]]
  \[Theta]13[x_]:=MNSParameters[mNu[x],Ye[x]][[1,2]]
  \[Theta]23[x_]:=MNSParameters[mNu[x],Ye[x]][[1,3]]
  LogLinearPlot[{\[Theta]12[x],\[Theta]13[x],\[Theta]23[x]},{x,100,2*10^16}]
\end{verbatim}
To produce nicer plots, the notebook RGEPlots.nb, which is included in
the package, can be used.

\end{enumerate}


In a second run, let us try some more modifications of the defaults.
For example, model parameters can be changed by including a command
after step (2):
\begin{verbatim}
  RGESetOptions["MSSM",RGEtan\[Beta]->20]
\end{verbatim}

Furthermore, we set the SUSY breaking scale to 200 GeV and use the SM
as an effective theory below this scale.
\begin{verbatim}
  RGEAdd["SM",RGECutoff->200]
\end{verbatim}

The initial values of the neutrino mass parameters can be changed by
adding replacement rules in step (3).  For instance, to set the
GUT-scale value of $\theta_{13}$ to $6^\circ$ and the Majorana phases to
$50^\circ$ and $120^\circ$:
\begin{verbatim}
  RGESetInitial[2*10^16,
    RGEY\[Nu]->{{1,0,0},{0,0.5,0},{0,0,0.1}},RGE\[Theta]13->6 Degree,
    RGE\[CurlyPhi]1->50 Degree,RGE\[CurlyPhi]2->120 Degree]
\end{verbatim}
The results of the RG evolution with these parameters are now obtained
by repeating the above steps (4)--(7).

%\begin{verbatim}
%{g1,g2,g3,Yu,Yd,Ye,Y\[Nu],\[Kappa],M,\[Lambda]}=
%       RGEGetParameters["SM"]/.RGESuggestInitialValues["SM","GUT"]
%       /.{RGEMassHierarchy -> "r", RGE\[Theta]12 -> 45, RGE\[Theta]13 -> 0,
%         RGE\[Theta]23 -> 45, RGEDeltaCP -> 0, RGE\[Phi]1 -> 0, RGE\[Phi]2 -> 0, 
%    RGEMlightest -> 0.05, RGE\[Delta]atm -> 2.5*10^-3, 
%    RGE\[Delta]sol -> 7.3*10^-5, RGE\[Tau]\[Nu] -> 0.25, RGE\[Nu]Ratio -> 2};
%RGESetInitial[RGEInitialScale/.RGESuggestInitialValues["SM","GUT"],
%    {g1,g2,g3,Yu,Yd,Ye,Y\[Nu],\[Kappa],M,\[Lambda]}];
%\end{verbatim}

\endinput
