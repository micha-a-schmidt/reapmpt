\NeedsTeXFormat{LaTeX2e}
\documentclass[10pt,a4paper,twoside]{scrartcl}

\usepackage{amsmath}
\usepackage{amssymb}
\usepackage{amsfonts}
\usepackage{amscd}
\usepackage[amsthm,thmmarks]{ntheorem}
%%\usepackage{TheoremCollection}
\usepackage{accents}
\usepackage{bbm} % BlackBoeard letters
%\usepackage[bbgreekl]{MyBbol} % Other Doublestroke Charakters
\usepackage{bbold}
\usepackage{fancyhdr} %
\usepackage{a4wide} %
\usepackage[small]{caption} %
\usepackage{makeidx} %
\usepackage{fleqn} 
\usepackage{indentfirst} 
\usepackage{cancel} % \cancel{stuff} provides slashed stuff
\usepackage{graphicx} % \including PostScript
\usepackage{color}
\usepackage{braket} %\bra{stuff} -> <stuff|
\usepackage{pstricks}
\usepackage{pst-node,pst-plot}
\usepackage{nomencl} %Nomenklatur
\usepackage{mathrsfs}   % Schoenes Lagrange -L mit \mathscr{L}
\usepackage{pifont} % dinbgbats
\usepackage[small]{subfigure}
\usepackage{longtable}
\usepackage[plain]{fancyref}
\usepackage{cite}
%\usepackage[bbgreekl]{MyBbol}  % bb-Symbole auch griechisch
\usepackage[sort&compress,numbers,colon]{natbib}
\bibliographystyle{apsrev}

%-- page parameters -------------------------------------------------

\pagestyle{fancyplain}

\advance \headheight by 3.0truept       % for 12pt mandatory...
\lhead[\fancyplain{}{\thepage}]{\fancyplain{}{\rightmark}}
\rhead[\fancyplain{}{\leftmark}]{\fancyplain{\thepage}{\thepage}}
\cfoot{}

%\addtolength{\oddsidemargin}{1.0truecm}
%\addtolength{\evensidemargin}{-0.3truecm}
\setlength{\parindent}{0cm}

%-- end of page parameters ------------------------------------------


%\makeindex
%\makeglossary

% \newcommand{\WeightConnect}[4]{\ncline{->}{#1}{#2}\mput*{\ovalnode{#3}{#4}}
% \ncline{-}{#1}{#3}\ncline{->}{#3}{#2}}
% 
% \def\Nf{i}
% \def\Ng{j}
% \newcommand{\GroupIndex}[1]{\ifcase#1\or \or a\or b\or c\or d\or e\or f\or g\or h\or i\or j\or
% k\or l\or m\or n\or o\or p\or q\or r\or s\or t\or u\or v\or w\or x\or
% y\or z\else\@ctrerr\fi}
% \newcommand{\FamilyIndex}[1]{\ifcase#1\or \or f\or g\or h\or i\or j\or
% k\or l\or m\or n\or o\or p\or q\or r\or s\or t\or u\or v\or w\or x\or
% y\or z\else\@ctrerr\fi}
\def\chargec{\mathrm{C}}        
\def\ChargeC{\mathrm{C}}        
\def\NuMSSM{{$\nu$MSSM}\ }
\newcommand{\SimpleRoot}[1]{\alpha^{(#1)}}
\newcommand{\FundamentalWeight}[1]{\mu^{(#1)}}
\newcommand{\ChargeConjugate}[1]{#1^\chargec}
\newcommand{\SFConjugate}[1]{#1^\chargec}
\newcommand{\CenterFmg}[1]{\ensuremath{\vcenter{\hbox{\input{#1.fmg}}}}}
\newcommand{\CenterObject}[1]{\ensuremath{\vcenter{\hbox{#1}}}}
\newcommand{\CenterEps}[2][1]{\ensuremath{\vcenter{\hbox{\includegraphics[scale=#1]{#2.eps}}}}} % Input eps files - Usage: \CenterEps[ScaleFactor]{FileName}
\newcommand{\Commutator}[2]{{\left[ #1,#2\right]}_-}
\newcommand{\AntiCommutator}[2]{{\left\{ #1,#2\right\}}}
\newcommand{\RightHandedNeutrino}{\nu}
\newcommand{\SuperCommutator}[2]{\left\Lbracket #1,#2\right\Rbracket}
\newcommand{\SuperField}[1]{\bbsymbol{#1}}
\newcommand{\package}[1]{\texttt{#1}}
\newcommand{\function}[1]{\texttt{#1}}
\newcommand{\param}[1]{\texttt{#1}}
\newcommand{\option}[1]{\texttt{#1}}
\newcommand{\variable}[1]{\texttt{#1}}
\newcommand{\optparam}[1]{\texttt{\textit{#1}}}
\newcommand{\warning}[1]{\hspace{2ex} \textit{warning: #1}}
\newcommand{\internal}[1]{#1 \warning{This function is for internal use only.}}
\newcommand{\eV}{\ensuremath{\,\mathrm{eV}}}
\newcommand{\keV}{\ensuremath{\,\mathrm{keV}}}
\newcommand{\MeV}{\ensuremath{\,\mathrm{MeV}}}
\newcommand{\GeV}{\ensuremath{\,\mathrm{GeV}}}
\newcommand{\TeV}{\ensuremath{\,\mathrm{TeV}}}

\newenvironment{implementation}{\em}{\normalfont}

\DeclareMathOperator{\re}{Re}
\DeclareMathOperator{\im}{Im}
\DeclareMathOperator{\tr}{tr}
\DeclareMathOperator{\Tr}{Tr}
\DeclareMathOperator{\diag}{diag}
\DeclareMathOperator{\quabla}{\boldsymbol{\square}}
\DeclareMathOperator{\STr}{STr}
\DeclareMathOperator{\Li}{Li}
\DeclareMathOperator{\ad}{ad}
\DeclareMathOperator{\Ad}{Ad}
\DeclareMathOperator{\AD}{AD}
\DeclareMathOperator{\ind}{ind}
\DeclareMathOperator{\ch}{ch}
\DeclareMathOperator{\Pf}{Pf}
\DeclareMathOperator{\arcosh}{arcosh}

\def\D{\mathrm{d}}
\def\I{\mathrm{i}}
\def\Nf{f}
\def\Ng{g}
\def\PlusIEpsilon{}
\newcommand{\ChargeConjMatrix}{\mathsf{C}}      % Charge conjugation matrix
\newcommand{\ChargeConjOp}{\boldsymbol{\ChargeConjMatrix}}% cc. operator
\def\NuMSSM{{$\nu$MSSM}\ }
\def\secname{section}
\newif\ifappendix
\newcommand{\secref}[1]{%
        \ifappendix\appendixname~\ref{#1}\else\secname~\ref{#1}\fi
        }
\appendixfalse
\renewcommand{\thesection}{\arabic{section}}
\renewcommand{\thesubsection}{\arabic{section}.\arabic{subsection}}
%\renewcommand{\thesubsubsection}{\arabic{section}.\arabic{subsection}.\arabic{subsubsection}}
% 
% \renewcommand{\thesection}{\arabic{section}}
% \renewcommand{\thesection}{\arabic{section}.\arabic{section}}
% \renewcommand{\thesubsection}{(\roman{subsection})}

\newcommand*{\fancyrefsubseclabelprefix}{subsec}
\newcommand*{\subsecname}{subsection}
\fancyrefaddcaptions{english}{%
\newcommand*{\frefsubsecname}{subsection}
\newcommand*{\Frefsubsecname}{subsection}
}
\numberwithin{equation}{section}
\numberwithin{table}{section}
\renewcommand{\thetable}{\arabic{section}-\arabic{table}}
\renewcommand{\theequation}{\arabic{section}.\arabic{equation}}
\renewcommand{\labelenumi}{(\arabic{enumi})}


\definecolor{MyBlue}{rgb}{0.8,0.85,1}
\def\labelitemi{$\bullet$}
\def\labelitemii{--}
\unitlength=1mm

\allowdisplaybreaks[1]


\begin{document}
\font\TitleFont=cmbx10 at 40pt \font\SubTitleFont=cmbx10 at 25pt
\pagenumbering{arabic} \title{\TitleFont{REAP 1.11.5
}\\[1cm]\SubTitleFont{Documentation} } \author{S.~Antusch, J.~Kersten,
M.~Lindner, M.~Ratz and M.A.~Schmidt} \maketitle
\begin{abstract}
  This is a more detailed documentation of the \package{REAP} add-on
  for Mathematica. We describe the functions which allow to calculate the
  evolution of the neutrino mass matrix in different models (SM, MSSM, 2HDM).
  Besides a function reference there is short HowTo on how you can build your
  own model.
\end{abstract}
\thispagestyle{empty}
\tableofcontents
\clearpage

%%%%%%%%%%%%%%%%%%%%%%%%%%%%%%%%%%%%%%%%%%%%%%%%%%%%%%%%%%%%%%%%%%%%%%%%%%%%%%%%%%%%%%%%%%%%%%%%%%%%
\vspace{3ex}

\begin{center}
  \noindent\fbox{
  \begin{minipage}{0.9\textwidth}
The package \package{REAP} is written for Mathematica 5 and is distributed under the terms of GNU Public License http://www.gnu.org/copyleft/gpl.html
\end{minipage}
}
\end{center}

\vspace{3ex}

\endinput


\section{Introduction}

The \package{REAP} (\textbf{R}enormalization group \textbf{E}volution of \textbf{A}ngles
and \textbf{P}hases) package is a Mathematica package to solve the
renormalization group equations (RGE) of the quantities relevant for neutrino
masses, for example the dimension 5 neutrino mass operator, the Yukawa matrices
and the gauge couplings.  So far, the $\beta$-functions for the standard model
(SM), the minimal supersymmetric standard model (MSSM) and the two higgs doublet
model with $\mathbb{Z}_2$ symmetry (2HDM) with and without right-handed
neutrinos are implemented.  Heavy degrees of freedom such as singlet neutrinos
can be integrated out automatically at the correct mass thresholds which are
determined by a fixed-point iteration. Thus the evolution is described by
several effective theories.
In addition all models are implemented with Dirac neutrinos.  By means of the
\package{MixingParameterTools} package, the calculated running of the neutrino
mass matrix can be translated into the running of the mixing parameters and the
mass eigenvalues.


If you would like to refer to REAP in a publication or talk, please cite
the accompanying paper hep-ph/0501272.


\endinput


%\begin{center}
% \fbox{\textbf{Note}: The documentation for the mixing parameter tools is in a
% separate directory!}
%\end{center}

\section{Installation}

\subsection{Automatic Installation under UNIX/Linux}

Execute REAP.installer and you are done.
\begin{verbatim}
sh REAP.installer
\end{verbatim}
After the execution the Mathematica packages are copied to
\verb+~/.Mathematica/Applications/REAP+ and the documentation and notebooks are in a
subdirectory of the working directory, which is called REAP.

In addition, you have to install the package MixingParameterTools. There is also
a script which installs both REAP and MPT at the same time:
REAP\_MPT.installer.


\subsection{Semi-Automatic Installation under UNIX/Linux}

Unpack the archive REAP.tar.gz.
\begin{verbatim}
tar -xvzf REAP.tar.gz
\end{verbatim}
Then go to the directory REAPInstall and execute the script install.sh.
\begin{verbatim}
cd REAPInstall
sh ./install.sh
\end{verbatim}
The script copies the Mathematica packages to
\verb+~/.Mathematica/Applications/REAP+.  
The documentation and some sample notebooks are placed in a new
subdirectory of the working directory called REAP.  Hence, the folder
REAPInstall can be deleted now.

In addition, you have to install the package MixingParameterTools.
REAP and MPT can be installed simultaneously by using the archive
REAP\_MPT.tar.gz.  The procedure is completely analogous to the one
described above.


\subsection{Installation by Hand}

In order to install the package(s) manually, unpack the archive
REAP.tar.gz first.  Under UNIX/Linux, type
\begin{verbatim}
tar -xvzf REAP.tar.gz
\end{verbatim}
On Windows systems, a program like WinZip can be used.
Then move the directory REAP from the folder
REAPInstall to the directory
where the Mathematica add-ons are located, e.g.\
\begin{verbatim}
mv REAPInstall/REAP ~/.Mathematica/Applications/
\end{verbatim}
under UNIX/Linux. 
Under Windows XP, the path to the add-on directory should be something
like \verb+Application Data\Mathematica\Applications+.
The documentation and some sample notebooks can be found in
REAPInstall/Doc/REAP/.

In addition, you have to install the package MixingParameterTools.
To install both REAP and MPT at the same time, you can use the archive
REAP\_MPT.tar.gz.  The procedure is the same as above (except that the 
installation directory is called REAP\_MPTInstall now), supplemented by
an analogous step for moving the MPT directory, e.g.
\begin{verbatim}
mv REAP_MPTInstall/MixingParameterTools ~/.Mathematica/Applications/
\end{verbatim}


\endinput

%\section{Solving the RG equation for the Neutrino mass matrix}
\section{First Steps\label{sec:FirstSteps}}

The following simple example demonstrates how to calculate the RG
evolution of the neutrino mass matrix in the MSSM extended by three
heavy singlet neutrinos.

\begin{enumerate}

\item The package corresponding to the model at the highest energy has
to be loaded.  All other packages needed in the course of the
calculation are loaded automatically.
\begin{verbatim}
  Needs["REAP`RGEMSSM`"]
\end{verbatim}
Note that ` is the backquote, which is used in opening quotation marks,
for example.

\item Next, we specify that we would like to use the MSSM with singlet
neutrinos:
\begin{verbatim}
  RGEAdd["MSSM"]
\end{verbatim}

%the model has to be defined. \function{RGEAddEFT[\param{model name},\param{cutoff},\optparam{options}]} adds a new part to the model. The parameters are the cutoff, the maximum valid energy, the name of the used model (e.g. ``SM'') and several options of the model (e.g. RGEIntegratedOut, which is a list of righthanded Neutrinos which are integrated out at that specific scale).
%The SM and the MSSM have a function implemented to search transitions, which are induced by integrating out heavy neutrinos.
%So you don't have to define the transitions by hand.
%This option is controlled by the option $\mathrm{SearchTransition}\rightarrow\mathrm{True/False}$.
%\begin{verbatim}
%  RGEAddEFT["SM",10^15];
%\end{verbatim}
%adds the SM as an effective model with cutoff $10^{15}$ GeV. The default options of the model are used.
%\begin{verbatim}
%  RGEAddEFT["SM",10^4,RGEIntegratedOut->1];
%\end{verbatim}
%adds the SM with cutoff $10^4$ GeV, but one right-handed neutrino is already integrated out.

\item Now we have to provide the initial values.  Here we use the
default values of the package (see Sec.~\ref{sec:Models} for details) and a
simple diagonal pattern for the neutrino Yukawa matrix.
%The function \function{RGESetInitial[\param{scale},\param{list of initial values}]} takes as parameters the scale where the initial values are given and a list of the initial values.
% The specific format of the list of initial values depends on the model used at that scale. (see \ref{models}).
%
%A model can give suggestions of initial values by the function \function{RGESuggestInitialValues[\param{model name},\optparam{name of suggestion}]}
\begin{verbatim}
  RGESetInitial[2*10^16,RGEY\[Nu]->{{1,0,0},{0,0.5,0},{0,0,0.1}}]
\end{verbatim}

\item \function{RGESolve[\param{low},\param{high}]} solves the RGEs
between the energy scales low and high.  The heavy singlets are
integrated out automatically at their mass thresholds.
%
%\function{RGESolve[\param{down},\param{up},\optparam{options}]}
%RGESolve accepts the same options as \function{NDSolve}.
\begin{verbatim}
  RGESolve[100,2*10^16]
\end{verbatim}

\item Using \function{RGEGetSolution[\param{scale},\optparam{quantity}]}
we can query the value of the quantity given in the second argument at
the energy given in the first one.  For example, this returns the mass
matrix of the light neutrinos at $100\GeV$:
 %returns the solution at a specific scale. The second parameter of RGEGetSolution specifies the output format of RGEGetSolution (e.g. RGEM$\nu$ returns the Neutrino mass, RGEMu returns the mass matrix of the up-type quarks,..., the standard behavior is to return the whole solution)
\begin{verbatim}
  MatrixForm[RGEGetSolution[100,RGEM\[Nu]]]
\end{verbatim}
%returns the light neutrino mass matrix $M\nu$ and 
%\begin{verbatim}
%  RGEGetSolution[100];
%\end{verbatim}
%returns all parameters.

\item To find the leptonic mass parameters, we use the function
\function{MNSParameters[\param{$m_\nu$},\param{$Y_e$}]} (which also
needs the Yukawa matrix of the charged leptons).  The results are given
in the order
$\{\{\theta_{12},\theta_{13},\theta_{23},\delta,\delta_e,\delta_\mu,
\delta_\tau,\varphi_1,\varphi_2\},
\{m_1,m_2,m_3\},\{y_e,y_\mu,y_\tau\}\}$.
\begin{verbatim}
  MNSParameters[RGEGetSolution[100,RGEM\[Nu]],RGEGetSolution[100,RGEYe]]
\end{verbatim}

\item Finally, we can plot the running of the mixing angles:
\begin{verbatim}
  Needs["Graphics`Graphics`"]
  mNu[x_]:=RGEGetSolution[x,RGEM\[Nu]]
  Ye[x_]:=RGEGetSolution[x,RGEYe]
  \[Theta]12[x_]:=MNSParameters[mNu[x],Ye[x]][[1,1]]
  \[Theta]13[x_]:=MNSParameters[mNu[x],Ye[x]][[1,2]]
  \[Theta]23[x_]:=MNSParameters[mNu[x],Ye[x]][[1,3]]
  LogLinearPlot[{\[Theta]12[x],\[Theta]13[x],\[Theta]23[x]},{x,100,2*10^16}]
\end{verbatim}
To produce nicer plots, the notebook RGEPlots.nb, which is included in
the package, can be used.

\end{enumerate}


In a second run, let us try some more modifications of the defaults.
For example, model parameters can be changed by including a command
after step (2):
\begin{verbatim}
  RGESetOptions["MSSM",RGEtan\[Beta]->20]
\end{verbatim}

Furthermore, we set the SUSY breaking scale to 200 GeV and use the SM
as an effective theory below this scale.
\begin{verbatim}
  RGEAdd["SM",RGECutoff->200]
\end{verbatim}

The initial values of the neutrino mass parameters can be changed by
adding replacement rules in step (3).  For instance, to set the
GUT-scale value of $\theta_{13}$ to $6^\circ$ and the Majorana phases to
$50^\circ$ and $120^\circ$:
\begin{verbatim}
  RGESetInitial[2*10^16,
    RGEY\[Nu]->{{1,0,0},{0,0.5,0},{0,0,0.1}},RGE\[Theta]13->6 Degree,
    RGE\[CurlyPhi]1->50 Degree,RGE\[CurlyPhi]2->120 Degree]
\end{verbatim}
The results of the RG evolution with these parameters are now obtained
by repeating the above steps (4)--(7).

%\begin{verbatim}
%{g1,g2,g3,Yu,Yd,Ye,Y\[Nu],\[Kappa],M,\[Lambda]}=
%       RGEGetParameters["SM"]/.RGESuggestInitialValues["SM","GUT"]
%       /.{RGEMassHierarchy -> "r", RGE\[Theta]12 -> 45, RGE\[Theta]13 -> 0,
%         RGE\[Theta]23 -> 45, RGEDeltaCP -> 0, RGE\[Phi]1 -> 0, RGE\[Phi]2 -> 0, 
%    RGEMlightest -> 0.05, RGE\[Delta]atm -> 2.5*10^-3, 
%    RGE\[Delta]sol -> 7.3*10^-5, RGE\[Tau]\[Nu] -> 0.25, RGE\[Nu]Ratio -> 2};
%RGESetInitial[RGEInitialScale/.RGESuggestInitialValues["SM","GUT"],
%    {g1,g2,g3,Yu,Yd,Ye,Y\[Nu],\[Kappa],M,\[Lambda]}];
%\end{verbatim}

\endinput



\section{Reference}

\subsection{Implementation details}

\package{REAP} is divided in three parts. The main part
is \package{RGESolver} which
provides a standard interface between the different models and the user. Thus
the user does not have to know anything about the implementation details of the
different models besides the parameters of the models. 
The second part are the different models, like \package{RGESM},
\package{RGEMSSM}, \dots which contain the model
specific parts of the package. 
The third part is formed by some utility packages
(\package{RGEUtilities}, \package{RGEParameters}, 
\package{RGEInitial}, \package{RGEFusaokaYukawa}, \package{RGESymbol}, \package{RGETakagi}) which provide several useful functions
to the different models. In principle, a user only needs a limited set of
functions of \package{RGESolver}.



%% The package \package{REAP} is divided in three parts. The main part is the Mathematica File \package{REAP`RGESolver`} which provides a standard interface to all model. The second part are the different models, like \package{REAP`RGESM`}, \dots ,which contain the model specific parts of the package. The third part is formed by some utility packages (\package{REAP`RGEUtilities`}, \package{REAP`RGEParameters`}, \package{REAP`RGESymbol`}, \package{REAP`RGEInitial`}, \package{REAP`RGEFusaoka`}, \package{REAP`RGETakagi`}) which provide several useful functions to the different models.

%% The \package{RGESolver} provides the interface between the user and the different models. Every model has to register with \function{RegisterModel}.
%% First the model name is saved in the list named Model. The other parameters are stored in similar lists.
%% The user has to tell to \package{RGESolver} how his model does look like. He adds several EFT's to his model by \function{RGEAddEFT}. These EFT's are stored in the list \variable{RGEModel}.

\subsection[\package{RGESolver}]{\package{REAP`RGESolver`}}

The package distinguishes between two different kind of functions. On the one
hand, there are functions which directly work with the supplied models. They are
named \function{RGE*Model*}. On the other hand, there are functions dealing
with the models which are used as an effective field theory (EFT), i.e.\ have been added by
\function{RGEAddEFT}. These functions are named \function{RGE*EFT*}.

At the beginning, all models have to be loaded by
\function{RGERegisterModel} in order to make them accessible through
\package{RGESolver}. \function{RGERegisterModel} takes as argument
different functions to communicate with the model. After all models have been
registered which is done by the packages, the models are contained in, the user
has to specify, how his sequence of EFTs is made up. Different models can be
added as EFT by \function{RGEAddEFT}. The cutoff is specified by the option
\option{RGECutoff}. Next, the initial values have to be supplied by the function 
\function{RGESetInitial}. Then the renormalization group equations are solved
by executing \function{RGESolve} which uses \function{NDSolve} to numerically
integrate the differential equations. Finally, the parameters can be obtained
through \function{RGEGetSolution} at any scale. In order to illustrate the use
of \package{REAP}, an example is given in Sec.~\ref{sec:FirstSteps} and the
algorithm to solve the different ranges is demonstrated in the following example.

The setup is the MSSM extended by 3 right-handed neutrinos at the GUT scale of
$2\cdot10^{16} \GeV$ and set the SUSY breaking scale to 1 TeV. The initial values
are set to the suggested values which are specified in Sec.~\ref{sec:Models}.
At first, we define the model and set the initial values.
\begin{verbatim}
  RGEAddModel["MSSM"];
  RGEAddModel["SM",RGECutoff->1000];
  RGESetInitial[2 10^16];
\end{verbatim}
The execution of \function{RGESolve[$91.19$,$2\cdot 10^{16}$]} solves the RGE and finds the scales where
the right-handed neutrinos are integrated out.

\begin{enumerate}
\item Solve the RGEs for the MSSM with $3$ right-handed neutrinos between the
  GUT scale and the SUSY breaking scale without considering any thresholds.
\item Find the heaviest right-handed neutrino with mass $M_3$ and add a new EFT
  by \\ \function{RGEAddEFT["MSSM",RGECutoff->$M_3$, RGEIntegratedOut->1]}.
\item Calculate initial values for MSSM with $2$ right-handed neutrinos by matching $\kappa,
  Y_\nu$, $M$ and the other parameters at the scale where the first right-handed neutrino is integrated out.
\item Solve the RGEs for the MSSM with $2$ right-handed neutrinos between $M_3$ and
  the SUSY breaking scale.
\item Find the second to heaviest right-handed neutrino with mass $M_2$ and add
  a new EFT by \function{RGEAddEFT["MSSM", RGECutoff->$M_2$,
  RGEIntegratedOut->2]}.
%\item etc.
\item Calculate initial values for MSSM with $1$ right-handed neutrino.
\item Solve the RGEs for the MSSM with $1$ right-handed neutrinos between $M_2$ and
  the SUSY breaking scale.
\item Find the lightest right-handed neutrino with mass $M_1$ and add
  a new EFT by\\ \function{RGEAddEFT["MSSM0N", RGECutoff->$M_1$]}.
\item Calculate initial values for MSSM without right-handed neutrinos.
\item Solve the RGEs for the MSSM without right-handed neutrinos between $M_1$
  and the SUSY breaking scale.
\item Calculate initial values for the SM
\item Since all right-handed neutrinos have been integrated out already, change
  SM to SM0N.
\item Solve the RGEs for SM0N between the SUSY breaking scale and the mass of $Z^0$.
\end{enumerate}

\subsubsection{RGEAdd}

\function{RGEAdd[\param{model},\optparam{options}]} 
specifies that \param{model} should be used as an effective theory (EFT)
up to a cutoff energy given in the \optparam{options}.  If no cutoff is
given, $\infty$ is used.  \optparam{options} can also be used to specify
various parameters such as $\tan\beta$.  See Sec.~\ref{sec:Models} for a
complete list of the models and options available.

\begin{verbatim}
  RGEAdd["MSSM",RGEtan\[Beta]->50]
  RGEAdd["SM",RGECutoff->10^3]
\end{verbatim}
In this case, the MSSM with $\tan\beta=50$ is used at high energies.
Below $10^3\GeV$ (the SUSY breaking scale in this example), the SM is
used as an EFT.



\subsubsection{RGEAddEFT}

This command is identical to \function{RGEAdd}.

\subsubsection{RGEGetEFTOptions}

Same as \function{RGEGetOptions}.

\subsubsection{RGEGetInitial}

\function{RGEGetInitial[]} returns the scale at which the initial values
are given and the initial values.
%
% \begin{verbatim}
%  RGEGetInitial[]
%\end{verbatim}
%This returns \begin{verbatim}{2 10^16, {RGEg1->0.2, RGEg2->0.5, RGEg3->0.3}}\end{verbatim}
%

\subsubsection{RGEGetModelOptions}

\function{RGEGetModelOptions[\param{model name}]} returns the options of the model
\param{model name}.
 
  \begin{verbatim}
   RGEGetModelOptions["SM"]
 \end{verbatim}
% This returns the options which are currently set for the ``SM''.
 

\subsubsection{RGEGetOptions}

\function{RGEGetOptions[\param{model}]} returns the options set by
\function{RGEAdd} or \function{RGESetOptions} for the EFT \param{model}.
Wildcards can be used in \param{model}.
 
  \begin{verbatim}
   RGEGetOptions["SM*"]
 \end{verbatim}
 This returns the options which are currently set for all EFTs whose
 names start with ``SM''.
 

\subsubsection{RGEGetParameters}

\function{RGEGetParameters[\param{model}]} returns the quantities that run
in the \param{model}.
% 
%  \begin{verbatim}
%   RGEGetParameters["SM"];
% \end{verbatim}
% This returns the parameters of the ``SM''.
% 

\subsubsection{RGEGetSolution}

\label{sec:RGEGetSolution}
\function{RGEGetSolution[\param{scale},\optparam{parameter}]} 
returns the solution of the RGEs at the energy \param{scale}.
The \optparam{parameter} (optional) specifies the quantity
of interest (cf.\ Sec.~\ref{sec:Models} for the lists for each model).
If no \optparam{parameter} is given, the values of all running
quantities are returned.


\begin{verbatim}
  RGEGetSolution[100,RGEM\[Nu]]
\end{verbatim}
returns the neutrino mass matrix at $100\GeV$.
\begin{verbatim}
  RGEGetSolution[100]
\end{verbatim}
returns all parameters at $100\GeV$.



\subsubsection{RGEGetTransitions}

\function{RGEGetTransitions[]} returns the transitions (thresholds)
between the various EFTs in a list
containing the energy scale, the model name and its options.
%
%\begin{verbatim}
%  Print[RGEGetTransitions[]];
%  {{10000000,MSSM,{}},{10000,MSSM,{RGEIntegratedOut->1}}}
%\end{verbatim}
%


\subsubsection{RGELoadAll}

\function{RGELoadAll[\param{filename}]} loads the saved state which is given in
\param{filename}.

\subsubsection{RGELoadResults}

\function{RGELoadResults[\param{model}]} loads the saved model which is given in
\param{model}.

\subsubsection{RGERegisterModel}

\function{RGERegisterModel\label{func:RGERegisterModel}[\param{name}, \param{get
parameters}, \param{solve
RGE}, \param{return solution}, \param{transition functions}, \param{provide
initial values}, \param{set options}, \param{get options}]} of the package
\package{REAP} registers a new model.  Its 8 parameters are:
\begin{enumerate}
  \item a string containing the name of the model
  \item a string containing the name of the package
  \item a function returning a list of the parameters of the model
  \item a function to solve the RGE of this model in a given range
  \item a list of replacement rules with the functions to return the result,
  like \textit{Symbol$\rightarrow$function returning the solution}
  \item a list containing the transition functions in the form \textit{\{``name of
  the target model'',name of the function\}}
  \item a function to provide initial values
  \item a function to set options of the model
  \item a function to get the initial values
\end{enumerate}


\begin{verbatim}
RGERegisterModel["SM","SolveNeutrinoRGES`RGESM`" `Private`GetParameters,
        `Private`SolveModel, {RGEAll->`Private`GetSolution,
        RGEM\[Nu]->`Private`GetM\[Nu], RGEMe->`Private`GetMe,
        RGEMu->`Private`GetMu, RGEMd->`Private`GetMd},
        {{"SM",`Private`TransSM},{"SM0N",`Private`TransSM0N}},
        `Private`GetInitial, `Private`ModelSetOptions, `Private`ModelGetOptions
        ];
\end{verbatim}

This example registers the SM.  There are 5 functions to return solutions:
Private`GetSolution, Private`GetM$\nu$,\dots. Moreover the only 2 transition
functions are the transition functions to the SM (with righthanded neutrinos)
itself: \{"SM",Private`TransSM\} and to the SM without righthanded Neutrinos
\{"SM0N",Private`TransSM0N\}.



\subsubsection{RGEReset}

\function{RGEReset[]} removes all EFTs and resets all options which have
been changed by \function{RGEAdd} or \function{RGESetOptions}
%functions named RGE*EFT* 
to their default values. 
Options which have been changed by
%functions named RGE*Model* (e.g. RGESetModelOptions)
\function{RGESetModelOptions} are not reset.

\subsubsection{RGESaveAll}

\function{RGESaveAll[\param{filename}]} saves the state to \param{filename}.

\subsubsection{RGESaveInitialData}

\function{RGESaveInitialData[]} returns all data which is relevant to rerun the
calculation,i.e. Initial values and the range. The data is returned as a list of replacement rules which
is self-explaining (RGEUpperBound, RGELowerBound determine the range which is
passed to \function{RGESolve}, RGEBoundaryScale is the scale where the
initial data is given and RGEModelData is the model with cutoff and options. The remaining parameters are the initial values.).

\subsubsection{RGESaveResults}

\function{RGESaveModel[]} returns the current model. It can be loaded again by \function{RGELoadResults}.

\subsubsection{RGESetEFTOptions}

Same as \function{RGESetOptions}.

\subsubsection{RGESetInitial}

\function{RGESetInitial[\param{scale},\param{initial conditions}]} sets
the initial values at the energy \param{scale}.  They are entered as
replacement rules and can also contain options (e.g.\ to select the
neutrino mass hierarchy).  See Sec.~\ref{sec:Models} for the names of
the variables and options in the different models.
The option \variable{RGESuggestion} chooses between
several sets of default values. If it is not given, the first set of default
values is taken.  In general, these are the default values at the GUT scale.

\begin{verbatim}
  RGESetInitial[10^16,RGE\[Theta]13->4 Degree,RGEMlightest->0.1]
\end{verbatim}
This sets the initial values at $10^{16}\GeV$.  The mixing angle
$\theta_{13}$ is set to $4^\circ$, and the mass of the lightest neutrino
to $0.1\eV$.  For the other parameters, the default values are used.



\subsubsection{RGESetModelOptions}

\function{RGESetModelOptions[\param{model name},\param{options}]} globally changes the options
of \param{model name} to \param{options}. Metacharacters, like * and @, are
allowed in the \param{model name}. \param{model name} is matched against all
model names with \function{StringMatchQ}.

\begin{verbatim}
  RGESetModelOptions["SM",RGEvEW->246];
\end{verbatim}
This sets the option RGEvEW of the ``SM'' to 246.  The other options are
unchanged.


\subsubsection{RGESetOptions}

\function{RGESetOptions[\param{model},\param{options}]} changes the options
of the EFTs defined by \function{RGEAdd} with name matching \param{model} to \param{options}. Metacharacters, like * and @, are
allowed in the name.

\begin{verbatim}
  RGESetOptions["MSSM",RGEtan\[Beta]->40]
\end{verbatim}
This sets $\tan\beta$ of the ``MSSM'' to 40.  The EFT must have been
added earlier by \function{RGEAdd["MSSM"]}.  The other options are
unchanged.



\subsubsection{RGESolve}

\function{RGESolve[\param{low},\param{high},\optparam{options}]} solves
the RGEs between the energies \param{low} and \param{high}. 
It accepts the same options as \function{NDSolve}.  In addition, the option
\variable{RGERemoveAutoGeneratedEntries} determines whether automatically
generated EFTs (such as the MSSM with 2 singlet neutrinos, if one
started with 3 singlets) are removed before solving the RGEs. The default
value is ``True''.
% which means that automatically generated entries are removed. 
If it is set to ``False'', no EFT will be removed.

\begin{verbatim}
  RGESolve[100,10^15]
\end{verbatim}
This solves the RGEs between $100\GeV$ and $10^{15}\GeV$.
%using the method \variable{StiffnessSwitching} for the calculation.





\subsection[\package{RGESymbol}]{\package{REAP`RGESymbol`}}
\package{RGESymbol} defines several symbols which are used in exception handling
and as parameters in RGEGetSolution. 

%% These symbols are used in \package{RGESolver} to control automatically generated entries:\\
%%   \begin{itemize}
%%     \item RGEAutoGenerated is a tag to mark the EFT's which have been generated
%%   automatically or changed. It can have three different values:
%%   \begin{itemize}
%%   \item ``True'' means that the EFT has been generated automatically.
%%   \item ``False'' means that the EFT was not generated automatically.
%%   \item The name of some model means that the current EFT has replaced the given
%%   EFT. e.g. The current EFT is the SM w/o right-handed neutrinos (SM0N) and
%%   RGEAutoGenerated is set to ``SM''.
%%   \end{itemize}
%%     \item RGERemoveAutoGeneratedEntries determines whether automatically generated
%%   entries are removed.
%%   \end{itemize}
%% \item symbols used to return initial values
%%   \begin{itemize}
%%   \item RGEg1 is the coupling constant of $U(1)_Y$.
%%   \item RGEg2 is the coupling constant of $SU(2)_L$.
%%   \item RGEg3 is the coupling constant of $SU(3)_C$.
%%   \item RGEYu is the Yukawa matrix of the up-type quarks.
%%   \item RGEYd is the Yukawa matrix of the down-type quarks.
%%   \item RGEYe is the Yukawa matrix of the charged leptons.
%%   \item RGEY$\nu$ is the Yukawa matrix of the neutrinos.
%%   \item RGEM$\nu$r is the mass matrix of the right-handed neutrinos.
%%   \item RGE$\kappa$1 is the parameter of the dimension 5 operator associated
%%   with the first Higgs in the 2HDM.
%%   \item RGE$\kappa$2 is the parameter of the dimension 5 operator associated
%%   with the second Higgs in the 2HDM.
%%   \item RGE$\kappa$ is the parameter of the dimension 5 operator.
%%   \item RGE$\lambda$ is the quartic Higgs self coupling.
%%   \item RGE$\lambda$1 is $\lambda$1 in the Higgs potential of the 2HDM.
%%   \item RGE$\lambda$2 is $\lambda$2 in the Higgs potential of the 2HDM.
%%   \item RGE$\lambda$3 is $\lambda$3 in the Higgs potential of the 2HDM.
%%   \item RGE$\lambda$4 is $\lambda$4 in the Higgs potential of the 2HDM.
%%   \item RGE$\lambda$5 is $\lambda$5 in the Higgs potential of the 2HDM.
%%   \end{itemize}

%% \item symbols used in RGEModelOptions
%%   \begin{itemize}
%%   \item RGEu is a vector defining which Higgs the up-type quarks are coupling
%%   to.
%%   \item RGEd is a vector defining which Higgs the down-type quarks are coupling
%%   to.
%%   \item RGE$\nu$ is a vector defining which Higgs the neutrinos are coupling to.
%%   \end{itemize}

  

%% \item options used by RGEGetSolution
%% \begin{itemize}
%% \item RGEAll is used to get the all parameters
%% \item RGECoupling is used to get the coupling constants
%% \item RGE$\alpha$ is used to get the fine structure constants
%% \item RGE$\lambda$ is used by RGEGetSolution to get the Higgs couplings,
%% e.g. the quartic Higgs self coupling in the case of the SM

%% \item RGEM$\nu$ is used to get the neutrino masses
%% \item RGEMe is used to get the charged lepton masses
%% \item RGEMu is used to get the up-type quark masses
%% \item RGEMd is used to get the down-type quark masses

%% \item RGEY$\nu$ is used by RGEGetSolution to get the Yukawa coupling matrix of
%% the neutrinos
%% \item RGEYe is used by RGEGetSolution to get the Yukawa coupling matrix of the
%% charged leptons
%% \item RGEYu is used by RGEGetSolution to get the Yukawa coupling matrix of the
%% up-type quarks
%% \item RGEYd is used by RGEGetSolution to get the Yukawa coupling matrix of the
%% down-type quarks


%% \item RGE$\kappa$ is used by RGEGetSolution to get $\kappa$.
%% \item RGE$\kappa1$ is used by RGEGetSolution to get $\kappa1$.
%% \item RGE$\kappa2$ is used by RGEGetSolution to get $\kappa2$.
%% \item RGEM$\nu$r is used by RGEGetSolution to get the mass matrix of the
%% right-handed neutrinos.

%% \item RGERaw is used by RGEGetSolution to get the raw values of the
%% parameters. A raw parameter is the internal representation of the parameter,
%% e.g. RGERawM$\nu$r returns a 2x2 matrix after the first right-handed neutrino is
%% integrated out, in contrast to RGEM$\nu$r which returns a 3x3 matrix.
%% \item RGERawM$\nu$r is used by RGEGetSolution to get the raw mass matrix of the
%% right-handed neutrinos.
%% \item RGERawY$\nu$ is used by RGEGetSolution to get the raw Yukawa coupling
%% matrix of the neutrinos.
%% \end{itemize}


%% \item symbols used by \function{RGESetInitial}
%% \begin{itemize}
%% \item RGESuggestion specifies which default values are taken.
%% \item names of options accepted by RGESetInitial (see \ref{models})
  
%% \end{itemize}

%% \item names of options taken by the models (see \ref{models})



\subsection[\package{RGEInitial}]{\package{REAP`RGEInitial`}}

This package contains some functions for converting mass and mixing parameters
into mass matrices. They are mainly intended for internal use by REAP, but may
be helpful for the user in some occasions.


\subsubsection{RGEGetDiracY$\nu$}

\function{RGEGetDiracY$\nu$[\param{$\theta_{12}$}, \param{$\theta_{13}$},
    \param{$\theta_{23}$}, \param{$\delta$}, \param{$\delta_e$}, \param{$\delta_\mu$}, \param{$\delta_\tau$},
    \param{$\varphi_1$}, \param{$\varphi_2$}, \param{Mlightest}, \param{$\Delta m^2_\mathrm{atm}$}, \param{$\Delta m^2_\mathrm{sol}$}, \param{mass hierarchy}, \param{vu}]} returns a suggestion for Y$\nu$
    in the case of Dirac neutrinos.
\begin{itemize}
\item The first 9 parameters specify the mixing matrix.
\item The 10th parameter is the mass of the lightest neutrino.
\item The 11th and 12th parameter are the mass squared differences of the
atmospheric and solar neutrino oscillations respectively.
\item The 13th parameter is the mass hierarchy. "i" means inverted and "r" or
"n" means normal.
\item The 14th parameter is the vev of the Higgs coupling
to the neutrinos.
\end{itemize}


\subsubsection{RGEGetM}

\function{RGEGetM[\param{$\theta_{12}$},\param{$\theta_{13}$},
\param{$\theta_{23}$},\param{$\delta$},\param{$\delta_e$},\param{$\delta_\mu$},\param{$\delta_\tau$},\param{$\varphi_1$},\param{$\varphi_2$},\param{Mlightest},\param{$\Delta
m^2_\mathrm{atm}$},\param{$\Delta m^2_\mathrm{sol}$},\param{mass hierarchy}, \param{vu},\param{$Y_\nu$}]} returns a
    suggestion for the mass matrix of the right-handed neutrinos.
\begin{itemize}
\item The first 9 parameters specify the mixing matrix.
\item The 10th parameter is the mass of the lightest neutrino.
\item The 11th and 12th parameter are the mass squared differences of the
atmospheric and solar neutrino oscillations respectively.
\item The 13th parameter is the mass hierarchy. "i" means inverted and "r" or
"n" means normal.
\item The 14th parameter the neutrino Yukawa coupling matrix
\item The 15th parameter is the vev of the Higgs coupling
to the neutrinos.
\end{itemize}


\subsubsection{RGEGetY$\nu$}

\function{RGEGetY$\nu$[\param{$\left(Y_\nu\right)_{33}$},\param{ratio}]} returns a suggestion of
the Yukawa matrix of $\nu$ at the GUT scale.  The first parameter is the mass of
the heaviest neutrino and the second parameter specifies the mass ratio between
the neutrinos. The result is a diagonal hierarchical matrix.

\subsubsection{RGEGetYd}

\function{RGEGetYd[\param{$y_1$}, \param{$y_2$}, \param{$y_3$}, \param{$\theta_{12}$},\param{$\theta_{13}$}, \param{$\theta_{23}$},\param{$\delta$},\param{$\delta_e$},\param{$\delta_\mu$},\param{$\delta_\tau$},\param{$\varphi_1$},\param{$\varphi_2$}]} returns a suggestion for
    Yd.
\begin{itemize}
\item The first 3 parameters are the eigenvalues of $Y_d$.
\item The next 9 parameters are the mixing parameters.
\end{itemize}


\subsubsection{RGEGetYe}

\function{RGEGetYe[\param{Yukawa $\tau$}]} returns a suggestion for the Yukawa
matrix of the charged leptons at the GUT scale.  The parameter is the Yukawa
coupling of the $\tau$. The suggested matrix is diagonal.

\subsubsection{RGEGet$\kappa$}

\function{RGEGet$\kappa$[\param{$\theta_{12}$},\param{$\theta_{13}$},\param{$\theta_{23}$},\param{$\delta$},\param{$\delta_e$},\param{$\delta_\mu$},\param{$\delta_\tau$},\param{$\varphi_1$},\param{$\varphi_2$},\param{Mlightest},\param{$\Delta
m^2_\mathrm{atm}$},\param{$\Delta m^2_\mathrm{sol}$},\param{mass hierarchy},\param{vu}]} returns a suggestion for
    $\kappa$, the coupling of the dimension 5 operator, in
$\mathrm{GeV}^{-1}$.
\begin{itemize}
\item The first 9 parameters specify the mixing matrix.
\item The 10th parameter is the mass of the lightest neutrino.
\item The 11th and 12th parameter are the mass squared differences of the
atmospheric and solar neutrino oscillations, respectively.
\item The 13th parameter is the mass hierarchy. "i" means inverted and "r" or
"n" means normal.
\item The 14th parameter is the vev of the Higgs coupling
to the neutrinos.
\end{itemize}





\subsection[\package{RGEParameters}]{\package{REAP`RGEParameters`}}
\package{RGEParameters} contains measured parameters of the SM like mixing
angles, masses and coupling constants. %\cite{PDBook}
  
\subsubsection{RGEMass}

\function{RGEMass[\param{particle name}]} returns the mass of the given particle.

\begin{verbatim}
RGEMass["t"]
\end{verbatim}
returns 174, the mass of the top quark.


\subsubsection{gMZ}

\function{RGEgMZ[\param{i}]} returns the value of the coupling constant i at the
mass of the Z boson.

\begin{verbatim}
RGEgMZ[3]
\end{verbatim}
returns the coupling constant of QCD at mZ.





\subsection[\package{RGEUtilities}]{\package{REAP`RGEUtilities`}}
This package contains some functions needed by \package{RGESM},\package{RGEMSSM}
and \package{2HDM}.




\subsection[\package{RGETakagi}]{\package{REAP`RGETakagi`}}
This package contains a function implementing the Takagi diagonalization, which was implemented by Vinzenz Maurer following the algorithm described in Ref.~\cite{Hahn:2006xx}.


\subsubsection{RGETakagiDecomposition}

\function{RGETakagiDecomposition[\param{M}]} performs a Takagi decomposition of \param{M} and returns in a list the unitary matrix \param{u} and the diagonalised matrix \param{d}, i.e. {\param{u},\param{d}} with \param{d}=\param{u}.\param{M}.param{u}$^T$. This was implemented by Vinzenz Maurer following the algorithm described in arXiv:physics/0607103 [physics.comp-ph].

\begin{verbatim}
  {u,d}=RGETakagiDecomposition[M];
\end{verbatim}




%%%%%%%%%%%%%%%%%%%%%%%%%%%%%%%%%%%%%%%%%%%%%%%%%%%%%%%%%%%%%%%%%%%%%%%%%%%%%%%%%%%
%% models
%%%%%%%%%%%%%%%%%%%%%%%%%%%%%%%%%%%%%%%%%%%%%%%%%%%%%%%%%%%%%%%%%%%%%%%%%%%%%%%%%%%

\section{Models\label{sec:Models}}

%\subsection{\package{RGEToyModel}}
%This package contains a toy model for testing purposes.

\subsection{Standard Model (SM)}

\subsubsection[\package{RGESM}]{\package{REAP`RGESM`}}
This package contains the Standard Model extended by an arbitrary number of right-handed neutrinos (SM)
to 1 loop order.  It is possible to automatically find transitions where heavy
neutrinos are integrated out.  However, quarks are not integrated out.

\vspace{2ex}
Options:
\begin{itemize}
\item RGEIntegratedOut\ is the number of right-handed neutrinos which are
  integrated out. (default: 0)
\item RGESearchTransition\ enables/disables the automatic search for
  transitions, i.e.\ automatically integrating out right-handed
neutrinos. (default: True)
\item RGEThresholdFactor\ determines where heavy degrees of freedom are integrated
  out: RGEThresholdFactor*Mass=Scale where degree of freedom is integrated
  out. (default: 1)
\item RGE$\lambda$\ sets the initial value of the quartic Higgs coupling
  $\lambda$ which is used when changing from the MSSM to the SM at
  $M_\mathrm{SUSY}$.  (default: 0.5)
\item RGEvEW\ is the vev of the Higgs at the electroweak scale
  in GeV (default: 246).  It is treated as a constant, i.e.\ its running
  is not taken into account.

\end{itemize}

Options used by \function{RGESetInitial}:

If the default values of all parameters are used, the resulting parameters will
be compatible to the experimental data at the Z boson mass. The number of right-handed neutrinos is given by the initial conditions. There
is no need to specify the number of neutrinos somewhere else.
\begin{itemize}
\item RGEM$\nu$r\ is the mass matrix of the right-handed neutrinos.
  If this parameter is specified, it also determines the light neutrino
  mass matrix via the see-saw formula (together with RGEY$\nu$).  Thus,
  RGEMassHierarchy, RGEMlightest, RGE$\Delta$m2atm, RGE$\Delta$m2sol,
  RGE$\varphi$1, RGE$\varphi$2, RGE$\delta$, RGE$\delta$e,
  RGE$\delta\mu$, RGE$\delta\tau$, RGE$\theta$12, RGE$\theta$13, and
  RGE$\theta$23 do not have any effect in this case.
  
\item RGEMassHierarchy\ is the hierarchy of the neutrino masses; "r" or "n"
  means normal hierarchy, "i" means inverted hierarchy (default: "r").
  
\item RGEMlightest \ is the mass of the lightest neutrino in eV (default: $\mathcal{O}(0.01)
  \eV$).
  \ The default of RGEMlightest depends on the
model. It is chosen in such a way, that the parameters are compatible with the
experimental data.
\item RGEY$\nu$\ is the neutrino Yukawa matrix in ``RL convention''. This option overrides the
  built-in Yukawa matrix, i.e.\ RGEY$\nu33$ and RGEY$\nu$Ratio do not have any
  effect. 
    (default: RGEGetY$\nu$(RGEY$\nu33$, RGE$Y\nu$Ratio))
  
\item RGEY$\nu$33\ is the (3,3) entry in the neutrino Yukawa matrix at the GUT
  scale.\ The default value depends on the
  model and it is chosen in such a way, that it is compatible with the
  experimental data (default: $\mathcal{O}(1)$).
  
\item RGEY$\nu$Ratio\ determines the relative value of the neutrino Yukawa couplings.\ The default value depends on the
  model and it is chosen in such a way, that it is compatible with the
  experimental data (default: $\mathcal{O}(1)$).
\item RGEYd\ is the Yukawa matrix of the down-type quarks.
  If this parameter is given, RGEyd, RGEys, RGEyb, RGEq$\varphi$1,
  RGEq$\varphi$2, RGEq$\delta$, RGEq$\delta$e, RGEq$\delta\mu$,
  RGEq$\delta\tau$, RGEq$\theta$12, RGEq$\theta$13, and RGEq$\theta$23
  are ignored.
  
\item RGEYe\ is the charged lepton Yukawa matrix.
  If this parameter is given, RGEye, RGEy$\mu$ and RGEy$\tau$ are
  ignored.
    \\ (default:
  RGEGetYe(0.8*Mass$\left[\mathrm{"}\tau\mathrm{"}\right]$*Sqrt[2]/RGEvd))
  
\item RGEYu\ is the Yukawa matrix of the up-type quarks.
  If this parameter is given, RGEyu, RGEyc and RGEyt are ignored;
  it is recommended not to use RGEq$\varphi$1, RGEq$\varphi$2,
  RGEq$\delta$, RGEq$\delta$e, RGEq$\delta\mu$, RGEq$\delta\tau$,
  RGEq$\theta$12, RGEq$\theta$13, and RGEq$\theta$23 in this case, since
  they are not necessarily equal to the CKM mixing parameters.
\item RGE$\Delta$m2atm\ is the atmospheric mass squared difference (default: $ \mathcal{O}(10^{-3}) \eV^2$).\ The default value depends on the
  model and it is chosen in such a way, that it is compatible with the
  experimental data.
  
\item RGE$\Delta$m2sol\ is the solar mass squared difference (default:
  $\mathcal{O}(10^{-4}) \eV^2$).\ The default value depends on the
  model and it is chosen in such a way, that it is compatible with the
  experimental data.
\item RGE$\varphi$1\ and RGE$\varphi2$ are the Majorana CP phases $\varphi_1$ and $\varphi_2$ in radians (default: $0$).
  
\item RGE$\delta$\ is the Dirac CP phase $\delta$ in radians (default: $0$).
\item RGE$\delta$e, RGE$\delta\mu$ and RGE$\delta\tau$ are the unphysical phases $\delta_e$,
  $\delta_\mu$ and $\delta_\tau$ (default: $0$). 
\item RGE$\kappa$\ is the coupling of the dimension 5 neutrino mass operator.
  
\item RGE$\lambda$\ is the quartic Higgs self-coupling (default: 0.5).  We use the
  convention that the corresponding term in the Lagrangian is
  $-\frac{\lambda}{4} (\phi^\dagger \phi)^2$.
  
\item RGE$\theta$12, RGE$\theta13$ and RGE$\theta23$ are the angles $\theta_{12}$, $\theta_{13}$
and $\theta_{23}$ of the MNS matrix in radians. (default: $\theta_{13}=0$ and
$\theta_{23}=\frac{\pi}{4}$). The default of $\theta_{12}$ depends on the
model. It is chosen in such a way, that the parameters are compatible with the
experimental data. 
\item RGEg RGEg is the coupling constants of SU(5)
  
\item RGEg1, RGEg2 and RGEg3 are the coupling constants of U$(1)_\mathrm{Y}$,
  SU$(2)_\mathrm{L}$ and SU$(3)_\mathrm{C}$, respectively.  GUT charge
  normalization is used for $g_1$.
  
\item RGEm RGEm is the Higgs mass
  
\item RGEq$\varphi$1\ and RGEq$\varphi2$ are the unphysical phases $\varphi_1$ and $\varphi_2$ of the
 CKM matrix which correspond to the Majorana phases in the MNS matrix (default: $0$).
\item RGEq$\delta$\ is the Dirac CP phase $\delta$ of the CKM matrix.
\item RGEq$\delta$e, RGEq$\delta\mu$ and RGE$\delta\tau$ are the unphysical phases $\delta_e$,
$\delta_\mu$ and $\delta_\tau$ of the CKM matrix (default: $0$).
\item RGEq$\theta$12, RGEq$\theta13$ and RGEq$\theta23$ are the angles of the CKM matrix. 
\item RGEyd, RGEys and RGEyb are the Yukawa coupling of the down-type quarks $d$,
  $s$ and $b$.
\item RGEye, RGEy$\mu$ and RGEy$\tau$ are the Yukawa couplings of the charged
  leptons $e$, $\mu$ and $\tau$.
\item RGEyu, RGEyc and RGEyt are the Yukawa couplings of the up-type quarks $u$,
  $c$ and $t$.

\end{itemize}

Parameters accepted by \function{RGEGetSolution}:
\begin{itemize}
\item 
RGECoupling is used to get the coupling constants.
\item 
RGEGWCondition returns the Gildener Weinberg condition.
\item 
RGEGWConditions returns all Gildener Weinberg conditions.
\item 
RGEM1Tilde returns the effective light-neutrino mass $\widetilde{m}_1 =
\frac{\left(m_D m_D^\dagger\right)_{11}}{M_1}=\frac{(Y_\nu Y_\nu^\dagger)_{11} 
v^2}{2 M_1}$ which is commonly used in thermal leptogenesis. $\widetilde{m}_1$ is
given in eV.
\item 
RGEM$\nu$ is used to get the mass matrix of the left-handed neutrinos.
\item 
RGEM$\nu$r is the mass matrix of the right-handed neutrinos.
\item 
RGEMd is used to get the mass matrix of the down-type quarks.
\item 
RGEMe is used to get the mass matrix of the charged leptons.
\item 
RGEMixingParameters returns the mixing parameters in the leptonic sector as they
are returned by MNSParameters: $\left\{\left\{\theta_{12},\theta_{13},\theta_{23},\delta,\delta_e,\delta_\mu,\delta_\tau,\varphi_1,\varphi_2\right\},\left\{y_1,y_2,y_3\right\},\left\{y_e,y_\mu,y_\tau\right\}\right\}$
\item 
RGEMu is used to get the mass matrix of the up-type quarks.
\item 
RGEPoleMTop is used to get the pole mass of the top quark in the
$\overline{\text{MS}}$ scheme. The pole mass term of the top quark is given by
\begin{equation}
m_t^\text{Pole} = m_t(m_t) \cdot (1 + \frac{4\alpha_s}{3\pi})
\end{equation}
to 1-loop order.
\item 
RGERawM$\nu$r is used to get the raw mass matrix of the right-handed neutrinos.
\item 
RGERaw is used to get the raw values of all parameters. A raw parameter is
the internal representation of the parameter
\item 
RGERawY$\Delta$ is used to get the Yukawa coupling matrix of the coupling to the Higgs triplet.
\item 
RGERawY$\nu$ is used to get the raw Yukawa coupling matrix of the
neutrinos.
\item 
RGEAll returns all parameters of the model.
\item 
RGEVEVratio returns the squared ratio of $v_R$ over the EW symmetry breaking scale.
\item 
RGEVEVratios returns the squared ratio of $v_R$ over the EW symmetry breaking scale.
\item 
RGEY$\nu$ is used to get the Yukawa coupling matrix of the neutrinos.
\item 
RGEYd is used to get the Yukawa coupling matrix of the down-type quarks.
\item 
 RGEYe is used to get the Yukawa coupling matrix of the charged leptons.
\item 
RGEYu is used to get the Yukawa coupling matrix of the up-type quarks.
\item 
RGE$\alpha$ is used to get the fine structure constants.
\item 
RGE$\epsilon$1 is used to get the CP asymmetry \cite{Covi:1996wh} for 
leptogenesis for $M_1 \ll M_2, M_3$,
   \begin{equation}
     \epsilon_1=\frac{3}{8\pi}\frac{M_1}{v^2}
     \frac{\sum_{f,g}\im\left[\left(Y_\nu\right)_{1f}\left(Y_\nu\right)_{1g}
     \left(m^*_\nu\right)_{fg}\right]}{\left(Y_\nu Y_\nu^\dagger\right)_{11}}\; .\label{eq:RGEepsilonSM}
   \end{equation}
   Eq.~\eqref{eq:RGEepsilonSM} 
also holds if there are additional contributions to the neutrino mass operator, as it is for example 
the case in the type II see-saw mechanism \cite{Antusch:2004xy}.
\item 
RGE$\epsilon$1Max is used to get the upper bound \cite{Buchmuller:2003gz} 
on the CP asymmetry for leptogenesis in the type I see-saw mechanism 
for $M_1 \ll M_2, M_3$,
  \begin{equation}
    \epsilon_1^\text{max}=\frac{3}{8\pi}\frac{M_1 m_3}{v^2}\left[1-\frac{m_1}{m_3}\left(1+\frac{m_3^2-m_1^2}{\widetilde{m}_1}\right)^\frac{1}{2}\right]\;.
  \end{equation}
\item 
RGE$\lambda$ is used to get the quartic Higgs self coupling.

\end{itemize}

\subsubsection[\package{RGESM0N}]{\package{REAP`RGESM0N`}}
This package contains the Standard Model without any right-handed neutrinos
(SM0N) to 1 loop order.

\vspace{2ex} It has the same parameters and options as \package{RGESM},
with the following exceptions:  The only
missing options are RGEIntegratedOut, RGESearchTransition, RGEThresholdFactor,
RGEPrecision and RGEMaxNumberIterations, which are used to control the process
of integrating out.  Besides, RGEM$\nu$r and RGEY$\nu$ are no parameters of
\function{RGESetInitial}, and RGE$\epsilon$Max, RGE$\epsilon$, RGEM1Tilde, RGERawM$\nu$r and RGERawY$\nu$ are not accepted as
parameters by \function{RGEGetSolution}.  \function{RGESetInitial} has
an additional option: RGESuggestion can be used to choose between
different sets of default values, ``GUT'' (default) and ``MZ''.  They
refer to typical parameter values at the GUT scale or at the $Z$ mass,
respectively.

\subsubsection[\package{RGESMDirac}]{\package{REAP`RGESMDirac`}}
This package contains the Standard Model with Dirac Neutrinos to 1 loop order.

\vspace{2ex} It has the same parameters and options as \package{RGESM},
with the following exceptions:  The only
missing options are RGEIntegratedOut, RGESearchTransition, RGEThresholdFactor,
RGEPrecision and RGEMaxNumberIterations, which are used to control the process
of integrating out.  In addition RGE$\kappa$ and RGEM$\nu$r are no parameters of
\function{RGESetInitial} and RGEMixingParameters, RGE$\epsilon$Max, RGE$\epsilon$, RGEM1Tilde, RGERawM$\nu$r, RGERawY$\nu$ and RGE$\kappa$ are not
accepted as parameters by \function{RGEGetSolution}.
\function{RGESetInitial} has an additional option: RGESuggestion can be
used to choose between different sets of default values, ``GUT''
(default) and ``MZ''.  They refer to typical parameter values at the GUT
scale or at the $Z$ mass, respectively.

%%%%%%%%%%%%%%%%%%%%%%%%%%%%%%%%%%%%%%%%%%%%%%%%%%%%%%%%%%%%%%%%%%%%%%%%%%%%%%%%%%%%%%%%%%%%%%%%%%%%%%%%%%%%%%%%%%%%%%%%%%%%%

\subsection{Minimal Supersymmetric Standard Model (MSSM)}

\subsubsection[\package{RGEMSSM}]{\package{REAP`RGEMSSM`}}
This package contains the Minimal Supersymmetric Standard Model extended by an arbitrary number of 
right-handed neutrinos (MSSM) to 1 and 2 loop order.

It is possible to automatically find transitions where heavy neutrinos are
integrated out.  But neither quarks are integrated out nor MSSM thresholds are
considered. 

Options:
\begin{itemize}
\item RGEIntegratedOut\ is the number of right-handed neutrinos which are
  integrated out. (default: 0)
\item RGEModelVariant\ is a switch to change between different versions, but
  there are only two versions right now: 1Loop and 2Loop (default: 1Loop).
\item RGESearchTransition\ enables/disables the automatic search for
  transitions, i.e.\ automatically integrating out right-handed
neutrinos. (default: True)
\item RGEThresholdFactor\ determines where heavy degrees of freedom are integrated
  out: RGEThresholdFactor*Mass=Scale where degree of freedom is integrated
  out. (default: 1)
\item RGE$\Gamma$d\ parameterizes the finite supersymmetric threshold corrections
\begin{equation}
Y_d^\mathrm{SM} = Y_d^\mathrm{MSSM} (1 + \mathrm{RGE}\Gamma\mathrm{d}) * \cos(\beta)
\end{equation}
in the basis, in which $Y_u$ is diagonal and the left-handed mixing is entirely contained in $Y_d$. It is related to the notation in \cite{Blazek:1995nv}
\begin{equation}
\mathrm{RGE}\Gamma\mathrm{d} \equiv \epsilon (V_{CKM} \Gamma_D^\dagger V_{CKM}^\dagger +\Gamma_U^\dagger)
\end{equation}
with $\epsilon = \tan\beta/(16 \pi^2)$ and $\Gamma_{U,D}$ defines as in Eq.~(1) of Ref.~\cite{Blazek:1995nv}. 
\item RGE$\Gamma$e\ parameterizes the finite supersymmetric threshold corrections
\begin{equation}
Y_e^\mathrm{SM} = Y_e^\mathrm{MSSM} (1 + \mathrm{RGE}\Gamma\mathrm{e}) * \cos\beta
\end{equation}
in the basis, in which the Weinberg operator $\kappa$ is diagonal and the left-handed mixing is entirely contained in $Y_e$. It is defined in a similar way to RGE$\Gamma$d.
\item RGEtan$\beta$\ is the value of $\tan\beta=\frac{v_u}{v_d}$, the ratio of the 2
  Higgs vevs (default: 50).
\item RGEvEW\ is the combination $v=\sqrt{v_u^2+v_d^2}$ of the Higgs vevs at the
  electroweak scale in GeV (default: 246).  The vevs are treated as
  constants, i.e.\ their running is not taken into account.

\end{itemize}

Options used by \function{RGESetInitial}:

If the default values of all parameters are used, the resulting parameters will
be compatible to the experimental data at the Z boson mass. The number of right-handed neutrinos is given by the initial conditions. There
is no need to specify the number of neutrinos somewhere else.
\begin{itemize}
\item RGEM$\nu$r\ is the mass matrix of the right-handed neutrinos.
  If this parameter is specified, it also determines the light neutrino
  mass matrix via the see-saw formula (together with RGEY$\nu$).  Thus,
  RGEMassHierarchy, RGEMlightest, RGE$\Delta$m2atm, RGE$\Delta$m2sol,
  RGE$\varphi$1, RGE$\varphi$2, RGE$\delta$, RGE$\delta$e,
  RGE$\delta\mu$, RGE$\delta\tau$, RGE$\theta$12, RGE$\theta$13, and
  RGE$\theta$23 do not have any effect in this case.
  
\item RGEMassHierarchy\ is the hierarchy of the neutrino masses; "r" or "n"
  means normal hierarchy, "i" means inverted hierarchy (default: "r").
  
\item RGEMlightest \ is the mass of the lightest neutrino in eV (default: $\mathcal{O}(0.01)
  \eV$).
  \ The default of RGEMlightest depends on the
model. It is chosen in such a way, that the parameters are compatible with the
experimental data.
\item RGEY$\nu$\ is the neutrino Yukawa matrix in ``RL convention''. This option overrides the
  built-in Yukawa matrix, i.e.\ RGEY$\nu33$ and RGEY$\nu$Ratio do not have any
  effect. 
    (default: RGEGetY$\nu$(RGEY$\nu33$, RGE$Y\nu$Ratio))
  
\item RGEY$\nu$33\ is the (3,3) entry in the neutrino Yukawa matrix at the GUT
  scale.\ The default value depends on the
  model and it is chosen in such a way, that it is compatible with the
  experimental data (default: $\mathcal{O}(1)$).
  
\item RGEY$\nu$Ratio\ determines the relative value of the neutrino Yukawa couplings.\ The default value depends on the
  model and it is chosen in such a way, that it is compatible with the
  experimental data (default: $\mathcal{O}(1)$).
\item RGEYd\ is the Yukawa matrix of the down-type quarks.
  If this parameter is given, RGEyd, RGEys, RGEyb, RGEq$\varphi$1,
  RGEq$\varphi$2, RGEq$\delta$, RGEq$\delta$e, RGEq$\delta\mu$,
  RGEq$\delta\tau$, RGEq$\theta$12, RGEq$\theta$13, and RGEq$\theta$23
  are ignored.
  
\item RGEYe\ is the charged lepton Yukawa matrix.
  If this parameter is given, RGEye, RGEy$\mu$ and RGEy$\tau$ are
  ignored.
    \\ (default:
  RGEGetYe(0.8*Mass$\left[\mathrm{"}\tau\mathrm{"}\right]$*Sqrt[2]/RGEvd))
  
\item RGEYu\ is the Yukawa matrix of the up-type quarks.
  If this parameter is given, RGEyu, RGEyc and RGEyt are ignored;
  it is recommended not to use RGEq$\varphi$1, RGEq$\varphi$2,
  RGEq$\delta$, RGEq$\delta$e, RGEq$\delta\mu$, RGEq$\delta\tau$,
  RGEq$\theta$12, RGEq$\theta$13, and RGEq$\theta$23 in this case, since
  they are not necessarily equal to the CKM mixing parameters.
\item RGE$\Delta$m2atm\ is the atmospheric mass squared difference (default: $ \mathcal{O}(10^{-3}) \eV^2$).\ The default value depends on the
  model and it is chosen in such a way, that it is compatible with the
  experimental data.
  
\item RGE$\Delta$m2sol\ is the solar mass squared difference (default:
  $\mathcal{O}(10^{-4}) \eV^2$).\ The default value depends on the
  model and it is chosen in such a way, that it is compatible with the
  experimental data.
\item RGE$\varphi$1\ and RGE$\varphi2$ are the Majorana CP phases $\varphi_1$ and $\varphi_2$ in radians (default: $0$).
  
\item RGE$\delta$\ is the Dirac CP phase $\delta$ in radians (default: $0$).
\item RGE$\delta$e, RGE$\delta\mu$ and RGE$\delta\tau$ are the unphysical phases $\delta_e$,
  $\delta_\mu$ and $\delta_\tau$ (default: $0$). 
\item RGE$\kappa$\ is the coupling of the dimension 5 neutrino mass operator.
  
\item RGE$\theta$12, RGE$\theta13$ and RGE$\theta23$ are the angles $\theta_{12}$, $\theta_{13}$
and $\theta_{23}$ of the MNS matrix in radians. (default: $\theta_{13}=0$ and
$\theta_{23}=\frac{\pi}{4}$). The default of $\theta_{12}$ depends on the
model. It is chosen in such a way, that the parameters are compatible with the
experimental data. 
\item RGEg RGEg is the coupling constants of SU(5)
  
\item RGEg1, RGEg2 and RGEg3 are the coupling constants of U$(1)_\mathrm{Y}$,
  SU$(2)_\mathrm{L}$ and SU$(3)_\mathrm{C}$, respectively.  GUT charge
  normalization is used for $g_1$.
  
\item RGEm RGEm is the Higgs mass
  
\item RGEq$\varphi$1\ and RGEq$\varphi2$ are the unphysical phases $\varphi_1$ and $\varphi_2$ of the
 CKM matrix which correspond to the Majorana phases in the MNS matrix (default: $0$).
\item RGEq$\delta$\ is the Dirac CP phase $\delta$ of the CKM matrix.
\item RGEq$\delta$e, RGEq$\delta\mu$ and RGE$\delta\tau$ are the unphysical phases $\delta_e$,
$\delta_\mu$ and $\delta_\tau$ of the CKM matrix (default: $0$).
\item RGEq$\theta$12, RGEq$\theta13$ and RGEq$\theta23$ are the angles of the CKM matrix. 
\item RGEyd, RGEys and RGEyb are the Yukawa coupling of the down-type quarks $d$,
  $s$ and $b$.
\item RGEye, RGEy$\mu$ and RGEy$\tau$ are the Yukawa couplings of the charged
  leptons $e$, $\mu$ and $\tau$.
\item RGEyu, RGEyc and RGEyt are the Yukawa couplings of the up-type quarks $u$,
  $c$ and $t$.

\end{itemize}

Parameters accepted by \function{RGEGetSolution}:
\begin{itemize}
\item 
RGECoupling is used to get the coupling constants.
\item 
RGEGWCondition returns the Gildener Weinberg condition.
\item 
RGEGWConditions returns all Gildener Weinberg conditions.
\item 
RGEM1Tilde returns the effective light-neutrino mass $\widetilde{m}_1 =
\frac{\left(m_D m_D^\dagger\right)_{11}}{M_1}=\frac{(Y_\nu Y_\nu^\dagger)_{11} 
v^2}{2 M_1}$ which is commonly used in thermal leptogenesis. $\widetilde{m}_1$ is
given in eV.
$v$ is the vev of the Higgs doublet which couples to the neutrinos.
\item 
RGEM$\nu$ is used to get the mass matrix of the left-handed neutrinos.
\item 
RGEM$\nu$r is the mass matrix of the right-handed neutrinos.
\item 
RGEMd is used to get the mass matrix of the down-type quarks.
\item 
RGEMe is used to get the mass matrix of the charged leptons.
\item 
RGEMixingParameters returns the mixing parameters in the leptonic sector as they
are returned by MNSParameters: $\left\{\left\{\theta_{12},\theta_{13},\theta_{23},\delta,\delta_e,\delta_\mu,\delta_\tau,\varphi_1,\varphi_2\right\},\left\{y_1,y_2,y_3\right\},\left\{y_e,y_\mu,y_\tau\right\}\right\}$
\item 
RGEMu is used to get the mass matrix of the up-type quarks.
\item 
RGEPoleMTop is used to get the pole mass of the top quark in the
$\overline{\text{MS}}$ scheme. The pole mass term of the top quark is given by
\begin{equation}
m_t^\text{Pole} = m_t(m_t) \cdot (1 + \frac{4\alpha_s}{3\pi})
\end{equation}
to 1-loop order.
\item 
RGERawM$\nu$r is used to get the raw mass matrix of the right-handed neutrinos.
\item 
RGERaw is used to get the raw values of all parameters. A raw parameter is
the internal representation of the parameter
\item 
RGERawY$\Delta$ is used to get the Yukawa coupling matrix of the coupling to the Higgs triplet.
\item 
RGERawY$\nu$ is used to get the raw Yukawa coupling matrix of the
neutrinos.
\item 
RGEAll returns all parameters of the model.
\item 
RGEVEVratio returns the squared ratio of $v_R$ over the EW symmetry breaking scale.
\item 
RGEVEVratios returns the squared ratio of $v_R$ over the EW symmetry breaking scale.
\item 
RGEY$\nu$ is used to get the Yukawa coupling matrix of the neutrinos.
\item 
RGEYd is used to get the Yukawa coupling matrix of the down-type quarks.
\item 
 RGEYe is used to get the Yukawa coupling matrix of the charged leptons.
\item 
RGEYu is used to get the Yukawa coupling matrix of the up-type quarks.
\item 
RGE$\alpha$ is used to get the fine structure constants.
\item 
RGE$\epsilon$1 is used to get the CP asymmetry \cite{Covi:1996wh} for 
leptogenesis for $M_1 \ll M_2, M_3$,
   \begin{equation}
     \epsilon_1=\frac{3}{4\pi}\frac{M_1}{v^2}
     \frac{\sum_{f,g}\im\left[\left(Y_\nu\right)_{1f}\left(Y_\nu\right)_{1g}
     \left(m^*_\nu\right)_{fg}\right]}{\left(Y_\nu Y_\nu^\dagger\right)_{11}}\;. \label{eq:RGEepsilonMSSM}
   \end{equation}
   Eq.~\eqref{eq:RGEepsilonMSSM} 
also holds if there are additional contributions to the neutrino mass operator, as it is for example 
the case in the type II see-saw mechanism \cite{Antusch:2004xy}.
\item 
RGE$\epsilon$1Max is used to get the upper bound \cite{Buchmuller:2003gz} 
on the CP asymmetry for leptogenesis in the type I see-saw mechanism 
for $M_1 \ll M_2, M_3$,
  \begin{equation}
    \epsilon_1^\text{max}=\frac{3}{4\pi}\frac{M_1 m_3}{v^2}\left[1-\frac{m_1}{m_3}\left(1+\frac{m_3^2-m_1^2}{\widetilde{m}_1}\right)^\frac{1}{2}\right]\;.
  \end{equation}
\item 
RGE$\kappa$ is used to get $\kappa$.

\end{itemize}

\subsubsection[\package{RGEMSSM0N}]{\package{REAP`RGEMSSM0N`}}
This package contains the Minimal Supersymmetric Standard Model (MSSM) without
any right-handed neutrinos to 1 and 2 loop order.

\vspace{2ex} It has the same parameter and options as \package{RGEMSSM}.  The
only missing options are RGEIntegratedOut, RGESearchTransition,
RGEThresholdFactor, RGEPrecision and RGEMaxNumberIterations, which are used to
control the process of integrating out.  In addition RGEM$\nu$r and RGEY$\nu$
are no parameters of \function{RGESetInitial} and RGE$\epsilon$Max,
RGE$\epsilon$, RGEM1Tilde, RGERawM$\nu$r and RGERawY$\nu$
are not accepted as parameters by \function{RGEGetSolution}.


\subsubsection[\package{RGEMSSMDirac}]{\package{REAP`RGEMSSMDirac`}}
This package contains the MSSM with Dirac Neutrinos to 1 loop order and 2 loop
order.

\vspace{2ex} It has the same parameter and options as \package{RGEMSSM}.  The
only missing options are RGEIntegratedOut, RGESearchTransition,
RGEThresholdFactor, RGEPrecision and RGEMaxNumberIterations, which are used to
control the process of integrating out.  In addition RGEM$\nu$r and RGE$\kappa$
are no parameter of \function{RGESetInitial} and RGEMixingParameters, RGE$\epsilon$Max,
RGE$\epsilon$, RGEM1Tilde, RGERawM$\nu$r, RGERawY$\nu$ and
RGE$\kappa$ are not accepted as parameters by \function{RGEGetSolution}.

\subsection{Two Higgs Doublet Model (2HDM)}

\subsubsection[\package{RGE2HDM}]{\package{REAP`RGE2HDM`}}
This package contains the Two Higgs Doublet Model (2HDM) with a $\mathbb{Z}_2$
symmetry extended by an arbitrary number of right-handed neutrinos. The charged leptons always couple to the first Higgs. In addition
there are right-handed neutrinos. The $\beta$-functions are to 1 loop order. The
vevs of the Higgs fields are $v_1=\braket{\phi_1}$ and
$v_2=\braket{\phi_2}$. They obey $v^2=v_1^2+v_2^2$, $v_1=v\cos\beta$ and
$v_2=v\sin\beta$, where v is the v.e.v. of the SM Higgs and $\beta$
($\tan\beta=\frac{v_2}{v_1}$, $\beta\in\left(0,\frac{\pi}{2}\right)$) is used to
parametrize the Higgs vevs.

Thus there are 2 dimension 5 operators which give mass to the light neutrinos.
\begin{equation*}
  \mathcal{L}_\kappa^{(ii)}=\frac{1}{4}\kappa_{gf}^{(ii)}\overline{l_{Lc}^C}^g\epsilon^{cd}\phi_d^{(i)}l_{Lb}^f\epsilon^{ba}\phi_a^{(i)}+\mathrm{h.c.}\quad\quad(i=1\;
  or\; 2)
\end{equation*}
The Higgs potential is
\begin{eqnarray*}
\mathcal{L}_{2Higgs}&=& -\frac{\lambda_1}{4}
\left(\phi^{(1)\dagger}\phi^{(1)}\right)^2 -\frac{\lambda_2}{4}
\left(\phi^{(2)\dagger}\phi^{(2)}\right)^2\\
&&-\lambda_3\left(\phi^{(1)\dagger}\phi^{(1)}\right)\left(\phi^{(2)\dagger}\phi^{(2)}\right)
-\lambda_4\left(\phi^{(1)\dagger}\phi^{(2)}\right)\left(\phi^{(2)\dagger}\phi^{(1)}\right)
\\ &&
-\left[\frac{\lambda_5}{4}\left(\phi^{(1)\dagger}\phi^{(2)}\right)^2+\mathrm{h.c.}\right]\\
\end{eqnarray*}

The charged leptons always couple to the first Higgs field and the coupling of
the other fields to the Higgs fields is determined by RGEModelOptions.

It is possible to automatically find transitions where heavy neutrinos are
integrated out.  But no other particles are integrated out.

Options:
\begin{itemize}
\item RGEIntegratedOut\ is the number of right-handed neutrinos which are
  integrated out. (default: 0)
\item RGESearchTransition\ enables/disables the automatic search for
  transitions, i.e.\ automatically integrating out right-handed
neutrinos. (default: True)
\item RGEThresholdFactor\ determines where heavy degrees of freedom are integrated
  out: RGEThresholdFactor*Mass=Scale where degree of freedom is integrated
  out. (default: 1)
\item RGE$\lambda$1, RGE$\lambda2$, RGE$\lambda3$, RGE$\lambda4$ and RGE$\lambda5$
  set the initial values of the couplings $\lambda_i$
  $i\in\{1,\dots,5\}$ in the Higgs potential which are used when
  changing from the MSSM to the 2HDM at $M_\mathrm{SUSY}$
  (default: $\lambda_1=\lambda_2=0.75$, $\lambda_3=\lambda_4=0.2$, $\lambda_5=0.25$).
\item RGEtan$\beta$ is the value of $\tan\beta=\frac{v_2}{v_1}$, the ratio of the 2
  Higgs vevs (default: 50).
\item RGEvEW\ is the combination $v=\sqrt{v_1^2+v_2^2}$ of the Higgs vevs at the
  electroweak scale in GeV (default: 246).  The vevs are treated as
  constants, i.e.\ their running is not taken into account.
\item RGEz$\nu$\ is a list defining the Higgs the neutrinos are coupling to. If the $n^{th}$ component is one,
    the Higgs couples to the neutrinos. If it is 0, it won't
    couple (default: $\{0,1\}$). The charged leptons always couple to the
  first Higgs.
\item RGEzd\ is a list defining the Higgs the down-type quarks are coupling to. If the $n^{th}$ component is one,
    the Higgs couples to the down-type quarks. If it is 0, it won't
    couple (default: $\{1,0\}$).
\item RGEzu\ is a list defining the Higgs the up-type quarks are coupling to. If the $n^{th}$ component is one,
    the Higgs couples to the up-type quarks. If it is 0, it won't
    couple (default: $\{0,1\}$).

\end{itemize}

Options used by \function{RGESetInitial}:

If the default values of all parameters are used, the resulting parameters will
be compatible to the experimental data at the Z boson mass. The number of right-handed neutrinos is given by the initial conditions. There
is no need to specify the number of neutrinos somewhere else.
\begin{itemize}
\item RGEM$\nu$r\ is the mass matrix of the right-handed neutrinos.
  If this parameter is specified, it also determines the light neutrino
  mass matrix via the see-saw formula (together with RGEY$\nu$).  Thus,
  RGEMassHierarchy, RGEMlightest, RGE$\Delta$m2atm, RGE$\Delta$m2sol,
  RGE$\varphi$1, RGE$\varphi$2, RGE$\delta$, RGE$\delta$e,
  RGE$\delta\mu$, RGE$\delta\tau$, RGE$\theta$12, RGE$\theta$13, and
  RGE$\theta$23 do not have any effect in this case.
  
\item RGEMassHierarchy\ is the hierarchy of the neutrino masses; "r" or "n"
  means normal hierarchy, "i" means inverted hierarchy (default: "r").
  
\item RGEMlightest \ is the mass of the lightest neutrino in eV (default: $\mathcal{O}(0.01)
  \eV$).
  \ The default of RGEMlightest depends on the
model. It is chosen in such a way, that the parameters are compatible with the
experimental data.
\item RGEY$\nu$\ is the neutrino Yukawa matrix in ``RL convention''. This option overrides the
  built-in Yukawa matrix, i.e.\ RGEY$\nu33$ and RGEY$\nu$Ratio do not have any
  effect. 
    (default: RGEGetY$\nu$(RGEY$\nu33$, RGE$Y\nu$Ratio))
  
\item RGEY$\nu$33\ is the (3,3) entry in the neutrino Yukawa matrix at the GUT
  scale.\ The default value depends on the
  model and it is chosen in such a way, that it is compatible with the
  experimental data (default: $\mathcal{O}(1)$).
  
\item RGEY$\nu$Ratio\ determines the relative value of the neutrino Yukawa couplings.\ The default value depends on the
  model and it is chosen in such a way, that it is compatible with the
  experimental data (default: $\mathcal{O}(1)$).
\item RGEYd\ is the Yukawa matrix of the down-type quarks.
  If this parameter is given, RGEyd, RGEys, RGEyb, RGEq$\varphi$1,
  RGEq$\varphi$2, RGEq$\delta$, RGEq$\delta$e, RGEq$\delta\mu$,
  RGEq$\delta\tau$, RGEq$\theta$12, RGEq$\theta$13, and RGEq$\theta$23
  are ignored.
  
\item RGEYe\ is the charged lepton Yukawa matrix.
  If this parameter is given, RGEye, RGEy$\mu$ and RGEy$\tau$ are
  ignored.
    \\ (default:
  RGEGetYe(0.8*Mass$\left[\mathrm{"}\tau\mathrm{"}\right]$*Sqrt[2]/RGEvd))
  
\item RGEYu\ is the Yukawa matrix of the up-type quarks.
  If this parameter is given, RGEyu, RGEyc and RGEyt are ignored;
  it is recommended not to use RGEq$\varphi$1, RGEq$\varphi$2,
  RGEq$\delta$, RGEq$\delta$e, RGEq$\delta\mu$, RGEq$\delta\tau$,
  RGEq$\theta$12, RGEq$\theta$13, and RGEq$\theta$23 in this case, since
  they are not necessarily equal to the CKM mixing parameters.
\item RGE$\Delta$m2atm\ is the atmospheric mass squared difference (default: $ \mathcal{O}(10^{-3}) \eV^2$).\ The default value depends on the
  model and it is chosen in such a way, that it is compatible with the
  experimental data.
  
\item RGE$\Delta$m2sol\ is the solar mass squared difference (default:
  $\mathcal{O}(10^{-4}) \eV^2$).\ The default value depends on the
  model and it is chosen in such a way, that it is compatible with the
  experimental data.
\item RGE$\varphi$1\ and RGE$\varphi2$ are the Majorana CP phases $\varphi_1$ and $\varphi_2$ in radians (default: $0$).
  
\item RGE$\delta$\ is the Dirac CP phase $\delta$ in radians (default: $0$).
\item RGE$\delta$e, RGE$\delta\mu$ and RGE$\delta\tau$ are the unphysical phases $\delta_e$,
  $\delta_\mu$ and $\delta_\tau$ (default: $0$). 
\item RGE$\kappa$1\ is the coupling of the dimension 5 operator associated
  with the first Higgs in the 2HDM.
  
\item RGE$\kappa$2\ is the coupling of the dimension 5 operator associated
  with the second Higgs in the 2HDM.
  
\item RGE$\lambda$1, RGE$\lambda2$, RGE$\lambda3$, RGE$\lambda4$ and RGE$\lambda5$ are the parameters
$\lambda_1$, $\lambda_2$, $\lambda_3$, $\lambda_4$ and $\lambda_5$ in the Higgs potential
(default: $\lambda_1=\lambda_2=0.75$, $\lambda_3=\lambda_4=0.2$, $\lambda_5=0.25$).
\item RGE$\theta$12, RGE$\theta13$ and RGE$\theta23$ are the angles $\theta_{12}$, $\theta_{13}$
and $\theta_{23}$ of the MNS matrix in radians. (default: $\theta_{13}=0$ and
$\theta_{23}=\frac{\pi}{4}$). The default of $\theta_{12}$ depends on the
model. It is chosen in such a way, that the parameters are compatible with the
experimental data. 
\item RGEg RGEg is the coupling constants of SU(5)
  
\item RGEg1, RGEg2 and RGEg3 are the coupling constants of U$(1)_\mathrm{Y}$,
  SU$(2)_\mathrm{L}$ and SU$(3)_\mathrm{C}$, respectively.  GUT charge
  normalization is used for $g_1$.
  
\item RGEm RGEm is the Higgs mass
  
\item RGEq$\varphi$1\ and RGEq$\varphi2$ are the unphysical phases $\varphi_1$ and $\varphi_2$ of the
 CKM matrix which correspond to the Majorana phases in the MNS matrix (default: $0$).
\item RGEq$\delta$\ is the Dirac CP phase $\delta$ of the CKM matrix.
\item RGEq$\delta$e, RGEq$\delta\mu$ and RGE$\delta\tau$ are the unphysical phases $\delta_e$,
$\delta_\mu$ and $\delta_\tau$ of the CKM matrix (default: $0$).
\item RGEq$\theta$12, RGEq$\theta13$ and RGEq$\theta23$ are the angles of the CKM matrix. 
\item RGEyd, RGEys and RGEyb are the Yukawa coupling of the down-type quarks $d$,
  $s$ and $b$.
\item RGEye, RGEy$\mu$ and RGEy$\tau$ are the Yukawa couplings of the charged
  leptons $e$, $\mu$ and $\tau$.
\item RGEyu, RGEyc and RGEyt are the Yukawa couplings of the up-type quarks $u$,
  $c$ and $t$.

\end{itemize}

Parameters accepted by \function{RGEGetSolution}:
\begin{itemize}
\item 
RGECoupling is used to get the coupling constants.
\item 
RGEGWCondition returns the Gildener Weinberg condition.
\item 
RGEGWConditions returns all Gildener Weinberg conditions.
\item 
RGEM1Tilde returns the effective light-neutrino mass $\widetilde{m}_1 =
\frac{\left(m_D m_D^\dagger\right)_{11}}{M_1}=\frac{(Y_\nu Y_\nu^\dagger)_{11} 
v^2}{2 M_1}$ which is commonly used in thermal leptogenesis. $\widetilde{m}_1$ is
given in eV.
\item 
RGEM$\nu$ is used to get the mass matrix of the left-handed neutrinos.
\item 
RGEM$\nu$r is the mass matrix of the right-handed neutrinos.
\item 
RGEMd is used to get the mass matrix of the down-type quarks.
\item 
RGEMe is used to get the mass matrix of the charged leptons.
\item 
RGEMixingParameters returns the mixing parameters in the leptonic sector as they
are returned by MNSParameters: $\left\{\left\{\theta_{12},\theta_{13},\theta_{23},\delta,\delta_e,\delta_\mu,\delta_\tau,\varphi_1,\varphi_2\right\},\left\{y_1,y_2,y_3\right\},\left\{y_e,y_\mu,y_\tau\right\}\right\}$
\item 
RGEMu is used to get the mass matrix of the up-type quarks.
\item 
RGEPoleMTop is used to get the pole mass of the top quark in the
$\overline{\text{MS}}$ scheme. The pole mass term of the top quark is given by
\begin{equation}
m_t^\text{Pole} = m_t(m_t) \cdot (1 + \frac{4\alpha_s}{3\pi})
\end{equation}
to 1-loop order.
\item 
RGERawM$\nu$r is used to get the raw mass matrix of the right-handed neutrinos.
\item 
RGERaw is used to get the raw values of all parameters. A raw parameter is
the internal representation of the parameter
\item 
RGERawY$\Delta$ is used to get the Yukawa coupling matrix of the coupling to the Higgs triplet.
\item 
RGERawY$\nu$ is used to get the raw Yukawa coupling matrix of the
neutrinos.
\item 
RGEAll returns all parameters of the model.
\item 
RGEVEVratio returns the squared ratio of $v_R$ over the EW symmetry breaking scale.
\item 
RGEVEVratios returns the squared ratio of $v_R$ over the EW symmetry breaking scale.
\item 
RGEY$\nu$ is used to get the Yukawa coupling matrix of the neutrinos.
\item 
RGEYd is used to get the Yukawa coupling matrix of the down-type quarks.
\item 
 RGEYe is used to get the Yukawa coupling matrix of the charged leptons.
\item 
RGEYu is used to get the Yukawa coupling matrix of the up-type quarks.
\item 
RGE$\alpha$ is used to get the fine structure constants.
\item 
RGE$\kappa$1 is the parameter of the dimension 5 operator associated
  with the first Higgs in the 2HDM.
\item 
RGE$\kappa$2 is the parameter of the dimension 5 operator associated
  with the second Higgs in the 2HDM.
  
\item 
RGE$\lambda$ is used to get the Higgs couplings.

\end{itemize}

  

\subsubsection[\package{RGE2HDM0N}]{\package{REAP`RGE2HDM0N`}}
This package contains the Two Higgs Doublet Model (2HDM) with a $\mathbb{Z}_2$
symmetry without right-handed neutrinos.

\vspace{2ex} It has the same parameters and options as
\package{RGE2HDM}, with the following exceptions:  The
only missing options are RGEIntegratedOut, RGESearchTransition,
RGEThresholdFactor, RGEPrecision and RGEMaxNumberIterations, which are used to
control the process of integrating out.  In addition RGEM$\nu$r and RGEY$\nu$
are no parameters of \function{RGESetInitial} and RGEM1Tilde, RGERawM$\nu$r and RGERawY$\nu$
are not accepted as parameters by \function{RGEGetSolution}.
\function{RGESetInitial} has an additional option: RGESuggestion can be
used to choose between different sets of default values, ``GUT''
(default) and ``MZ''.  They refer to typical parameter values at the GUT
scale or at the $Z$ mass, respectively.


\subsubsection[\package{RGE2HDMDirac}]{\package{REAP`RGE2HDMDirac`}}
This package contains the 2HDM with Dirac neutrinos to 1 loop order.

\vspace{2ex} It has the same parameters and options as
\package{RGE2HDM}, with the following exceptions: The
only missing options are RGEIntegratedOut, RGESearchTransition,
RGEThresholdFactor, RGEPrecision and RGEMaxNumberIterations, which are used to
control the process of integrating out.  In addition RGEM$\nu$r, RGE$\kappa$1
and RGE$\kappa$2 are no parameter of \function{RGESetInitial} and
RGEMixingParameters, RGEM1Tilde, RGERawM$\nu$r,
RGERawY$\nu$, RGE$\kappa$1 and RGE$\kappa$2 are not accepted as parameters by
\function{RGEGetSolution}.
\function{RGESetInitial} has an additional option: RGESuggestion can be
used to choose between different sets of default values, ``GUT''
(default) and ``MZ''.  They refer to typical parameter values at the GUT
scale or at the $Z$ mass, respectively.


%\input{RGEModels}
\section{Conventions}

\subsection{Definition of the parameters}

\subsubsection{GUT Charge Normalization}
We use the GUT charge normalization in all models which is related to the charge normalization in the SM by
\begin{equation}
  q_Y^\text{GUT}=\sqrt{\frac{3}{5}}q^\text{SM}_Y\; .
\end{equation}
Therefore the coupling constant satisfies
\begin{equation}
  \left(g^\text{SM}_1\right)^2=\frac{3}{5}\left(g_1^\text{GUT}\right)^2\; .
\end{equation}

\subsubsection{Convention for Yukawa Matrices}
We use the RL convention for Yukawa matrices, i.e.\
\begin{equation}
  \mathscr{L}_\text{Yukawa}=-Y_{ij} \overline{\psi_\mathrm{R}^i}
  \psi_\mathrm{L}^j \cdot \phi + \text{h.c.} \;.
\end{equation}

\subsubsection{Vacuum Expectation Value of the SM Higgs}
The vacuum expectation value of the SM Higgs is $v=246\GeV$.
It is assumed to be constant.

\subsubsection{Standard Parameterization of the Lepton Mixing Matrix}

A unitary matrix can be described by 3 angles and 6 phases. Thus it can be
written in the following way:
\begin{equation}
 U 
 \,=\,
 \diag(e^{\I\delta_{e}},e^{\I\delta_{\mu}},e^{\I\delta_{\tau}}) \cdot V({ \theta_{12}},{ \theta_{13}},{\theta_{23}}) \cdot 
 \diag(e^{-\I\varphi_1/2},e^{-\I\varphi_2/2},1)
\end{equation}
V is a special unitary matrix and is parameterized in standard parameterization like the CKM
matrix in the quark sector with 3 angles ($\theta_{12},\,\theta_{13},\,\theta_{23}$) and 1 CP phase ($\delta$).
\begin{equation}
 V({ \theta_{12}},{ \theta_{13}},{\theta_{23}})=\left(
 \begin{array}{ccc}
 { c_{12}}{ c_{13}} & { s_{12}}{ c_{13}} & { s_{13}}
 \,e^{-\I{ \delta}}\\
 -{  c_{23}}{ s_{12}}-{  s_{23}}{ s_{13}}{ c_{12}}
 \,e^{\I{ \delta}} & {  c_{23}}{ c_{12}}-{  s_{23}}{ s_{13}}
 { s_{12}}\,e^{\I{ \delta}} & {  s_{23}}{ c_{13}}\\
 {  s_{23}}{ s_{12}}-{  c_{23}}{ s_{13}}{ c_{12}}
 \,e^{\I{ \delta}} & -{  s_{23}}{ c_{12}}-{  c_{23}}{ s_{13}}
 { s_{12}}\,e^{\I{ \delta}} & {  c_{23}}{ c_{13}}
 \end{array}
 \right)
\label{eq:DefV}
\end{equation}
where $s_{ij}$ and $c_{ij}$ are defined as$s_{ij}=\sin\theta_{ij}$ and
$c_{ij}=\cos\theta_{ij}$, respectively.
In addition there are phase matrices multiplied from both sides. The matrix on
the left-hand side is characterized by the unphysical phases
$\delta_e$, $\delta_\mu$ and $\delta_\tau$ which can be rotated away by a change of
the phases in the charged left-handed leptons in the extended (MS)SM.
\begin{equation}
\ket{\ell_L}\rightarrow
\diag(e^{-\I\delta_{e}},e^{-\I\delta_{\mu}},e^{-\I\delta_{\tau}}) \ket{\ell_L}
\end{equation}
The matrix on the right-hand side is described by the Majorana phases
$\varphi_1$ and $\varphi_2$. These can not be rotated away by a redefinition of
fields, because the effective neutrino mass term is a Majorana mass term which is
diagonalized by an unitary transformation and not by a biunitary transformation
like the Yukawa matrix of the charged leptons.

%%%%%%%%%%%%%%%%%%%%%%%%%%%%%%%%%%%%%%%%%%%%%%%%%%%%%%%%%%%%%%%%%%%%%%%%%%%%%%%%%%%
%% naming conventions
%%%%%%%%%%%%%%%%%%%%%%%%%%%%%%%%%%%%%%%%%%%%%%%%%%%%%%%%%%%%%%%%%%%%%%%%%%%%%%%%%%%

\subsection{Naming conventions}

\subsubsection{Variable names}
The first letter of each word and abbreviations like (``RGE'') in a variable
name are capitalized. The remaining letters are uncapitalized.  All local
variables have ``l'' as prefix and all parameters in functions have ``p'' as
prefix.

\subsubsection{Function names}
The first letter of each word and abbreviations like (``RGE'') in a function
name are capitalized. The remaining letters are uncapitalized.  All public
functions in \package{REAP.m} begin with ``RGE''.

\subsubsection{Exceptions}
The first letter of each word and abbreviations like (``RGE'') in the name of a
exception are capitalized. The remaining letters are uncapitalized.

The defined exceptions are:\\
\begin{itemize}
\item RGEModelAlreadyRegistered will be thrown if the model already is
registered.
\item RGELessThanZero will be thrown if a parameter is less than zero, thus out
of range.
\item RGEScaleTooBig will be thrown if the scale parameter is too big, thus out
of range.
\item RGENotImplementedYet will be thrown if the transition function isn't
implemented yet.
\item RGEOutOfRange will be thrown if the parameter is out of range.
\item RGEModelDoesNotExist will be thrown if the model name given does not exist
in the list Model. Thus the model hasn't been registered yet.
\item RGEWrongModel will be thrown if the model name does not corresponding to
the model valid at the given scale.
\item RGE$\nu$MassAboveCutoff will be thrown if an eigenvalue of M$\nu$ is above
the cutoff.
\item RGENotAValidMassHierarchy will be thrown if the type of the given mass
hierarchy does not exist.
\end{itemize}



\section{How to define a new model}

A model has to provide several functions.  In the following a simple example of
a toy model with one running coupling constant is shown.  It is contained in the
file \texttt{RGEToyModel.m}.


\begin{enumerate}
  \item First of all it must have a function returning the parameters with no arguments.
\begin{verbatim}
Parameters={\[Lambda]};

ClearAll[GetParameters];
GetParameters[]:= Module[{},
   Return[Parameters];
];
\end{verbatim}

\item Furthermore there has to be a function to solve the RGE.

\begin{verbatim}
ClearAll[RGE];
RGE:={  D[\[Lambda][t],t]==Beta\[Lambda][\[Lambda][t]] };

ClearAll[Beta\[Lambda]];
Beta\[Lambda][\[Lambda]_] :=-7 * 1/(16*Pi^2) * \[Lambda]^3;

ClearAll[SolveModel];
SolveModel[{pUp_,pUpModel_,pUpOptions_},{pDown_,pDownModel_,pDownOptions_},
           pDirection_,pBoundary_,pInitial_,pOpts___]
           :=Module[{lSolution,lNDSolveOpts,lNewScale,lInitial},
        lNDSolveOpt;
        Options[lNDSolveOpts]=Options[NDSolve];
        SetOptions[lNDSolveOpts,FilterOptions[NDSolve,Options[RGEOptions]]];
        SetOptions[lNDSolveOpts,FilterOptions[NDSolve,pOpts]];
        lInitial=SetInitial[pBoundary,pInitial];
        lSolution=NDSolve[RGE ~Join~ lInitial, Parameters,{t,pDown,pUp},
                Sequence[Options[lNDSolveOpts]]];
        If[lDirection>0,lNewScale=pUp,lNewScale=pDown];
        Return[{lSolution,lNewScale,0}];
];
\end{verbatim}

  The arguments of \function{SolveModel} are
  \begin{itemize}
    \item pUp is the upper bound. It is the logarithm of the renormalization scale $\log\mu$.
    \item pUpModel is the modelname of the model which is valid above pUp.
    \item pUpOptions are the options of the model which is valid above pUp.
      
    \item pDown, pDownModel and pDownOptions are the corresponding options for
    the lower bound.
  \item pDirection specifies whether the RGEs are solved upwards or downwards.
  \item pBoundary is the scale where the initial values are given. It is the
  logarithm of the renormalization scale scale $\log\mu$.
  \item pInitial is the list of initial values which given as replacement rules.
  \item pOpts are options for NDSolve,\dots .
  \end{itemize}

\item The transition functions provide the possibility to implement a model
which has several transitions to other EFT's (e.g. integrating out degrees of freedom like
heavy right handed neutrinos):
\begin{verbatim}
ClearAll[Transition];
Transition[pScale_?NumericQ,pDirection_?NumericQ,pSolution_,pToOpts_,pFromOpts_]
         :=Module[{},
       Return[({RGE\[Lambda]->\[Lambda][pScale]}/.pSolution)[[1]]];
];
\end{verbatim}

\item The model has to provide initial values. The only argument of the function
  are the specified initial values of the user. The initial values are returned as replacement rules.
\begin{verbatim}
ClearAll[GetInitial];
GetInitial[pOpts___:{}]:=Module[{lInitial,l\[Lambda]},
        lInitial={RGE\[Lambda]->0.1};
        l\[Lambda]=(RGE\[Lambda]/.pOpts)/.lInitial;
        Return[{RGE\[Lambda]->l\[Lambda]}];
];
\end{verbatim}

\item The model has to provide functions to set and return options.  The
function to set the options takes the options as parameters and the function
returning the options doesn't have any parameters.
\begin{verbatim}
ClearAll[ModelSetOptions];
ModelSetOptions[pOpts_]:= Module[{},
    SetOptions[RGEOptions,FilterOptions[RGEOptions,pOpts]];
];

ClearAll[ModelGetOptions];
ModelGetOptions[]:= Module[{},
   Return[Options[RGEOptions]];
];
\end{verbatim}

\item Finally the model has to provide functions to return the solution which
  take as argument the logarithmic energy scale $\log\mu$ and the solution for
  this energy range. \param{pOpts} might contain options which are relevant for
  the model.

\begin{verbatim}
ClearAll[GetSolution];
GetSolution[pScale_,pSolution_,pOpts___]:=Module[{},
        Return[({\[Lambda][pScale]}/.pSolution)[[1]]];
];
\end{verbatim}

\end{enumerate}

At last the model has to be registered.

The function \function{RGERegisterModel[\param{name}, \param{function returning
parameters}, \param{ solution}, \param{list of transition
functions}, \param{provide initial values}, \param{set options}, \param{get
options}]} of the package \package{REAP} is used to register a new
model (see \ref{func:RGERegisterModel}).

\begin{verbatim}
RGERegisterModel["Toy","REAP`Toy`",
        `Private`GetParameters,
        `Private`SolveModel,
        {RGEAll->`Private`GetSolution},
        {{"Toy",`Private`Transition}},
        `Private`GetInitial,
        `Private`ModelSetOptions,
        `Private`ModelGetOptions
];
\end{verbatim}


%%%%%%%%%%%%%%%%%%%%%%%%%%%%%%%%%%%%%%%%%%%%%%%%%%%%%%%%%%%%%%%%%%%%%%%%%%%%%%%%%%%%%%%%%%%%%%%%%%%%

\bibliography{reap}

\end{document}
