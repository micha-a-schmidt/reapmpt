\section{Installation}

\subsection{Automatic Installation under UNIX/Linux}

Execute REAP.installer and you are done.
\begin{verbatim}
sh REAP.installer
\end{verbatim}
After the execution the Mathematica packages are copied to
\verb+~/.Mathematica/Applications/REAP+ and the documentation and notebooks are in a
subdirectory of the working directory, which is called REAP.

In addition, you have to install the package MixingParameterTools. There is also
a script which installs both REAP and MPT at the same time:
REAP\_MPT.installer.


\subsection{Semi-Automatic Installation under UNIX/Linux}

Unpack the archive REAP.tar.gz.
\begin{verbatim}
tar -xvzf REAP.tar.gz
\end{verbatim}
Then go to the directory REAPInstall and execute the script install.sh.
\begin{verbatim}
cd REAPInstall
sh ./install.sh
\end{verbatim}
The script copies the Mathematica packages to
\verb+~/.Mathematica/Applications/REAP+.  
The documentation and some sample notebooks are placed in a new
subdirectory of the working directory called REAP.  Hence, the folder
REAPInstall can be deleted now.

In addition, you have to install the package MixingParameterTools.
REAP and MPT can be installed simultaneously by using the archive
REAP\_MPT.tar.gz.  The procedure is completely analogous to the one
described above.


\subsection{Installation by Hand}

In order to install the package(s) manually, unpack the archive
REAP.tar.gz first.  Under UNIX/Linux, type
\begin{verbatim}
tar -xvzf REAP.tar.gz
\end{verbatim}
On Windows systems, a program like WinZip can be used.
Then move the directory REAP from the folder
REAPInstall to the directory
where the Mathematica add-ons are located, e.g.\
\begin{verbatim}
mv REAPInstall/REAP ~/.Mathematica/Applications/
\end{verbatim}
under UNIX/Linux. 
Under Windows XP, the path to the add-on directory should be something
like \verb+Application Data\Mathematica\Applications+.
The documentation and some sample notebooks can be found in
REAPInstall/Doc/REAP/.

In addition, you have to install the package MixingParameterTools.
To install both REAP and MPT at the same time, you can use the archive
REAP\_MPT.tar.gz.  The procedure is the same as above (except that the 
installation directory is called REAP\_MPTInstall now), supplemented by
an analogous step for moving the MPT directory, e.g.
\begin{verbatim}
mv REAP_MPTInstall/MixingParameterTools ~/.Mathematica/Applications/
\end{verbatim}


\endinput
