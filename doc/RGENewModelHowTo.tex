
\section{How to define a new model}

A model has to provide several functions.  In the following a simple example of
a toy model with one running coupling constant is shown.  It is contained in the
file \texttt{RGEToyModel.m}.


\begin{enumerate}
  \item First of all it must have a function returning the parameters with no arguments.
\begin{verbatim}
Parameters={\[Lambda]};

ClearAll[GetParameters];
GetParameters[]:= Module[{},
   Return[Parameters];
];
\end{verbatim}

\item Furthermore there has to be a function to solve the RGE.

\begin{verbatim}
ClearAll[RGE];
RGE:={  D[\[Lambda][t],t]==Beta\[Lambda][\[Lambda][t]] };

ClearAll[Beta\[Lambda]];
Beta\[Lambda][\[Lambda]_] :=-7 * 1/(16*Pi^2) * \[Lambda]^3;

ClearAll[SolveModel];
SolveModel[{pUp_,pUpModel_,pUpOptions_},{pDown_,pDownModel_,pDownOptions_},
           pDirection_,pBoundary_,pInitial_,pOpts___]
           :=Module[{lSolution,lNDSolveOpts,lNewScale,lInitial},
        lNDSolveOpt;
        Options[lNDSolveOpts]=Options[NDSolve];
        SetOptions[lNDSolveOpts,FilterOptions[NDSolve,Options[RGEOptions]]];
        SetOptions[lNDSolveOpts,FilterOptions[NDSolve,pOpts]];
        lInitial=SetInitial[pBoundary,pInitial];
        lSolution=NDSolve[RGE ~Join~ lInitial, Parameters,{t,pDown,pUp},
                Sequence[Options[lNDSolveOpts]]];
        If[lDirection>0,lNewScale=pUp,lNewScale=pDown];
        Return[{lSolution,lNewScale,0}];
];
\end{verbatim}

  The arguments of \function{SolveModel} are
  \begin{itemize}
    \item pUp is the upper bound. It is the logarithm of the renormalization scale $\log\mu$.
    \item pUpModel is the modelname of the model which is valid above pUp.
    \item pUpOptions are the options of the model which is valid above pUp.
      
    \item pDown, pDownModel and pDownOptions are the corresponding options for
    the lower bound.
  \item pDirection specifies whether the RGEs are solved upwards or downwards.
  \item pBoundary is the scale where the initial values are given. It is the
  logarithm of the renormalization scale scale $\log\mu$.
  \item pInitial is the list of initial values which given as replacement rules.
  \item pOpts are options for NDSolve,\dots .
  \end{itemize}

\item The transition functions provide the possibility to implement a model
which has several transitions to other EFT's (e.g. integrating out degrees of freedom like
heavy right handed neutrinos):
\begin{verbatim}
ClearAll[Transition];
Transition[pScale_?NumericQ,pDirection_?NumericQ,pSolution_,pToOpts_,pFromOpts_]
         :=Module[{},
       Return[({RGE\[Lambda]->\[Lambda][pScale]}/.pSolution)[[1]]];
];
\end{verbatim}

\item The model has to provide initial values. The only argument of the function
  are the specified initial values of the user. The initial values are returned as replacement rules.
\begin{verbatim}
ClearAll[GetInitial];
GetInitial[pOpts___:{}]:=Module[{lInitial,l\[Lambda]},
        lInitial={RGE\[Lambda]->0.1};
        l\[Lambda]=(RGE\[Lambda]/.pOpts)/.lInitial;
        Return[{RGE\[Lambda]->l\[Lambda]}];
];
\end{verbatim}

\item The model has to provide functions to set and return options.  The
function to set the options takes the options as parameters and the function
returning the options doesn't have any parameters.
\begin{verbatim}
ClearAll[ModelSetOptions];
ModelSetOptions[pOpts_]:= Module[{},
    SetOptions[RGEOptions,FilterOptions[RGEOptions,pOpts]];
];

ClearAll[ModelGetOptions];
ModelGetOptions[]:= Module[{},
   Return[Options[RGEOptions]];
];
\end{verbatim}

\item Finally the model has to provide functions to return the solution which
  take as argument the logarithmic energy scale $\log\mu$ and the solution for
  this energy range. \param{pOpts} might contain options which are relevant for
  the model.

\begin{verbatim}
ClearAll[GetSolution];
GetSolution[pScale_,pSolution_,pOpts___]:=Module[{},
        Return[({\[Lambda][pScale]}/.pSolution)[[1]]];
];
\end{verbatim}

\end{enumerate}

At last the model has to be registered.

The function \function{RGERegisterModel[\param{name}, \param{function returning
parameters}, \param{ solution}, \param{list of transition
functions}, \param{provide initial values}, \param{set options}, \param{get
options}]} of the package \package{REAP} is used to register a new
model (see \ref{func:RGERegisterModel}).

\begin{verbatim}
RGERegisterModel["Toy","REAP`Toy`",
        `Private`GetParameters,
        `Private`SolveModel,
        {RGEAll->`Private`GetSolution},
        {{"Toy",`Private`Transition}},
        `Private`GetInitial,
        `Private`ModelSetOptions,
        `Private`ModelGetOptions
];
\end{verbatim}
