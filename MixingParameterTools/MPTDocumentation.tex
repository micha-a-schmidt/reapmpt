\NeedsTeXFormat{LaTeX2e}
\documentclass[12pt,a4paper,twoside]{scrartcl}
\usepackage{amsmath,amssymb,amsfonts,amscd}
\usepackage[amsthm,thmmarks]{ntheorem}
\usepackage{TheoremCollection}
\usepackage{accents}
\usepackage{bbm} % BlackBoeard letters
%\usepackage[bbgreekl]{MyBbol} % Other Doublestroke Charakters
\usepackage{fancyheadings} %
%\usepackage{a4wide} %
\usepackage[small]{caption2} %
\usepackage{makeidx} %
\usepackage{fleqn} 
\usepackage{indentfirst} 
\usepackage{cancel} % \cancel{stuff} provides slashed stuff
\usepackage{graphicx} % \including PostScript
\usepackage{color}
%\usepackage{braket} %\bra{stuff} -> <stuff|
\usepackage{pstricks}
\usepackage{pst-node,pst-plot}
\usepackage{nomencl} %Nomenklatur
\usepackage{mathrsfs}   % Schoenes Lagrange -L mit \mathscr{L}
\usepackage{pifont} % dinbgbats
\usepackage[small]{subfigure}
\usepackage{longtable}
\usepackage[plain]{fancyref}
\usepackage{cite}
%\usepackage[bbgreekl]{MyBbol}  % bb-Symbole auch griechisch
\usepackage[
        a4paper,      % A4
        dvips,        % Erzeugung durch dvips
        pdftitle={Proseminar Betriebssysteme - IO-Subsysteme}
	bookmarks=true,
	bookmarksnumbered=true, % Verwendete Bookmarks anzeigen
        colorlinks,   % Farbige Links
        linkcolor=blue,
        urlcolor=blue,
        citecolor=blue]{hyperref}
%\usepackage[dvips]{thumbpdf}

%-- page parameters -------------------------------------------------

\pagestyle{fancyplain}

\advance \headheight by 3.0truept       % for 12pt mandatory...
\lhead[\fancyplain{}{\thepage}]{\fancyplain{}{\rightmark}}
\rhead[\fancyplain{}{\leftmark}]{\fancyplain{\thepage}{\thepage}}
\cfoot{}

\addtolength{\oddsidemargin}{1.0truecm}
\addtolength{\evensidemargin}{-0.3truecm}

%-- end of page parameters ------------------------------------------


\makeindex
\makeglossary

% \newcommand{\WeightConnect}[4]{\ncline{->}{#1}{#2}\mput*{\ovalnode{#3}{#4}}
% \ncline{-}{#1}{#3}\ncline{->}{#3}{#2}}
% 
% \def\Nf{i}
% \def\Ng{j}
% \newcommand{\GroupIndex}[1]{\ifcase#1\or \or a\or b\or c\or d\or e\or f\or g\or h\or i\or j\or
% k\or l\or m\or n\or o\or p\or q\or r\or s\or t\or u\or v\or w\or x\or
% y\or z\else\@ctrerr\fi}
% \newcommand{\FamilyIndex}[1]{\ifcase#1\or \or f\or g\or h\or i\or j\or
% k\or l\or m\or n\or o\or p\or q\or r\or s\or t\or u\or v\or w\or x\or
% y\or z\else\@ctrerr\fi}
\def\chargec{\mathrm{C}}        
\def\ChargeC{\mathrm{C}}        
\def\NuMSSM{{$\nu$MSSM}\ }
\newcommand{\SimpleRoot}[1]{\alpha^{(#1)}}
\newcommand{\FundamentalWeight}[1]{\mu^{(#1)}}
\newcommand{\ChargeConjugate}[1]{#1^\chargec}
\newcommand{\SFConjugate}[1]{#1^\chargec}
\newcommand{\CenterFmg}[1]{\ensuremath{\vcenter{\hbox{\input{#1.fmg}}}}}
\newcommand{\CenterObject}[1]{\ensuremath{\vcenter{\hbox{#1}}}}
\newcommand{\CenterEps}[2][1]{\ensuremath{\vcenter{\hbox{\includegraphics[scale=#1]{#2.eps}}}}} % Input eps files - Usage: \CenterEps[ScaleFactor]{FileName}
\newcommand{\Commutator}[2]{{\left[ #1,#2\right]}_-}
\newcommand{\AntiCommutator}[2]{{\left\{ #1,#2\right\}}}
\newcommand{\RightHandedNeutrino}{\nu}
\newcommand{\SuperCommutator}[2]{\left\Lbracket #1,#2\right\Rbracket}
\newcommand{\SuperField}[1]{\bbsymbol{#1}}
\newcommand{\package}[1]{{\tt #1}}
\newcommand{\function}[1]{{\tt #1}}
\newcommand{\param}[1]{{\tt #1}}
\newcommand{\optparam}[1]{{\tt\em #1}}

\DeclareMathOperator{\re}{Re}
\DeclareMathOperator{\im}{Im}
\DeclareMathOperator{\tr}{tr}
\DeclareMathOperator{\Tr}{Tr}
\DeclareMathOperator{\diag}{diag}
\DeclareMathOperator{\quabla}{\boldsymbol{\square}}
\DeclareMathOperator{\STr}{STr}
\DeclareMathOperator{\Li}{Li}
\DeclareMathOperator{\ad}{ad}
\DeclareMathOperator{\Ad}{Ad}
\DeclareMathOperator{\AD}{AD}
\DeclareMathOperator{\ind}{ind}
\DeclareMathOperator{\ch}{ch}
\DeclareMathOperator{\Pf}{Pf}
\DeclareMathOperator{\arcosh}{arcosh}

\def\D{\mathrm{d}}
\def\I{\mathrm{i}}
\def\Nf{f}
\def\Ng{g}
\def\PlusIEpsilon{}
\newcommand{\ChargeConjMatrix}{\mathsf{C}}      % Charge conjugation matrix
\newcommand{\ChargeConjOp}{\boldsymbol{\ChargeConjMatrix}}% cc. operator
\def\NuMSSM{{$\nu$MSSM}\ }
\def\secname{section}
\newif\ifappendix
\newcommand{\secref}[1]{%
        \ifappendix\appendixname~\ref{#1}\else\secname~\ref{#1}\fi
        }
\appendixfalse
\renewcommand{\thesection}{\arabic{section}}
\renewcommand{\thesubsection}{\arabic{section}.\arabic{subsection}}
\renewcommand{\thesubsubsection}{(\roman{subsubsection})}
% 
% \renewcommand{\thesection}{\arabic{section}}
% \renewcommand{\thesection}{\arabic{section}.\arabic{section}}
% \renewcommand{\thesubsection}{(\roman{subsection})}

\newcommand*{\fancyrefsubseclabelprefix}{subsec}
\newcommand*{\subsecname}{subsection}
\fancyrefaddcaptions{english}{%
\newcommand*{\frefsubsecname}{subsection}
\newcommand*{\Frefsubsecname}{subsection}
}
\numberwithin{equation}{section}
\numberwithin{table}{section}
\renewcommand{\thetable}{\arabic{section}-\arabic{table}}
\renewcommand{\theequation}{\arabic{section}.\arabic{equation}}
\renewcommand{\labelenumi}{(\arabic{enumi})}


\definecolor{MyBlue}{rgb}{0.8,0.85,1}
\def\labelitemi{$\bullet$}
\def\labelitemii{--}
\unitlength=1mm

\allowdisplaybreaks[1]

\begin{document}
\font\TitleFont=cmbx10 at 40pt
\font\SubTitleFont=cmbx10 at 25pt
\pagenumbering{arabic}
\title{\TitleFont{Mixing parameter tools}\\[1cm]
\SubTitleFont{Internal documentation}
        }
\author{S.~Antusch, J.~Kersten, M.~Lindner, M.~Ratz and M.~Schmidt}
\maketitle
\begin{abstract}
 This is an internal documentation of the \package{MixingParameterTools} add-on
 for Mathematica. We describe the functions which allow to extract the mixing
 parameters from mass and Yukawa matrices in some detail. There is also some
 information concerning the installation.
\end{abstract}
\thispagestyle{empty}
\tableofcontents
\clearpage


\section{Short description and comments}

This is the documentation of the \package{MixingParametersTools} add-on. It
contains the \package{MPT3x3.m} package which provides various tools allowing
for the extraction of physical parameters from mass and Yukawa matrices.

Note that we do not adopt the naming conventions of the related
\package{SolveNeutrinoRGEs} add-on since the present add-on is intended to be a
`stand-alone' application, i.e.\ one may use it without loading the full set of
\package{RGE\dots} packages. This is because it might be useful in order to
study textures without running, and it is not bound to be applied to the
analysis of neutrino masses only but may be used for quark and superpartner
mass matrices as well.

The \package{MixingParameterTools} add-on is meant to replace the
\package{ExtractMixingAngles.m} package. It provides tools to extract mixing
parameters relating $3\times3$ mass matrices. It offers the treatment of both
Dirac and Majorana neutrino masses. In addition, functions evaluating quark
mixing parameters are also implemented.

\section{Installation}

\subsection{UNIX/Linux}

\subsubsection{Automatic installation}

Unpack the archive MixingParameterTools.tar.bz2.
\begin{verbatim}
tar -xvjf MixingParameterTools.tar.bz2
\end{verbatim}
Then go to the directory MixingParameterTools and execute the script install.sh in
this directory. The script copies the Mathematica package to the .Mathematica
folder in your home directory.
\begin{verbatim}
cd MixingParameterTools
./install.sh
\end{verbatim}

\subsubsection{Installation by hand}

In order to install the package(s), one has to switch to the directory where the
add-ons are, e.g.\
\begin{verbatim}
 cd .Mathematica/Applications/
\end{verbatim}
and create a directory for the mixing parameter tools:
\begin{verbatim}
 mkdir MixingParameterTools/
\end{verbatim}
Then one has to move the \package{MPT3x3.m} package to the new directory:
\begin{verbatim}
 cd
 mv MPT3x3.m .Mathematica/Applications/MixingParameterTools/
\end{verbatim}


\section{Functions}

We divide the functions into two classes:
\begin{itemize}
 \item `public' functions which will be explicitly mentioned in the publication, 
  and
 \item `private' functions which are useful, but have not necessarily to be
  invoked explicitly in order to extract the running mixing parameters.
\end{itemize}
Note that the terminology `public' or `private' is in quotation because it does
not correspond to what is public or private in the context of Mathematica
packages. The aim is to present only the absolutely necessary functions in the
publication so that the effort for maintenance is as low as possible. These are
the `public' functions. The `private' functions are those which will be useful
in several applications for us, but where there is no need to tell the rest of
the world that they exist (people familiar with Mathematica will notice their
existence anyway).

\subsection*{\function{MNSMatrix} (`public')}
\addcontentsline{toc}{subsection}{\function{MNSMatrix}}

\function{MNSMatrix[$m,Y_e$]} returns the MNS matrix, i.e.\ the matrix
$U_\mathrm{MNS}$ which diagonalizes the (neutrino mass) matrix $m$ in the basis
where the (charged lepton Yukawa coupling) matrix $Y_e$ is diagonal. By
convention, the parameters of $U_\mathrm{MNS}$ fulfill
$0\le\theta_{12}\le\pi/4$, $0\le\theta_{13},\theta_{23}\le \pi/2$ and all 
other parameters range from 0 to $2\pi$. It is possible to fix the hierarchy to
be inverted by calling \function{MNSMatrix[$m,Y_e$,``i'']}. Note that the input
matrices $m$ and $Y_e$ must be numeric.

\subsection*{\function{MNSParameters} (`public')}
\addcontentsline{toc}{subsection}{\function{MNSParameters}}

\function{MNSParameters[$m,Y_e$]} returns the MNS mixing and mass parameters
$\{\{\theta_{12},\theta_{13},\theta_{23},\delta,\delta_e,\delta_\mu,\delta_\tau,
\varphi_1,\varphi_2\},\{m_1,m_2,m_3\},\{(y_e,y_\mu,y_\tau\}\}$
for a Majorana neutrino matrix $m$ and a Yukawa coupling matrix $Y_e$. The
returned parameters obey the conventions $0\le\theta_{12}\le\pi/4$,
$0\le\theta_{13},\theta_{23}\le \pi/2$ and all other parameters range from 0 to
$2\pi$. It is possible to fix an inverted hierarchy to be inverted by calling
\function{MNSParameters[$m,Y_e$,``i'']}. Note that the input matrices $m$ and
$Y_e$ must be numeric.


\subsection*{\function{DiracMNSMatrix} (`public')}
\addcontentsline{toc}{subsection}{\function{DiracMNSMatrix}}

\function{DiracMNSMatrix[$Y_\nu,Y_e$]} returns the MNS matrix for Dirac
neutrinos with Yukawa coupling $Y_\nu$. It is the inverse of 
\function{CKMMatrix[$Y_\nu,Y_e$]} (see below).

\subsection*{\function{DiracMNSParameters} (`public')}
\addcontentsline{toc}{subsection}{\function{DiracMNSParameters}}
 
\function{DiracMNSParameters[$Y_\nu,Y_e$]} returns the MNS mixing parameters 
$\{\theta_{12},\theta_{13},\theta_{23},\delta\}$, $\{y_1,y_2,y_3\}$ (with $y_i=m_i/v$)
and $\{y_e,y_\mu,y_\tau\}$ for neutrino and charged lepton Yukawa
matrices $Y_\nu$ and $Y_e$. Note that these parameters are not sufficient to
determine the unitary matrix which diagonalizes $Y_\nu^\dagger Y_\nu$ in the
basis where  $Y_e^\dagger Y_e$ is diagonal. The additional parameters, required
to reconstruct $U_\mathrm{MNS}^\mathrm{Dirac}$, are unphysical. 

\subsection*{\function{CKMMatrix} (`public')}
\addcontentsline{toc}{subsection}{\function{CKMMatrix}}

\function{CKMMatrix[$Y_u,Y_d$]} returns the MNS matrix, i.e.\ the matrix
$U_\mathrm{CKM}$ which diagonalizes the (down-type quark Yukawa) matrix $Y_d$ 
in the basis where the (up-type quark Yukawa) matrix $Y_u$ is diagonal. 
Note that the input matrices $Y_u$ and $Y_d$ must be numeric. Note also that
this function can be used to extract the neutrino mixing parameters in the case
of Dirac neutrinos (see above).


\subsection*{\function{CKMParameters} (`public')}
\addcontentsline{toc}{subsection}{\function{CKMParameters}}

\function{CKMParameters[$Y_u,Y_d$]} returns the CKM mixing parameters 
$\{\theta_{12},\theta_{13},\theta_{23},\delta\}$, as well as the Yukawa 
couplings $\{y_u,y_c,y_t\}$ and $\{y_d,y_s,y_b\}$,  for up- and down-type
Yukawa matrices $Y_u$ and $Y_d$. Note that these parameters are not sufficient
to determine the unitary matrix which diagonalizes $Y_d^\dagger Y_d$ in the
basis  where $Y_u^\dagger Y_u$ is diagonal. The additional parameters, required
to reconstruct $U_\mathrm{CKM}$, are unphysical.  

\subsection*{\function{CKMReplacementRules} (`private')}
\addcontentsline{toc}{subsection}{\function{CKMReplacementRules}}

\function{CKMReplacementRules[$Y_u,Y_d$]} returns a list of replacement rules
which replaces the quark parameter by the numerical value calculated from $Y_u$
and $Y_d$.

\subsection*{\function{MPT3x3OrthogonalMatrix} (`private')}
\addcontentsline{toc}{subsection}{\function{MPT3x3OrthogonalMatrix}}

\function{MPT3x3OrthogonalMatrix[$\theta_{12},\theta_{13},\theta_{23},\delta$]}
returns the standard parametrized matrix
\begin{equation}\label{eq:StandardOrthogonalMatrix}
 V=\left(
 \begin{array}{ccc}
 c_{12}c_{13} & s_{12}c_{13} & s_{13}e^{-\I\delta}\\
 -c_{23}s_{12}-s_{23}s_{13}c_{12}e^{\I\delta} &
 c_{23}c_{12}-s_{23}s_{13}s_{12}e^{\I\delta} & s_{23}c_{13}\\
 s_{23}s_{12}-c_{23}s_{13}c_{12}e^{\I\delta} &
 -s_{23}c_{12}-c_{23}s_{13}s_{12}e^{\I\delta} & c_{23}c_{13}
 \end{array}
 \right)\;,
\end{equation}
with $c_{ij}$ and $s_{ij}$ defined as $\cos\theta_{ij}$ and
$\sin\theta_{ij}$, respectively. 
If $\delta$ is omitted, $V$ with $\delta=0$, i.e.\ a really orthogonal matrix, 
is returned.

\subsection*{\function{MPT3x3UnitaryMatrix} (`private')}
\addcontentsline{toc}{subsection}{\function{MPT3x3UnitaryMatrix}}

\function{MPT3x3UnitaryMatrix[
$\theta_{12},\theta_{13},\theta_{23},\delta,\delta_e,\delta_\mu,\delta_\tau,
\varphi_1,\varphi_2$]} returns the unitary matrix in
standard parametrization
\begin{eqnarray}\label{eq:StandardUnitaryMatrix}
 U & = &\diag(e^{\I\delta_{e}},e^{\I\delta_{\mu}},e^{\I\delta_{\tau}}) \cdot V \cdot 
 \diag(e^{-\I\varphi_1/2},e^{-\I\varphi_2/2},1)
\end{eqnarray}
with $V$ being defined by Eq.~\eqref{eq:StandardOrthogonalMatrix}.
Instead of 9 arguments one can invoke \function{StandardUnitaryMatrix} with a
list as argument. The entries of the list are then interpreted as mixing
parameters $\{\theta_{12},\theta_{13},\theta_{23},\delta,\delta_e,\delta_\mu,\delta_\tau,
\varphi_1,\varphi_2\}$, and if the length of the list is less than 9, the 
missing parameters are interpreted as 0. If the list is longer than 9, the
over-abundant entries are ignored.

\subsection*{\function{MPT3x3MixingMatrixL} (`private')} 
\addcontentsline{toc}{subsection}{\function{MPT3x3MixingMatrixL}}

\function{MPT3x3MixingMatrixL[$M$,\optparam{options}]} returns the matrix $U_\mathrm{L}$
which is used for diagonalizing a general complex matrix $M$, i.e. 
\begin{equation}\label{eq:BiUnitaryDiagonalization}
 U_\mathrm{R}^\dagger\,M\,U_\mathrm{L}\,=\,\diag(M_1,M_2,M_3)\;,
\end{equation}
with $M_1 \le M_2 \le M_3$. $U_\mathrm{L}$ is one representative of the 
matrices being suitable for fulfilling \eqref{eq:BiUnitaryDiagonalization}.
If the `eigenvalues' $M_i$ are non-degenerate, different representatives are
related by multiplication with unphysical phases. 
Note that in the case of degenerate $M_i$ mixing parameters which are otherwise
physical become unphysical. The label `L' refers to the fact that this matrix
rotates left-handed fields rather than the position in the diagonalization
formula \eqref{eq:BiUnitaryDiagonalization}.
The option is \optparam{MPTTolerance} which has as default value $10^{-6}$. It
controls the degree by which two numbers can disagree and still be considered
equal. Depending on the agreement, the diagonalization process is interpreted
to be successful or not.

\subsection*{\function{MPT3x3MixingMatrixR} (`private')} 
\addcontentsline{toc}{subsection}{\function{MPT3x3MixingMatrixR}}

\function{MPT3x3MixingMatrixR[$M$,\optparam{options}]} returns the corresponding matrix 
$U_\mathrm{R}$ (see above). The label `R' refers to the fact that this matrix
rotates right-handed fields rather than the position in the diagonalization
formula \eqref{eq:BiUnitaryDiagonalization}.
The option is \optparam{MPTTolerance} which has as default value $10^{-6}$. It
controls the degree by which two numbers can disagree and still be considered
equal. Depending on the agreement, the diagonalization process is interpreted
to be successful or not.

\subsection*{\function{MPT3x3NeutrinoMixingMatrix} (`private')}
\addcontentsline{toc}{subsection}{\function{MPT3x3NeutrinoMixingMatrix}}

\function{MPT3x3NeutrinoMixingMatrix[$m$,\optparam{MPTTolerance}]} 
returns the matrix $U$ which is used for diagonalizing a general 
complex symmetric Matrix $m$, i.e. 
\begin{equation}
 U^T\,m\,U
 \,=\,\diag(m_1,m_2,m_3) \;,
\end{equation}
and $|\Delta m_{32}^2|\ge|\Delta m_{21}^2|$ (with the usual definitions of
$\Delta m^2_{ij}$) and $U_{12}\le U_{11}$.\footnote{
The last relation is convention. It is implemented in order to read off
$\theta_{12}\le\pi/4$ finally.}\\ 
\function{MPT3x3NeutrinoMixingMatrix[$m,S$,\optparam{MPTTolerance}]} 
does the same only that the mass hierarchy can be fixed to be inverted by 
setting $S=\text{``i''}$. 
Any other $S$ leads to a \textbf{fixed} regular mass hierarchy.
The option is \optparam{MPTTolerance} which has as default value $10^{-6}$. It
controls the degree by which two numbers can disagree and still be considered
equal. Depending on the agreement, the diagonalization process is interpreted
to be successful or not.


\subsection*{\function{MPT3x3MixingParameters} (`private')}
\addcontentsline{toc}{subsection}{\function{MPT3x3MixingParameters}}

\function{MPT3x3MixingParameters[$U$]} inverts \eqref{eq:StandardUnitaryMatrix}
for a given  unitary $U$. More specifically,
\function{MPT3x3MixingParameters[$U$]} returns the mixing angles and phases 
$\theta_{12}$, $\theta_{13}$, $\theta_{23}$, $\delta$, $\delta_e$,
$\delta_\mu$, $\delta_\tau$, $\varphi_1$ and $\varphi_2$ of $U$ in standard
parametrization  where $0\le \theta_{ij}\le \pi/2$ holds.\\ 
\textbf{Note}: In
order to reproduce an arbitrary unitary matrix, one must allow the `Majorana
phases' $\varphi_i$ to range from $0$ to $4\pi$ (rather than $2\pi$).  However,
physically Majorana phases differing by $2\pi$ are ambiguous.  That is, if a
unitary matrix with a certain value of $\varphi_i$ diagonalizes a symmetric
mass matrix, the unitary matrix with  $\varphi_i+2\pi$ also does the job. As
the function \function{MPT3x3MixingParameters} is intended to invert 
\function{MPT3x3UnitaryMatrix}, it returns the mathematical rather than the
physical parameters.

\fbox{\parbox{0.9\textwidth}{
In some special cases, the mixing parameters are not defined uniquely.
In this case, a warning is printed and one set of possible parameters
is returned.  As a general rule, as many of the physical parameters as
possible are set to zero.  More details can be found in
sec.~\ref{sec:SpecialCases}.
}}

% In this standard-parametrization, the mixing angles $\theta_{13}$ 
% and $\theta_{23}$ can be chosen to lie between $0$ and $\frac{\pi}{2}$,
% and by reordering the masses, $\theta_{12}$ can be restricted to $0\le\theta_{12}\le\frac{\pi}{4}$.
% For the phases the range between $0$ and $2\pi$ is required.
% In order to read off the mixing parameters, we use the following procedure:
% \begin{enumerate}
%  \item $\theta_{13}=\arcsin(|U_{13}|)$.
%  \item $\displaystyle \theta_{12}=\left\{\begin{array}{ll}
%  \displaystyle \arctan\left(\frac{|U_{12}|}{|U_{11}|}\right) \quad
%         & \text{if}\;U_{11}\ne0\\
%  \frac{\pi}{2} & \text{else}
%  \end{array}\right.$
%  \item $\displaystyle \theta_{23}=\left\{\begin{array}{ll}
%  \displaystyle \arctan\left(\frac{|U_{23}|}{|U_{33}|}\right) \quad
%         & \text{if}\;U_{33}\ne0\\
%  \frac{\pi}{2} & \text{else}
%  \end{array}\right.$
%  \item $\delta_\mu = \arg(U_{23})$
%  \item $\delta_\tau = \arg(U_{33})$
%  \item \label{step6}$\displaystyle\delta=
%  -\arg\left(\frac{\displaystyle\frac{U_{ii}^*U_{ij}U_{ji}U_{jj}^*}
%         {c_{12}\,c_{13}^2\,c_{23}\,s_{13}}
%         +c_{12}\,c_{23}\,s_{13}}
%         {s_{12}\,s_{23}}\right)$\\
%  where $i,j\in\{1,2,3\}$ and $i\ne j$.
%  \item $\delta_e=\arg(e^{\I\delta}\,U_{13})$
%  \item $\displaystyle\varphi_1=2\arg(e^{\I\delta_e}\,U_{11}^*)$
%  \item \label{step9}$\displaystyle\varphi_2=2\arg(e^{\I\delta_e}\,U_{12}^*)$
% \end{enumerate}
% Here we used the relation
% \begin{eqnarray}
%  U_{ii}^*U_{ij}U_{ji}U_{jj}^*
%  & = &
%  c_{12}\,c_{13}^2\, 
%  c_{23}\,s_{13}
%  \left(e^{-\I\delta}\,s_{12}\,s_{23} - c_{12}\,c_{23}\,s_{13}\right)
%  \;,\nonumber
% \end{eqnarray}
% which holds for $i,j\in\{1,2,3\}$ and $i\ne j$.
% Note that this relation is often used in order to introduce 
% the Jarlskog invariants \cite{Jarlskog:1985ht}
% \begin{eqnarray}
%  J_\mathrm{CP} 
%  & = &
%  \frac{1}{2} \left| \im (U_{11}^*U_{12}U_{21}U_{22}^*)\right|
%  \,=\, 
%  \frac{1}{2} \left| \im (U_{11}^*U_{13}U_{31}U_{33}^*)\right|
%  \nonumber\\
%  & =  &
%  \frac{1}{2} \left| \im (U_{22}^*U_{23}U_{32}U_{33}^*)\right|
%  \, = \,
%  \frac{1}{2}\left|c_{12}\,c_{13}^2\,
%     c_{23}\,\sin \delta \,
%     s_{12}\,s_{13}\,
%     s_{23}\right|\;.
% \end{eqnarray}
% For the sake of a better numerical stability, one can choose any of the three
% combinations. In particular, if the modulus of one of the $U_{ij}$ is very
% small, it turns out to be more accurate to choose a combination in which this
% specific $U_{ij}$ does not appear.

% \subsection*{\function{MNSParameters}
% 
% \function{MNSParameters[$m_\nu,Y_e$]} returns the MNS mixing parameters
% $\theta_{12}$, $\theta_{13}$, $\theta_{23}$, $\delta$, $\varphi_1$ and
% $\varphi_2$ in physical convention, i.e.\ $0\le\theta_{12}\le\pi/2$,
% $0\le\delta\le 2\pi$ and $0\le\varphi_i\le 2\pi$. The input parameters are the
% neutrino mass matrix $m_\nu$ and the charged lepton Yukawa matrix $Y_e$.
% 
% \dots\textbf{not yet implemented}\dots
% 
% \subsection*{\function{CKMParameters}
% 
% \function{CMKParameters[$Y_u,Y_d$]} returns the MNS mixing parameters
% $\theta_{12}$, $\theta_{13}$, $\theta_{23}$ and $\delta$ in physical
% convention, i.e.\ $0\le\theta_{12}\le\pi/2$ and  $0\le\delta\le 2\pi$. The
% input parameters are the up-type and down-type Yukawa matrices, $Y_u$ and
% $Y_d$, respectively. Note that this function is also applicable to determine
% the mass and mixing parameters if neutrinos are Dirac.
% 
% \dots\textbf{not yet implemented}\dots

\section{Remarks}

\subsection{Remarks on the calculation of the CKM matrix}

The input parameters are the Yukawa couplings $Y=(Y_{fg})$ ($Y_u$ and $Y_d$) 
which are defined via the Lagrangean
\begin{equation}
 \mathscr{L}_\mathrm{Yukawa}
 \,=\,
 \overline{\psi_\mathrm{R}^f}\,Y_{fg}\,\psi_\mathrm{L}^g
 +\text{h.c.}\;,
\end{equation}
with R and L indicating right- and left-chiral fields, respectively.  $Y$ can
always be diagonalized by a bi-unitary transformation
\begin{subequations}
\begin{eqnarray}
 \psi_\mathrm{R} & \to & U_\mathrm{R}^\dagger\,\psi_\mathrm{R}\;,\\
 \psi_\mathrm{L} & \to & U_\mathrm{L}^\dagger\,\psi_\mathrm{L}\;,\\
 Y & \to & 
 U_\mathrm{R}^\dagger\,Y\,U_\mathrm{L}
 \,=\,
 \diag(y_1,y_2,y_3)\;,
 \end{eqnarray}
\end{subequations}
with $y_1\le y_2\le y_3$ being the `eigenvalues' of $Y$. Here, $U_\mathrm{L}$
and $U_\mathrm{R}$ are defined (or: can be computed) via
\begin{subequations}
\begin{eqnarray}
 U_\mathrm{L}^\dagger\,Y^\dagger\,Y\,U_\mathrm{L}
 & \stackrel{!}{=} & 
 \diag \left(|y_1|^2,|y_2|^2,|y_3|^2\right)\;,\\
 U_\mathrm{R}^\dagger\,Y\,Y^\dagger\,U_\mathrm{R}
 & \stackrel{!}{=} & 
 \diag \left(|y_1|^2,|y_2|^2,|y_3|^2\right)\;,
\end{eqnarray}
\end{subequations}
respectively. For most applications, $U_\mathrm{R}$ is irrelevant. 

The CKM matrix is calculated as follows:
\begin{enumerate}
 \item Switch to the basis where $Y_u$ is diagonal, i.e.
 \begin{subequations}
 \begin{eqnarray}
  Y_u & \to & (U_\mathrm{R}^{(u)})^\dagger\,Y_u\,U_\mathrm{L}^{(u)}
  \,=\,\diag(y_u,y_c,y_t)\;,\\
  Y_d & \to & (U_\mathrm{R}^{(u)})^\dagger\,Y_d\,U_\mathrm{L}^{(u)}
  \,=:\, Y_d'\;.
 \end{eqnarray}
 \end{subequations}
 \item Calculate $U_\mathrm{L}$ for $Y_d'$. This is $U_\mathrm{CKM}$.
\end{enumerate}

\subsection{Remarks on the calculation of the MNS matrix}

For the MNS matrix, switch to the basis where $Y_e$ is diagonal,
\begin{subequations}
\begin{eqnarray}
 Y_e & \to & U_\mathrm{R}^\dagger\,Y_e\,U_\mathrm{L} 
 \,=\,\diag(y_e,y_\mu,y_\tau)\;,\\
 m_\nu & \to & U_\mathrm{L}^T\,m_\nu\,U_\mathrm{L}
 \,=:\,m_\nu'\;.
\end{eqnarray}
\end{subequations} 
The MNS matrix has to fulfill
\begin{equation}
 U_\mathrm{MNS}^T\,m_\nu'\,U_\mathrm{MNS}
 \,=\,
 \diag(m_1,m_2,m_3)\;,
\end{equation}
where the $m_i$ are real and positive. However, this does not fix
$U_\mathrm{MNS}$ entirely. First of all, there is the obvious ambiguity of
ordering the mass eigenvalues $m_i$. In order to obtain a mixing matrix which
can be compared with the experimental data,  the choice of the prescription is
somewhat subtle. From experiment we know that there is a small mass difference,
called $\Delta m^2_\mathrm{sol}=m_i^2-m_j^2$, and a larger one, referred to as 
$\Delta m^2_\mathrm{atm}=m_k^2 - m_\ell^2$. By convention, the masses are
labeled such that $i,j\ne 3$ while either $k$ or $\ell$ equals 3.  The mass
label 2 is attached to the eigenvector with the lower modulus of the first
component. We are doing this since we want to read off a mixing angle
$\theta_{12}$ less than $45^\circ$. If it then turns out that $m_1>m_2$, the
corresponding mass matrix is most likely not physical.


\section{Definition and Extraction of Mixing Parameters}
\label{app:MixingParameters}

\subsection{Standard Parametrization}
In this section we describe our conventions and how mixing angles and 
phases can be extracted from mass matrices.
For a general unitary matrix we choose the so-called 
standard parametrization
\begin{eqnarray}\label{eq:StandardParametrizationU}
 U & = &\diag(e^{\I\delta_{e}},e^{\I\delta_{\mu}},e^{\I\delta_{\tau}}) \cdot V \cdot 
 \diag(e^{-\I\varphi_1/2},e^{-\I\varphi_2/2},1)
 \,=:\,
 K_\delta\cdot V\cdot K_\varphi\;,
\end{eqnarray}
where 
\begin{equation}
 V=\left(
 \begin{array}{ccc}
 c_{12}c_{13} & s_{12}c_{13} & s_{13}e^{-\I\delta}\\
 -c_{23}s_{12}-s_{23}s_{13}c_{12}e^{\I\delta} &
 c_{23}c_{12}-s_{23}s_{13}s_{12}e^{\I\delta} & s_{23}c_{13}\\
 s_{23}s_{12}-c_{23}s_{13}c_{12}e^{\I\delta} &
 -s_{23}c_{12}-c_{23}s_{13}s_{12}e^{\I\delta} & c_{23}c_{13}
 \end{array}
 \right)
\end{equation}
with $c_{ij}$ and $s_{ij}$ defined as $\cos\theta_{ij}$ and
$\sin\theta_{ij}$, respectively. 

\subsection{Physical vs.~unphysical parameters}

In the literature, the following parameters in standard parametrization
(cf.\ Eq.~\eqref{eq:StandardParametrizationU}) are called `physical':
\begin{subequations}
\begin{eqnarray}
 \{\theta_{12},\theta_{13},\theta_{23},\delta\}
 & : & \text{for the quark sector,}\\
 \{\theta_{12},\theta_{13},\theta_{23},\delta,\varphi_1,\varphi_2\}
 & : & \text{for the lepton sector.}
\end{eqnarray}
\end{subequations}
The parameters $\{\theta_{12},\theta_{13},\theta_{23},\delta\}$ for the quarks
are measurable in weak processes, i.e.\ in processes where $W^\pm$ is involved.
The parameters $\{\theta_{12},\theta_{13},\theta_{23},\delta\}$ for the leptons
are accessible in neutrino oscillations, in addition a combination of
$\varphi_1$ and $\varphi_2$ can, in principle,  be determined by neutrinoless
double $\beta$ decay experiments if neutrino masses are Majorana.

Let us work in the basis where $Y_e^\dagger Y_e$ is diagonal. Then
\begin{equation}
 m_\nu
 \,=\,
 U_\mathrm{MNS}^*\,\diag(m_1,m_2,m_3)\,U_\mathrm{MNS}^\dagger\;,
\end{equation}
i.e.\ the 12 parameters of $m_\nu$ can be decomposed into 3 eigenvalues, 6
physical rotations and 3 unphysical rotations according to the classification
of the literature. In other words, mass matrices $m_\nu$ and $m_\nu'$ are
equivalent if there exists a matrix
$K_\delta=\diag(e^{\I\delta_e},e^{\I\delta_\mu},e^{\I\delta_\tau})$ such that
\begin{equation}
 m_\nu'
 \,=\,K_\delta^*\,m_\nu\,K_\delta^\dagger 
 \,=\,K_\delta^{-1}\,m_\nu\,K_\delta^{-1}\;.
\end{equation}
For instance, we can always choose $m_\nu$ such that the $\delta_f$ phases
vanish.

Above the see-saw scales, we have to deal in addition with $Y_\nu^\dagger Y_\nu$
which can be written as
\begin{equation}
 Y_\nu^\dagger Y_\nu\,=\,
 U_\nu^\dagger\,\diag(y_1^2,y_2^2,y_3^2)\,U_\nu\;,
\end{equation}
i.e.\ it can be parametrized by three eigenvalues and 6 rotation parameters (the
parameters $\varphi_i$ of the standard parametrization and an overall phase are
obviously redundant). The crucial observation is now that a transformation
between equivalent effective mass matrices,
\begin{equation}\label{eq:DeltaTrafoMnu}
 m_\nu\,\to\,
 K_\delta\,m_\nu\,K_\delta\;,
\end{equation}
which can be interpreted as a rotation of the lepton doublets according to
\begin{equation}\label{eq:DeltaTrafoEll}
 \vec \ell\,\to\, K_\delta^{-1}\,\vec \ell\;,
\end{equation}
(obviously) implies a transformation of $Y_\nu^\dagger Y_\nu$, 
\begin{equation}\label{eq:DeltaTrafoYnu}
 Y_\nu^\dagger Y_\nu\,\to\,
 K_\delta^{-1}\,Y_\nu^\dagger Y_\nu\,K_\delta\;.
\end{equation}
The set of transformations \eqref{eq:DeltaTrafoMnu}, \eqref{eq:DeltaTrafoEll}
and \eqref{eq:DeltaTrafoYnu} corresponds to a symmetry of the theory, and leaves
in particular the RGEs invariant. 

The transformation \eqref{eq:DeltaTrafoMnu} (or \eqref{eq:DeltaTrafoYnu})
alone, however, is \textbf{not} a symmetry. That is to say that changing the
unphysical phases without changing the phases of the off-diagonal elements of
$Y_\nu^\dagger Y_\nu$ (or vice versa) leads to a different model,\footnote{It
would be interesting to check if there is a transformation in $M$ which
corresponds to this change of the effective neutrino mass operatore alone.}
i.e.\ different measurable parameters (at low energies). In order to define a
model, we hence need to specify both the $\delta$ phases and the off-diagonal
elements of $Y_\nu^\dagger Y_\nu$ at the same time. In other words, due to the
running, two models (in the `full' theory) with the same parameters except for
the $\delta_f$ phases do not describe the same physics. Hence, the $\delta_f$
phases are not `unphysical' in the see-saw model, but they can always be traded
for parameters of $Y_\nu^\dagger Y_\nu$.

One may wonder whether a similar thing occurs in the quark sector. This is
because the physical objects are $Y_u^\dagger Y_u$ and $Y_d^\dagger Y_d$,  and
there is no physical object analogous to the effective neurino mass operator. In
the basis where $Y_u^\dagger Y_u$ is diagonal, we can make the off-diagonal
elements of $Y_d^\dagger Y_d$ real and positive by the transformation
\begin{equation}
 \vec q\,\to\,K_\delta^{-1}\,\vec q\;,
\end{equation}
which does not change $Y_u^\dagger Y_u$. This transformation does not change
the physical parameters but only the unphysical ones. Because it is a symmetry
if we impose simultaneously
\begin{equation}
 Y_u\,\to\,Y_u\,K_\delta\;,\quad
 Y_d\,\to\,Y_d\,K_\delta\;,
\end{equation}
the RGEs have to be invariant under it. This means that to just change the
phases of the off-diagonal elements of $Y_d^\dagger Y_d$ without changing
$Y_u^\dagger Y_u$ is a symmetry of the theory, hence the RGEs in the quark
sector neither depend on the phases of the off-diagonal elements of $Y_d^\dagger
Y_d$ nor the unphysical phases.

\subsection{Extracting Mixing Angles and Phases}
\label{sec:ExtractingMixingAngles}

In the standard parametrization, the mixing angles $\theta_{13}$ 
and $\theta_{23}$ can be chosen to lie between $0$ and $\frac{\pi}{2}$,
and in the lepton sector by reordering the masses, $\theta_{12}$ can be restricted to $0\le\theta_{12}\le\frac{\pi}{4}$.
For the phases the range between $0$ and $2\pi$ is required.
In order to read off the mixing parameters in the generic case, i.e.\ for none
of the angles $\theta_{ij}$ equal to 0 or $\pi/2$, we use the following
procedure:
\begin{enumerate}
 \item $\theta_{13}=\arcsin(|U_{13}|)$.
 \item $\displaystyle \theta_{12}=\left\{\begin{array}{ll}
 \displaystyle \arctan\left(\frac{|U_{12}|}{|U_{11}|}\right) \quad
        & \text{if}\;U_{11}\ne0\\
 \frac{\pi}{2} & \text{else}
 \end{array}\right.$
 \item $\displaystyle \theta_{23}=\left\{\begin{array}{ll}
 \displaystyle \arctan\left(\frac{|U_{23}|}{|U_{33}|}\right) \quad
        & \text{if}\;U_{33}\ne0\\
 \frac{\pi}{2} & \text{else}
 \end{array}\right.$
 \item $\delta_\mu = \arg(U_{23})$
 \item $\delta_\tau = \arg(U_{33})$
 \item \label{step6}$\displaystyle\delta=
 -\arg\left(\frac{\displaystyle\frac{U_{11}^*U_{13}U_{31}U_{33}^*}
        {c_{12}\,c_{13}^2\,c_{23}\,s_{13}}
        +c_{12}\,c_{23}\,s_{13}}
        {s_{12}\,s_{23}}\right)$.
 \item $\delta_e=\arg(e^{\I\delta}\,U_{13})$
 \item $\displaystyle\varphi_1=2\arg(e^{\I\delta_e}\,U_{11}^*)$
 \item \label{step9}$\displaystyle\varphi_2=2\arg(e^{\I\delta_e}\,U_{12}^*)$
\end{enumerate}
Here we used the relation\footnote{There was an error in an earlier version of
this relation. We are grateful to Yang Bai for pointing it out to us.}
\begin{eqnarray}
 \im \left(U_{11}^*U_{13}U_{31}U_{33}^*\right)
 & = &
 c_{12}\,c_{13}^2\, 
 c_{23}\,s_{13}
 \left(e^{-\I\delta}\,s_{12}\,s_{23} - c_{12}\,c_{23}\,s_{13}\right)
 \;.\nonumber
\end{eqnarray}
%which holds for $i,j\in\{1,2,3\}$ and $i\ne j$.
Note that this relation is often used in order to introduce 
the Jarlskog invariants \cite{Jarlskog:1985ht}
\begin{eqnarray}
 J_\mathrm{CP} 
 & = &
 \frac{1}{2} \left| \im (U_{11}^*U_{12}U_{21}U_{22}^*)\right|
 \,=\, 
 \frac{1}{2} \left| \im (U_{11}^*U_{13}U_{31}U_{33}^*)\right|
 \nonumber\\
 & =  &
 \frac{1}{2} \left| \im (U_{22}^*U_{23}U_{32}U_{33}^*)\right|
 \, = \,
 \frac{1}{2}\left|c_{12}\,c_{13}^2\,
    c_{23}\,\sin \delta \,
    s_{12}\,s_{13}\,
    s_{23}\right|\;.
\end{eqnarray}
% For the sake of a better numerical stability, one can choose any of the three
% combinations. In particular, if the modulus of one of the $U_{ij}$ is very
% small, it turns out to be more accurate to choose a combination in which this
% specific $U_{ij}$ does not appear.

Another comment: by definition, $\theta_{13}$ is always positive. Technically
this is achieved by an approriate choice of $\delta$.

\subsection{Remarks on special (degenerate) cases}
\label{sec:SpecialCases}
\subsubsection{Diagonal mass matrices}

If $Y_e$ and $m_\nu$ are diagonal, the phases are mathematically not
well-defined. However, even though one may trade $\delta_e$ for $\varphi_1$,
none of the phases is physical. Hence, \function{MNSParameters} returns only
non-zero $\delta_e$, $\delta_\mu$ and $\delta_\tau$ in this case.\footnote{For
the quark sector, i.e.\ \function{CKMParameters}, a similar thing is not
implemented at the moment\dots}

\subsubsection{Degenerate masses}

In the case of degenerate masses, mixings are undefinded, because by definition
mixings only occur between mass eigenstates which are not fixed in this case.
\function{MNSParameters} will return arbitrary mixing angles and print a
warning.

\subsubsection[$\theta_{13}=0$]{$\boldsymbol{\theta_{13}=0}$}
\fbox{\parbox{0.9\textwidth}{
For a zero CHOOZ angle, the Dirac phase is undefined.  Besides, the
determination of $\varphi_2$ and $\delta_e$ has to be modified.  In
general, we use
\begin{subequations} \label{eq:ReadingOffT13Zero}
\begin{eqnarray}
        \delta &=& 0 \;, \\
        \varphi_2 &=& 2 \arg(e^{\I\delta_\mu} U_{22}^*) \;, \\
        \delta_e &=& \arg(e^{\I\frac{\varphi_2}{2}} U_{12}) \;.
\end{eqnarray}
\end{subequations}
For special values of $\theta_{12}$ and $\theta_{23}$, we apply the
following modifications:
\begin{description}
\item[$\theta_{12}=0$:]
 \begin{subequations}
 \begin{eqnarray}
        \varphi_1 &=& 0 \;, \\
        \delta_e &=& \arg(U_{11}) \;.
 \end{eqnarray}
 \end{subequations}
\item[$\theta_{12}=\pi/2$:]
 This case is not considered yet.
\item[$\theta_{23}=0$:]
 The phases $\delta_e$, $\delta_\mu$, $\varphi_1$ and $\varphi_2$ are
 linearly dependent due to the orthonormality of $U$ (cf.\ the case
 $\theta_{13}=\pi/2$ below), so that we can choose one of them to be
 zero.
 \begin{subequations}
 \begin{eqnarray}
        \varphi_1 &=& 0 \;, \\
        \delta_e &=& \arg(U_{11}) \;, \\
        \varphi_2 &=& 2 \arg(e^{\I\delta_e} U_{12}^*) \;, \\
        \delta_\mu &=& \arg(-U_{21}) \;.
 \end{eqnarray}
 \end{subequations}
\item[$\theta_{23}=\pi/2$:]
 \begin{subequations}
 \begin{eqnarray}
        \varphi_1 &=& 0 \;, \\
        \delta_e &=& \arg(U_{11}) \;, \\
        \varphi_2 &=& 2 \arg(e^{\I\delta_e} U_{12}^*) \;, \\
        \delta_\tau &=& \arg(-e^{\I\frac{\varphi_2}{2}} U_{32}) \;.
 \end{eqnarray}
 \end{subequations}
\end{description}
}}


\subsubsection[$\theta_{13}=\pi/2$]{$\boldsymbol{\theta_{13}=\pi/2}$}
\fbox{\parbox{0.9\textwidth}{
This case often occurs when the neutrino mass hierarchy changes, i.e.\
$\Delta m^2_\mathrm{sol}$ overtakes $\Delta m^2_\mathrm{atm}$ during the
running.  As the rows and columns of the mixing matrix have to be
normalized, it can be written as
\begin{equation} \label{eq:T13Pi2Int}
        U =
        \begin{pmatrix}
        0 & 0 & e^{\I\delta_1} \\
        -e^{\I\varphi_{21}} \sin\theta & e^{\I\varphi_{22}} \cos\theta & 0\\
        -e^{\I\varphi_{31}} \cos\theta &-e^{\I\varphi_{32}} \sin\theta & 0\\
        \end{pmatrix} ,
\end{equation}
where the positions of $\sin$ and $\cos$ and of the minus signs are
arbitrary, of course.  We choose the angle $\theta$ to equal
$\theta_{23}$ and the phase of the 13-element to equal $\delta_e$.  This
implies $\theta_{12}=0$ and $\delta=0$.  Furthermore, from the
orthogonality of the rows we find
$U_{21}^* U_{22} + U_{31}^* U_{32} = 0$, which means that
$\varphi_{21}-\varphi_{22}-\varphi_{31}+\varphi_{32} = n \, 2\pi$
($n\in\mathbbm{Z}$).  Consequently, one of the remaining four phases is
arbitrary, and we choose $\varphi_1=0$.  Comparing
eq.~\eqref{eq:T13Pi2Int} to the standard parametrization now leads to
the parametrization of the MNS matrix for $\theta_{13}=\pi/2$,
\begin{equation} \label{eq:MNSParametrizationT13Pi2}
        U =
        \begin{pmatrix}
        0 & 0 & e^{\I\delta_e} \\
        -e^{\I\delta_\mu} s_{23} & e^{\I(\delta_\mu-\varphi_2/2)}\,c_{23} & 0\\
        -e^{\I\delta_\tau} c_{23} &-e^{\I(\delta_\tau-\varphi_2/2)} s_{23} & 0\\
        \end{pmatrix} .
\end{equation}
Thus, the mixing parameters are determined as follows:
\begin{subequations} \label{eq:ReadingOffT13Pi2}
\begin{eqnarray}
   \delta & = & 0\;,\\
   \delta_e & = & \arg (U_{13})\;,\\
   \theta_{12} & = & 0\;,\\
   \theta_{23} & = & \arctan (|U_{21}/U_{31}|)\;,\\
   \varphi_1 & = & 0\;,\\
   \delta_\mu & = & \arg(-U_{21})\;,\\
   \varphi_2 & = & 2\,(\delta_\mu - \arg(U_{22}))\;,\\
   \delta_\tau & = & \arg(-U_{31})\;.
\end{eqnarray}
\end{subequations}
For $U_{21}=0$ ($\theta_{23}=0$), we use $\delta_\mu = \arg(U_{22})$,
$\varphi_2 = 0$.  For $U_{31}=0$ ($\theta_{23}=\pi/2$), we can also set
$\varphi_2=0$; the phase $\delta_\tau$ is then determined from
$\delta_\tau = \arg(-U_{32})$.
\\ This also affects functions based on \function{MPT3x3MixingParameters[$U$]},
such as \function{MNSParameters} and \function{CKMParameters}.
\dots\emph{under construction}\dots 
\emph{Ist dieser Kommentar noch aktuell? (J.)}
}}


\subsubsection[$\theta_{13}\neq0,\pi/2$, $\theta_{12}=0$ or
$\theta_{23}=0,\pi/2$]{$\boldsymbol{\theta_{13}\neq0,\pi/2}$,
$\boldsymbol{\theta_{12}=0}$ or $\boldsymbol{\theta_{23}=0,\pi/2}$}
\fbox{\parbox{0.9\textwidth}{
For special values of $\theta_{12}$ or $\theta_{23}$, we use the
following modifications after the standard procedure of
Sec.~\ref{sec:ExtractingMixingAngles}:
\begin{description}
\item[$\theta_{23}=0$:]
 \begin{eqnarray} \label{eq:Zerot23}
    \delta_\mu = \arg(e^{\I\frac{\varphi_2}{2}} U_{22}) \;.
 \end{eqnarray}
\item[$\theta_{23}=\pi/2$:]
 \begin{eqnarray} \label{eq:t23Pi2}
    \delta_\tau = \arg(-e^{\I\frac{\varphi_2}{2}} U_{32}) \;.
 \end{eqnarray}
\item[$\theta_{12}=0$:]
 \begin{eqnarray} \label{eq:Zerot12}
    \varphi_2 = 2 \arg(e^{\I\delta_\mu} U_{22}^*) \;.
 \end{eqnarray}
\item[$\theta_{12}=\pi/2$:]
 Not implemented yet.
\end{description}

The combination $\theta_{12}=\theta_{23}=0$ does not require a separate
If-query in the code: In step (9), of the standard procedure,
$\varphi_2$ is set to 0, since $U_{12}=0$.  Afterwards,
Eq.~\eqref{eq:Zerot23} yields $\delta_\mu=\arg(U_{22})$.  Finally,
$\varphi_2$ is again set to 0, this time because of
Eq.~\eqref{eq:Zerot12}.

The case $\theta_{12}=0$, $\theta_{23}=\pi/2$ is also ok: First, step
(9) of the standard procedure yields $\varphi_2=0$ due to $U_{12}=0$.
Next, Eq.~\eqref{eq:t23Pi2} yields the correct value
$\delta_\tau=\arg(-U_{32})$.  Afterwards, Eq.~\eqref{eq:Zerot12} gives
$\varphi_2$ once again, since $\delta_\mu$ has been determined in step
(4) of the standard procedure, so that $e^{\I\delta_\mu} U_{22}^*$ is
real.
}}



\appendix
\section{Theorems on Matrix-Diagonalization}
\subsection*{Hermitian Matrices}
\begin{Theorem}
 Hermitian matrices $M$ can be diagonalized by unitary transformations,
 \begin{equation}
        U^\dagger M U = \diag(M_1,\dots,M_n) \;,
 \end{equation}
 where $U$ is unitary and the eigenvalues $M_i$ are real. The columns of
 $U$ contain the eigenvectors of $M$.
\end{Theorem}
\begin{Proof}
 See the standard textbooks on linear algebra.
\end{Proof}

\subsection*{General Matrices (Biunitary Diagonalization)}
\begin{Theorem} \label{th:BiunitaryDiag}
 A general, non-singular matrix $M$ can be diagonalized by a 
 \emph{biunitary} transformation\index{biunitary},
 \begin{equation} \label{eq:BiunitaryDiag}
        U_\mathrm{L}^\dagger M U_\mathrm{R} = \diag(M_1,\dots,M_n) \;,
 \end{equation}
 if none of the eigenvalues of $M^\dagger M$ equals zero.
 $U_\mathrm{L}$ and $U_\mathrm{R}$ are unitary, and $M_i$ are real and
 positive.
 The matrices $U_\mathrm{L}$ and $U_\mathrm{R}$ can be found by
 determining the unitary transformations which diagonalize $M M^\dagger$
 and $M^\dagger M$, respectively, i.e.\
 \begin{subequations}
 \begin{eqnarray}
        U_\mathrm{L}^\dagger \, M M^\dagger \, U_\mathrm{L} &=&
         \diag(M_1^2,\dots,M_n^2) \;,
 \\
        U_\mathrm{R}^\dagger \, M^\dagger M \, U_\mathrm{R} &=&
         \diag(M_1^2,\dots,M_n^2) \;.
        \label{eq:DefUR}
 \end{eqnarray}
 \end{subequations}
\end{Theorem}
\begin{Proof}
 Define
 \begin{equation}\label{eq:Mohapatra4.75}
        H^2 := M M^\dagger \;,
 \end{equation}
 which is obviously Hermitian and can therefore be diagonalized by a
 unitary tranformation,
 \begin{equation}
        U_\mathrm{L}^\dagger \, M M^\dagger \, U_\mathrm{L} =
         \diag(M_1^2,\dots,M_n^2) =: D^2 \;,
 \end{equation}
 where $M_i$ are real and positive.
 Define $D$ as the diagonal matrix containing the squareroots of $D^2$.
 Then obviously
 \begin{equation} \label{eq:DefH}
        H := U_\mathrm{L} D U_\mathrm{L}^\dagger 
 \end{equation}
 satisfies \fref{eq:Mohapatra4.75}. With $V:=H^{-1}M$, 
 which is unitary because
 \begin{equation}
        V^\dagger V \stackrel{H^\dagger=H}{=}
        M^\dagger H^{-1} H^{-1} M \stackrel{\eqref{eq:Mohapatra4.75}}{=}
        M^\dagger (M M^\dagger)^{-1} M = \mathbbm{1} \;,
 \end{equation}
 we find
 \begin{equation} \label{eq:ProofCompl}
        M = H V \stackrel{\eqref{eq:DefH}}{=}
        U_\mathrm{L} D U_\mathrm{R}^\dagger \;,
 \end{equation}
 where $U_\mathrm{R} := V^\dagger U_\mathrm{L}$ is unitary, so that
 \fref{eq:BiunitaryDiag} is proven. Furthermore, $U_\mathrm{R}$
 diagonalizes $M^\dagger M$, since
 \begin{equation}
        U_\mathrm{R}^\dagger M^\dagger M U_\mathrm{R}
        \stackrel{\eqref{eq:ProofCompl}}{=}
        U_\mathrm{R}^\dagger \, U_\mathrm{R} D U_\mathrm{L}^\dagger \,
         U_\mathrm{L} D U_\mathrm{R}^\dagger \, U_\mathrm{R} = D^2 \;,
 \end{equation}
 which proves \fref{eq:DefUR}.
\end{Proof}

\subsection*{Symmetric Matrices}
\begin{Theorem} \label{th:SymmetricMatrixDiag}
 Complex symmetric matrices can be diagonalized by a unitary matrix $U$,
 \begin{equation}
        U^T M U = \diag(M_1,\dots,M_n) := D \;,
 \end{equation}
 where
 \begin{equation}
        U^\dagger \, M^\dagger M \, U = D^2 \;,
 \end{equation}
 i.e.\ the real numbers $M_i$ are the square roots of the eigenvalues of
 $M^\dagger M$.
\end{Theorem}
\begin{Proof}
 From theorem \ref{th:BiunitaryDiag} we know that 
 \begin{equation}
        M = U_\mathrm{L} D U_\mathrm{R}^\dagger \;,
 \end{equation}
 where $U_\mathrm{L}$, $U_\mathrm{R}$ and $D$ are uniquely 
 determined.\footnote{Note that $U_\mathrm{L}$, $U_\mathrm{R}$ are
 not always unique: If the eigenvalues
 of $D$ are degenerate, there exist matrices $U$ which diagonalize
 $M^\dagger M$, i.e. $U^\dagger\,M^\dagger M\, U=D$, which however
 do not diagonalize $M$. In this case, $M$ can still be diagonalized,
 but the matrix which does the job can not simply be obtained by calculating
 the eigenvectors of $M^\dagger M$.}
 As $M$ is symmetric, it follows that
 \begin{equation}
        M = M^T = U_\mathrm{R}^* D U_\mathrm{L}^T \;.
 \end{equation}
 On the other hand, we can view the last equation as the diagonalization
 of $M^T$, which is uniquely determined as well according to theorem
 \ref{th:BiunitaryDiag}.  Hence, we conclude 
 $U_\mathrm{L}=U_\mathrm{R}^*$, which completes the proof if we set
 $U:=U_\mathrm{R}$ and take into account \fref{eq:DefUR}.
\end{Proof}

\section{Some thoughts during implementation}

\subsection*{$\theta_{13}$ problem}

if $\theta_{13}=\pi/2$ is encountered. there are `angle and phase moduli', i.e.\
combinations of angles and phases which do not change the resulting unitary
matrix when plugged in. This is obvious since the only non-trivial information
is encoded in the bottom-left $2\times2$ sub-block (and the phase of $U_{13}$).
This sub-block can be parametrized by one angle.

\subsection*{degenerate eigenvalues}

Mathematica has the function \function{SingularValueDecomposition[$m$]} which
returns three matrices $U$, $V$ and $W$ such that $W$ is diagonal, $U$ and
$V$ are unitary and 
\begin{equation}\label{eq:SingularValueDecomposition}
 m\,=\,U\,W\,V^\dagger\;.
\end{equation} 
\begin{itemize}
 \item This function is obviously very  useful for matrix diagonalization (it almost
does everything we need). 
 \item For non-degenerate eigenvalues and symmetric $m$ we
have due to \ref{th:SymmetricMatrixDiag} that $U^*=V$.
 \item However, for degenerate mass eigenvalues,
\function{SingularValueDecomposition[$m$]} returns still $U$, $V$ and $W$ such
that \eqref{eq:SingularValueDecomposition} holds, but $U^*\ne V$ in general.
In this case, one (combination of) mixing angle(s) is unphysical. $U^*$  and $V$
differ by an unphysical rotation. If one takes the average, one can construct a
matrix $V$ such that $V^TM\,V$ is diagonal.
 \item Unfortunately, when trying to identify these unphysical parameters, one
 runs into the $\theta_{13}=\pi/2$ problem. So, this has to be solved first.
\end{itemize}

\end{document}
